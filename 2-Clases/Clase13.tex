\documentclass[a4,11pt]{aleph-notas}

% -- Paquetes adicionales 
\usepackage{enumitem}
\usepackage{url}
\usepackage{array}
\usepackage{booktabs}
\hypersetup{
    urlcolor=blue,
    linkcolor=blue,
}

% -- Datos 
\institucion{Facultad de Ciencias Exactas, Naturales y Ambientales}
\carrera{Ciencia de Datos}
\asignatura{Aprendizaje Automático Inicial}
\tema{Clase 13: Perceptrón Sigmoide y Perceptrón Multiclase}
\autor{Andrés Merino}
\fecha{Periodo 2025-2}

\logouno[0.14\textwidth]{Logos/logoPUCE_04_ac}
\definecolor{colortext}{HTML}{0030A1}
\definecolor{colordef}{HTML}{0030A1}
\fuente{montserrat}

\begin{document}

\encabezado
% En todo el documento, las indicaciones deben ser simples y directas, con una sola oración.

%%%%%%%%%%%%%%%%%%%%%%%%%%%%%%%%%%%%%%%%
\section*{Resultado de Aprendizaje}
%%%%%%%%%%%%%%%%%%%%%%%%%%%%%%%%%%%%%%%%

%%%%%%%%%%%%%%%%%%%%%%%%%%%%%%%%%%%%%%%%
\subsection*{RdA de la asignatura:}
% Se toma uno de los siguientes
\begin{itemize}[leftmargin=*]
    \item \textbf{RdA 2:} Aplicar modelos de aprendizaje automático supervisado y no supervisado, así como su validación y optimización, en la resolución de problemas tanto reales como simulados.
\end{itemize}

%%%%%%%%%%%%%%%%%%%%%%%%%%%%%%%%%%%%%%%%
\subsection*{RdA de la clase:}
% Máximo 3 resultados
\begin{itemize}[leftmargin=*]
    \item Comprender las funciones de combinación, activación y pérdida del perceptrón sigmoide y multiclase.
    \item Derivar los gradientes asociados a las funciones de pérdida de cada perceptrón.
    \item Implementar en Python los algoritmos del perceptrón sigmoide y multiclase.
\end{itemize}

%%%%%%%%%%%%%%%%%%%%%%%%%%%%%%%%%%%%%%%%
\section*{Introducción}
%%%%%%%%%%%%%%%%%%%%%%%%%%%%%%%%%%%%%%%%

%%%%%%%%%%%%%%%%%%%%%%%%%%%%%%%%%%%%%%%%
\paragraph{Pregunta inicial:} 
Si una neurona se equivoca al clasificar, ¿cómo debería medir qué tan grave fue su error para ajustar mejor sus conexiones la próxima vez?

%%%%%%%%%%%%%%%%%%%%%%%%%%%%%%%%%%%%%%%%
\section*{Desarrollo}
%%%%%%%%%%%%%%%%%%%%%%%%%%%%%%%%%%%%%%%%

%%%%%%%%%%%%%%%%%%%%%%%%%%%%%%%%%%%%%%%%
\subsection*{Actividad 1: Perceptrón Sigmoide}

En esta actividad se explorará el funcionamiento del perceptrón sigmoide mediante clase magistral y exploración de cuaderno de Jupyter, abordando sus funciones de combinación, activación sigmoide, pérdida por entropía cruzada binaria y derivación del gradiente para el entrenamiento.

\paragraph{¿Cómo lo haremos?}  
\begin{itemize}[leftmargin=*]
    \item \textbf{Clase magistral:}
    Se explicará la teoría que sustenta el perceptrón sigmoide:
    \begin{itemize}
        \item Funciones de combinación y activación (sigmoide).
        \item Pérdida por entropía cruzada binaria.
        \item Derivación del gradiente para los pesos y el sesgo.
    \end{itemize}
    
    \item \textbf{Implementación en Python:} Los estudiantes accederán a un cuaderno de Jupyter previamente preparado.
    \begin{quote}
        Enlace al cuaderno: \href{https://github.com/andres-merino/AprendizajeAutomaticoInicial-05-N0105/blob/main/2-Notebooks/13-1-Perceptron-Sigmoide.ipynb}{13-1-Perceptron-Sigmoide.ipynb}.
    \end{quote}
    
    \item \textbf{Materiales de apoyo:} 
    Se utilizará el documento \href{https://andres-merino.github.io/AprendizajeAutomaticoInicial-05-N0105/2-Resumenes/Resumen13.pdf}{Resumen13.pdf}
    
\end{itemize}

%%%%%%%%%%%%%%%%%%%%%%%%%%%%%%%%%%%%%%%%
\subsection*{Actividad 2: Perceptrón Multiclase}

Esta actividad se centra en el perceptrón multiclase mediante clase magistral y exploración de cuaderno de Jupyter, destacando las funciones de combinación, activación softmax, pérdida categórica cruzada y derivación del gradiente para entrenamiento en problemas multiclase.

\paragraph{¿Cómo lo haremos?}  
\begin{itemize}[leftmargin=*]
    \item \textbf{Clase magistral:}
    Se cubrirá la teoría que sustenta el perceptrón multiclase:
    \begin{itemize}
        \item Funciones de combinación y activación (softmax).
        \item Pérdida categórica cruzada.
        \item Derivación del gradiente para los pesos y sesgos.
    \end{itemize}
    
    \item \textbf{Implementación en Python:} Los estudiantes accederán a un cuaderno de Jupyter previamente preparado.
    \begin{quote}
        Enlace al cuaderno: \href{https://github.com/andres-merino/AprendizajeAutomaticoInicial-05-N0105/blob/main/2-Notebooks/13-2-Perceptron-Multiclase.ipynb}{13-2-Perceptron-Multiclase.ipynb}.
    \end{quote}
    
    \item \textbf{Materiales de apoyo:} 
    Se utilizará el documento \href{https://andres-merino.github.io/AprendizajeAutomaticoInicial-05-N0105/2-Resumenes/Resumen13.pdf}{Resumen13.pdf}
    
\end{itemize}

%%%%%%%%%%%%%%%%%%%%%%%%%%%%%%%%%%%%%%%%
\section*{Cierre}
%%%%%%%%%%%%%%%%%%%%%%%%%%%%%%%%%%%%%%%%

\paragraph{Verificación de aprendizaje:}
\begin{enumerate}[leftmargin=*]
    \item ¿Cuál es el propósito de la función de activación sigmoide en el perceptrón sigmoide y qué rango de valores produce?
    
    \item ¿Qué es la función softmax y por qué se utiliza junto con la pérdida categórica cruzada en problemas multiclase?
\end{enumerate}

\paragraph{Tarea:}
Desarrollar los ejercicios planteados en el siguiente cuaderno:
\begin{quote}
    Enlace al cuaderno: \href{https://colab.research.google.com/github/andres-merino/AprendizajeAutomaticoInicial-05-N0105/blob/main/2-Ejercicios/07-Perceptron.ipynb}{07-Perceptron.ipynb}.
\end{quote}

\paragraph{Pregunta de investigación:}  
\begin{enumerate}[leftmargin=*]
    \item ¿Cómo se compara el perceptrón multiclase con otros métodos de clasificación, como regresión logística multinomial?
    \item En el perceptrón multiclase, ¿por qué se utiliza la función softmax junto con la pérdida categórica cruzada? ¿Cómo trabajan juntas para mejorar la clasificación?
\end{enumerate}
    
\paragraph{Para la próxima clase:} 
Investigar qué son los frameworks de aprendizaje profundo como PyTorch y TensorFlow, y responder a las siguientes preguntas:
\begin{itemize}[leftmargin=*]
    \item ¿Qué es un framework? ¿Qué es una API en este sentido?
    \item ¿Qué características y ventajas ofrecen estos frameworks para implementar redes neuronales?
    \item ¿Cuáles son las diferencias clave entre PyTorch y TensorFlow en términos de facilidad de uso, flexibilidad y rendimiento?
    \item ¿Qué ejemplos prácticos existen del uso de PyTorch y TensorFlow en la resolución de problemas de clasificación y predicción?
    \item ¿Cómo podrías instalar y configurar uno de estos frameworks en tu entorno de trabajo?
\end{itemize}

\end{document}