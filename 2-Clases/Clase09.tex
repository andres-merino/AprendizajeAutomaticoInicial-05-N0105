\documentclass[a4,11pt]{aleph-notas}

% -- Paquetes adicionales 
\usepackage{enumitem}
\usepackage{url}
\usepackage{array}
\usepackage{booktabs}
\hypersetup{
    urlcolor=blue,
    linkcolor=blue,
}

% -- Datos 
\institucion{Facultad de Ciencias Exactas, Naturales y Ambientales}
\carrera{Ciencia de Datos}
\asignatura{Aprendizaje Automático Inicial}
\tema{Clase 09: Agrupamiento $k$-Means}
\autor{Andrés Merino}
\fecha{Periodo 2025-2}

\logouno[0.14\textwidth]{Logos/logoPUCE_04_ac}
\definecolor{colortext}{HTML}{0030A1}
\definecolor{colordef}{HTML}{0030A1}
\fuente{montserrat}

\begin{document}

\encabezado

%%%%%%%%%%%%%%%%%%%%%%%%%%%%%%%%%%%%%%%%
\section*{Actividad previa}
%%%%%%%%%%%%%%%%%%%%%%%%%%%%%%%%%%%%%%%%

\begin{itemize}
    \item Realizar el cuestionario de este \href{https://gemini.google.com/share/475317ea89c6}{enlace} sobre la sección 12.1 del artículo Beginning with machine learning: a comprehensive primer.
\end{itemize}

%%%%%%%%%%%%%%%%%%%%%%%%%%%%%%%%%%%%%%%%
\section*{Resultado de Aprendizaje}
%%%%%%%%%%%%%%%%%%%%%%%%%%%%%%%%%%%%%%%%

%%%%%%%%%%%%%%%%%%%%%%%%%%%%%%%%%%%%%%%%
\subsection*{RdA de la asignatura:}
% Se toma uno de los siguientes
\begin{itemize}[leftmargin=*]
    \item \textbf{RdA 2:} Aplicar modelos de aprendizaje automático supervisado y no supervisado, así como su validación y optimización, en la resolución de problemas tanto reales como simulados.
\end{itemize}

%%%%%%%%%%%%%%%%%%%%%%%%%%%%%%%%%%%%%%%%
\subsection*{RdA de la clase:}
% Máximo 3 resultados
\begin{itemize}[leftmargin=*]
    \item Comprender el funcionamiento del algoritmo $k$-Means, su pseudocódigo y criterios para la selección del valor de $k$.
    \item Explorar métodos derivados del algoritmo $k$-Means y sus aplicaciones prácticas.
    \item Implementar el algoritmo $k$-Means en un entorno práctico, evaluando resultados.
\end{itemize}

%%%%%%%%%%%%%%%%%%%%%%%%%%%%%%%%%%%%%%%%
\section*{Introducción}
%%%%%%%%%%%%%%%%%%%%%%%%%%%%%%%%%%%%%%%%

%%%%%%%%%%%%%%%%%%%%%%%%%%%%%%%%%%%%%%%%
\paragraph{Pregunta inicial:} 
Imagina que eres el planificador urbano de la región de Kanto y tienes presupuesto para construir exactamente tres Centros Pokémon. ¿En qué coordenadas del mapa los colocarían para que la distancia promedio que cualquier entrenador tenga que caminar para curar a su equipo sea la mínima posible?

%%%%%%%%%%%%%%%%%%%%%%%%%%%%%%%%%%%%%%%%
\section*{Desarrollo}
%%%%%%%%%%%%%%%%%%%%%%%%%%%%%%%%%%%%%%%%

%%%%%%%%%%%%%%%%%%%%%%%%%%%%%%%%%%%%%%%%
\subsection*{Actividad 1: Introducción al algoritmo $k$-Means}

Se introducirá a los estudiantes el algoritmo $k$-Means mediante visualización de video y clase magistral, explicando su concepto, pseudocódigo y la importancia del valor de $k$, seguido de discusión sobre aplicaciones prácticas.

\paragraph{¿Cómo lo haremos?}  
\begin{itemize}[leftmargin=*]
    \item \textbf{Video introductorio:} 
    Visualización del video que explica de manera visual y dinámica el concepto y pasos del algoritmo $k$-Means.
    \begin{quote}
        Enlace al video: \href{https://youtu.be/2kfY0R34Dy0?si=60bbeVyyJ1g7WX63}{K Means Clustering}.
    \end{quote}
    
    \item \textbf{Clase magistral:} 
    Presentación de los elementos teóricos principales:
    \begin{itemize}
        \item Concepto de $k$-Means.
        \item Pseudocódigo del algoritmo.
        \item Importancia del valor de $k$.
    \end{itemize}
    
    \item \textbf{Materiales de apoyo:} 
    Se utilizará el documento \href{https://andres-merino.github.io/AprendizajeAutomaticoInicial-05-N0105/2-Resumenes/Resumen09.pdf}{Resumen09.pdf}
\end{itemize}

%%%%%%%%%%%%%%%%%%%%%%%%%%%%%%%%%%%%%%%%
\subsection*{Actividad 2: Implementación práctica de $k$-Means}

Los estudiantes utilizarán un cuaderno de Google Colab mediante exploración práctica para implementar $k$-Means con datos simulados, experimentando con diferentes parámetros del algoritmo como el número de clusters y el método de inicialización.

\paragraph{¿Cómo lo haremos?}  
\begin{itemize}[leftmargin=*]
    \item \textbf{Explicación de parámetros:} 
    Se explicará cómo los parámetros del algoritmo (número de clusters, método de inicialización) afectan los resultados del agrupamiento.
    
    \item \textbf{Implementación en Python:} Los estudiantes accederán a un cuaderno de Jupyter previamente preparado.
    \begin{quote}
        Enlace al cuaderno: \href{https://github.com/andres-merino/AprendizajeAutomaticoInicial-05-N0105/blob/main/2-Notebooks/09-Agrupamiento-kMeans.ipynb}{09-Agrupamiento-kMeans.ipynb}.
    \end{quote}
    
    \item \textbf{Experimentación:} 
    Modifique el código para:
    \begin{itemize}[leftmargin=*]
        \item Utilizar $k=3$ y comparar el resultado con $k=4$.
        \item Cambiar el método de inicialización a 'random' y observar los efectos en la convergencia.
    \end{itemize}
    
\end{itemize}

%%%%%%%%%%%%%%%%%%%%%%%%%%%%%%%%%%%%%%%%
\section*{Cierre}
%%%%%%%%%%%%%%%%%%%%%%%%%%%%%%%%%%%%%%%%

\paragraph{Verificación de aprendizaje:} 
\begin{enumerate}[leftmargin=*]
    \item ¿Cuáles son los pasos principales del algoritmo $k$-Means?
    
    \item ¿Qué significa minimizar la inercia en el contexto de $k$-Means y cómo se relaciona con la calidad del agrupamiento?
    
    \item ¿Cómo afecta la elección del valor de $k$ a los resultados del algoritmo y qué métodos existen para seleccionar un valor adecuado?
\end{enumerate}

\paragraph{Preguntas tipo entrevista:} 
\begin{enumerate}[leftmargin=*]
    \item Aplicaste $k$-Means a datos de clientes y obtuviste 3 clusters, pero visualmente los datos parecen tener formas alargadas y no circulares. ¿Por qué $k$-Means no funciona bien aquí y qué alternativas propondrías?
    
    \item Ejecuté $k$-Means 10 veces con los mismos datos y obtuve resultados ligeramente diferentes cada vez. ¿Cómo lo solventas?
\end{enumerate}

\paragraph{Tarea:}
Desarrollar los ejercicios planteados en el siguiente cuaderno y entregarlo por el aula virtual:
\begin{quote}
    Enlace al cuaderno: \href{https://colab.research.google.com/github/andres-merino/AprendizajeAutomaticoInicial-05-N0105/blob/main/2-Ejercicios/04-Agrupamiento-kMeans.ipynb}{04-Agrupamiento-kMeans.ipynb}.
\end{quote}
Adicional, visualizar el siguiente video:
\begin{quote}
    Enlace al video: \href{https://youtu.be/mICySHB0fh4?si=YuMBh5uQs4xyRt5u}{K-means (o K-medias) para detección de Clusters}.
\end{quote}

\paragraph{Pregunta de investigación:}  
\begin{enumerate}[leftmargin=*]
    \item ¿Qué alternativas existen al algoritmo $k$-Means y en qué escenarios son más útiles?
    \begin{quote}
        Ver estos dos videos: \href{https://www.youtube.com/watch?v=RDZUdRSDOok}{Clustering with DBSCAN}; \href{https://www.youtube.com/watch?v=HMis89lGdkA}{Identifica Clusters con DBSCAN}.
    \end{quote}
    \item ¿Cómo se podrían mejorar los resultados de $k$-Means en presencia de datos con formas no esféricas?
    \item ¿Qué otras métricas se pueden usar para evaluar la calidad de los clusters generados?
\end{enumerate}

\paragraph{Para la próxima clase:}
\begin{itemize}
    \item La siguiente sesión se destinará para el desarrollo del Reto 1.
    \item La sesión subsiguiente se tendrá evaluación.
    \item Realizar la Clase invertida: Aprendizaje Supervisado, disponible en el aula virtual y aquí: \href{https://andres-merino.github.io/AprendizajeAutomaticoInicial-05-N0105/2-ClaseInvertida/02Est-AprendizajeSupervisado.pdf}{02Est-AprendizajeSupervisado.pdf}.

\end{itemize}

 

\end{document}