\documentclass[a4,11pt]{aleph-notas}
% Se puede ver la documentación aquí: 
% https://github.com/alephsub0/LaTeX_aleph-notas

% -- Paquetes adicionales 
\usepackage{enumitem}
\usepackage{url}
\usepackage{array}
\usepackage{booktabs}
\usepackage{longtable}
\usepackage{environ}  % Para definir nuevos entornos
\hypersetup{
    urlcolor=blue,
    linkcolor=blue,
}

% Definición del nuevo ambiente
\NewEnviron{materiales}{%
    \begin{minipage}{12cm}
        \vspace{0.1mm}
        \begin{itemize}[leftmargin=*]
            \BODY % Contenido de la lista
        \end{itemize}
        \vspace{1mm}
    \end{minipage}
}


% -- Datos 
\institucion{Escuela de Ciencias Físicas y Matemática}
\carrera{Ciencia de Datos}
\asignatura{Aprendizaje Automático Inicial}
\tema{Clase 09: Agrupamiento k-Means}
\autor{Andrés Merino}
\fecha{Semestre 2025-1}

\logouno[0.14\textwidth]{Logos/logoPUCE_04_ac}
\definecolor{colortext}{HTML}{0030A1}
\definecolor{colordef}{HTML}{0030A1}
\fuente{montserrat}


% -- Comandos para tablas
\usepackage{listings}
\input{listings-python.prf}

\usepackage[spanish,onelanguage,vlined,linesnumbered]{algorithm2e}

% -- Comandos adicionales
\newtcolorbox{pscodigo}
    {icono=\faCogs,color=lightgray,postit,top=-1.5mm,bottom=-1.5mm}
    
\definecolor{colcod}{RGB}{174,218,255}
\newtcolorbox{pycodigo}
    {icono=\faKeyboardO, color=colcod, postit, 
    top=-2mm, bottom=-2mm, 
    extras first={bottom=0mm},
    extras last={top=0mm},
    extras middle={top=0mm,bottom=0mm},
    }

\lstloadlanguages{Python}
\lstset{
  language=Python,
  basicstyle=\small\sffamily,
  stringstyle=\color[HTML]{933797},
  commentstyle=\color[HTML]{228B22}\sffamily,
  emph={[2]from,import,pass,return}, emphstyle={[2]\color[HTML]{DD52F0}},
  emph={[3]range}, emphstyle={[3]\color[HTML]{D17032}},
  emph={[4]for,in,def}, emphstyle={[4]\color{blue}},
  showstringspaces=false,
  breaklines=true,
  prebreak=\mbox{{\color{gray}\tiny$\searrow$}},
  xleftmargin=3pt,
  inputencoding=utf8,
  extendedchars=true,
  columns=fullflexible,
  literate={á}{{\'a}}1 {é}{{\'e}}1 {í}{{\'i}}1 {ó}{{\'o}}1 {ú}{{\'u}}1,
}


\SetKwFunction{concat}{Concatenar}
\SetKwProg{Fn}{Función}{\string:}{}
\SetKwFunction{ult}{Ultimo}
\SetKwFunction{pri}{Primero}
\SetKwFunction{sinul}{SinUltimo}
\SetKw{Salir}{Salir}

\newcommand{\fuentecomentario}[1]{\scriptsize\ttfamily #1}
\SetCommentSty{fuentecomentario}
\SetAlFnt{\footnotesize}


\begin{document}

\encabezado


%%%%%%%%%%%%%%%%%%%%%%%%%%%%%%%%%%%%%%%%
\section*{Resultado de Aprendizaje}
%%%%%%%%%%%%%%%%%%%%%%%%%%%%%%%%%%%%%%%%

%%%%%%%%%%%%%%%%%%%%%%%%%%%%%%%%%%%%%%%%
\subsection*{RdA de la asignatura:}
\begin{itemize}[leftmargin=*]
    \item \textbf{RdA 2:} 
    Aplicar modelos de aprendizaje automático supervisado y no supervisado, así como su validación y optimización, en la resolución de problemas tanto reales como simulados.
\end{itemize}

\subsection*{Resultados específicos:}
\begin{itemize}[leftmargin=*]
    \item Comprender el funcionamiento del algoritmo k-Means y su pseudocódigo.
    \item Analizar criterios para la selección del valor de \( k \).
    \item Explorar métodos derivados del algoritmo k-Means y sus aplicaciones prácticas.
    \item Implementar el algoritmo k-Means en un entorno práctico, evaluando resultados.
\end{itemize}

%%%%%%%%%%%%%%%%%%%%%%%%%%%%%%%%%%%%%%%%
\section*{Introducción}
%%%%%%%%%%%%%%%%%%%%%%%%%%%%%%%%%%%%%%%%

%%%%%%%%%%%%%%%%%%%%%%%%%%%%%%%%%%%%%%%%
\paragraph{Pregunta inicial:} 
¿Qué significa agrupar datos y cómo podríamos identificar patrones en un conjunto de datos sin etiquetas?

%%%%%%%%%%%%%%%%%%%%%%%%%%%%%%%%%%%%%%%%
\section*{Desarrollo}
%%%%%%%%%%%%%%%%%%%%%%%%%%%%%%%%%%%%%%%%

%%%%%%%%%%%%%%%%%%%%%%%%%%%%%%%%%%%%%%%%
\subsection*{Actividad 1: Introducción al algoritmo k-Means}
%%%%%%%%%%%%%%%%%%%%%%%%%%%%%%%%%%%%%%%%

\paragraph{¿Cómo lo haremos?}  
\begin{itemize}[leftmargin=*]
    \item \textbf{Video introductorio:} 
    Visualización del video \href{https://youtu.be/2kfY0R34Dy0?si=60bbeVyyJ1g7WX63}{K Means Clustering}, que explica de manera visual y dinámica el concepto y pasos del algoritmo k-Means.
    \item \textbf{Clase magistral:} 
    Presentación de los elementos teóricos principales:
    \begin{itemize}
        \item Concepto de k-Means.
        \item Pseudocódigo del algoritmo.
        \item Importancia del valor de \(k\).
    \end{itemize}
    \item \textbf{Discusión:}
    Reflexión grupal sobre cómo se aplicaría k-Means en problemas reales. Los estudiantes compartirán ideas sobre escenarios prácticos.
\end{itemize}

\paragraph{Verificación de aprendizaje:}  
\begin{itemize}[leftmargin=*]
    \item ¿Cuáles son los pasos principales del algoritmo k-Means?
    \item ¿Qué significa minimizar la inercia en el contexto de k-Means?
    \item ¿Qué ocurre si seleccionamos un valor incorrecto de \(k\)?
\end{itemize}

%%%%%%%%%%%%%%%%%%%%%%%%%%%%%%%%%%%%%%%%
\subsection*{Actividad 2: Implementación práctica de k-Means}
%%%%%%%%%%%%%%%%%%%%%%%%%%%%%%%%%%%%%%%%

\paragraph{¿Cómo lo haremos?}  
\begin{itemize}[leftmargin=*]
    \item \textbf{Preparación:} 
    Los estudiantes accederán a un cuaderno de Google Colab preparado para esta actividad. El cuaderno incluye una implementación básica de k-Means con datos simulados.
    \begin{quote}
        Enlace al cuaderno: \href{https://colab.research.google.com/github/andres-merino/AprendizajeAutomaticoInicial-05-N0105/blob/main/2-Notebooks/09-Agrupamiento-kMeans.ipynb}{09-Agrupamiento-kMeans.ipynb}.
    \end{quote}
    \item \textbf{Experimentación:} 
    Los estudiantes cambiarán los parámetros del algoritmo, como el número de clusters (\texttt{n\_clusters}) y el método de inicialización (\texttt{init}), para observar cómo afectan los resultados.
\end{itemize}

\begin{ejer}
Modifique el código para:
\begin{enumerate}[leftmargin=*]
    \item Utilizar \(k = 3\) y comparar el resultado con \(k = 4\).
    \item Cambiar el método de inicialización a \texttt{'random'} y observar los efectos en la convergencia.
\end{enumerate}
\end{ejer}

\paragraph{Verificación de aprendizaje:}  
\begin{itemize}[leftmargin=*]
    \item ¿Cómo cambia la agrupación al modificar \(k\)?
    \item ¿Qué ventajas tiene la inicialización \texttt{k-means++} sobre \texttt{random}?
    \item ¿Cómo interpretarías los centroides obtenidos?
\end{itemize}

%%%%%%%%%%%%%%%%%%%%%%%%%%%%%%%%%%%%%%%%
\section*{Cierre}
%%%%%%%%%%%%%%%%%%%%%%%%%%%%%%%%%%%%%%%%

\paragraph{Tarea:}
    Desarrollar los ejercicios planteados en el siguiente cuaderno y entregarlo por el aula virtual:
    \begin{quote}
        Enlace al cuaderno: \href{https://colab.research.google.com/github/andres-merino/AprendizajeAutomaticoInicial-05-N0105/blob/main/2-Ejercicios/03-Agrupamiento-Jerarquico.ipynb}{04-Agrupamiento-kMeans.ipynb}.
    \end{quote}
    Adicional, visualizar el siguiente video:
    \begin{quote}
        Enlace al video: \href{https://youtu.be/mICySHB0fh4?si=YuMBh5uQs4xyRt5u}{K-means (o K-medias) para detección de Clusters}.
    \end{quote}


\paragraph{Pregunta de investigación:}  
\begin{enumerate}[leftmargin=*]
    \item ¿Qué alternativas existen al algoritmo k-Means y en qué escenarios son más útiles?
    \begin{quote}
        Ver estos dos videos: \href{https://www.youtube.com/watch?v=RDZUdRSDOok}{Clustering with DBSCAN}; \href{https://www.youtube.com/watch?v=HMis89lGdkA}{Identifica Clusters con DBSCAN}
    \end{quote}
    \item ¿Cómo se podrían mejorar los resultados de k-Means en presencia de datos con formas no esféricas?
    \item ¿Qué otras métricas se pueden usar para evaluar la calidad de los clusters generados?
\end{enumerate}

\paragraph{Para la próxima clase:}  
Realizar la actividad de Clase invertida sobre Aprendizaje Supervisado, disponible aquí: \href{https://andres-merino.github.io/AprendizajeAutomaticoInicial-05-N0105/2-ClaseInvertida/02Est-AprendizajeSupervisado.pdf}{02Est-AprendizajeSupervisado.pdf}.

\end{document}