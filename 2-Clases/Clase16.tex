\documentclass[a4,11pt]{aleph-notas}
% Se puede ver la documentación aquí: 
% https://github.com/alephsub0/LaTeX_aleph-notas

% -- Paquetes adicionales 
\usepackage{enumitem}
\usepackage{url}
\usepackage{array}
\usepackage{booktabs}
\usepackage{longtable}
\usepackage{environ}  % Para definir nuevos entornos
\hypersetup{
    urlcolor=blue,
    linkcolor=blue,
}

% Definición del nuevo ambiente
\NewEnviron{materiales}{%
    \begin{minipage}{12cm}
        \vspace{0.1mm}
        \begin{itemize}[leftmargin=*]
            \BODY % Contenido de la lista
        \end{itemize}
        \vspace{1mm}
    \end{minipage}
}


% -- Datos 
\institucion{Escuela de Ciencias Físicas y Matemática}
\carrera{Ciencia de Datos}
\asignatura{Aprendizaje Automático Inicial}
\tema{Clase 16: Árboles de decisión}
\autor{Andrés Merino}
\fecha{Semestre 2025-1}

\logouno[0.14\textwidth]{Logos/logoPUCE_04_ac}
\definecolor{colortext}{HTML}{0030A1}
\definecolor{colordef}{HTML}{0030A1}
\fuente{montserrat}


% -- Comandos para tablas
\usepackage{listings}
\input{listings-python.prf}

\usepackage[spanish,onelanguage,vlined,linesnumbered]{algorithm2e}

% -- Comandos adicionales
\newtcolorbox{pscodigo}
    {icono=\faCogs,color=lightgray,postit,top=-1.5mm,bottom=-1.5mm}
    
\definecolor{colcod}{RGB}{174,218,255}
\newtcolorbox{pycodigo}
    {icono=\faKeyboardO, color=colcod, postit, 
    top=-2mm, bottom=-2mm, 
    extras first={bottom=0mm},
    extras last={top=0mm},
    extras middle={top=0mm,bottom=0mm},
    }

\lstloadlanguages{Python}
\lstset{
  language=Python,
  basicstyle=\small\sffamily,
  stringstyle=\color[HTML]{933797},
  commentstyle=\color[HTML]{228B22}\sffamily,
  emph={[2]from,import,pass,return}, emphstyle={[2]\color[HTML]{DD52F0}},
  emph={[3]range}, emphstyle={[3]\color[HTML]{D17032}},
  emph={[4]for,in,def}, emphstyle={[4]\color{blue}},
  showstringspaces=false,
  breaklines=true,
  prebreak=\mbox{{\color{gray}\tiny$\searrow$}},
  xleftmargin=3pt,
  inputencoding=utf8,
  extendedchars=true,
  columns=fullflexible,
  literate={á}{{\'a}}1 {é}{{\'e}}1 {í}{{\'i}}1 {ó}{{\'o}}1 {ú}{{\'u}}1,
}


\SetKwFunction{concat}{Concatenar}
\SetKwProg{Fn}{Función}{\string:}{}
\SetKwFunction{ult}{Ultimo}
\SetKwFunction{pri}{Primero}
\SetKwFunction{sinul}{SinUltimo}
\SetKw{Salir}{Salir}

\newcommand{\fuentecomentario}[1]{\scriptsize\ttfamily #1}
\SetCommentSty{fuentecomentario}
\SetAlFnt{\footnotesize}


\begin{document}

\encabezado


%%%%%%%%%%%%%%%%%%%%%%%%%%%%%%%%%%%%%%%%
\section*{Resultado de Aprendizaje}
%%%%%%%%%%%%%%%%%%%%%%%%%%%%%%%%%%%%%%%%

%%%%%%%%%%%%%%%%%%%%%%%%%%%%%%%%%%%%%%%%
\subsection*{RdA de la asignatura:}
\begin{itemize}[leftmargin=*]
    % \item \textbf{RdA 1:} Plantear los conceptos fundamentales del aprendizaje automático, incluyendo los principios básicos, técnicas de preprocesado de datos, métodos de evaluación y ajuste de modelos, destacando su importancia en el análisis y resolución de problemas de datos.
    \item \textbf{RdA 2:} 
    Aplicar modelos de aprendizaje automático supervisado y no supervisado, así como su validación y optimización, en la resolución de problemas tanto reales como simulados.
    % \item \textbf{RdA 3:} Resolver problemas prácticos mediante el uso de modelos de aprendizaje automático, ajustándolos para la mejora de su rendimiento y precisión.
\end{itemize}

%%%%%%%%%%%%%%%%%%%%%%%%%%%%%%%%%%%%%%%%
\subsection*{RdA de la actividad:}
\begin{itemize}[leftmargin=*]
    \item Comprender los fundamentos teóricos de los Árboles de Decisión y su uso en clasificación y regresión.
    \item Analizar los criterios principales asociados con la construcción y optimización de árboles de decisión (parada, selección, clasificación, partición y poda).
    \item Implementar un Árbol de Decisión en Python mediante un cuaderno de Jupyter.
\end{itemize}

%%%%%%%%%%%%%%%%%%%%%%%%%%%%%%%%%%%%%%%%
\section*{Introducción}
%%%%%%%%%%%%%%%%%%%%%%%%%%%%%%%%%%%%%%%%

%%%%%%%%%%%%%%%%%%%%%%%%%%%%%%%%%%%%%%%%
\paragraph{Pregunta inicial:}
¿Por qué es importante la interpretabilidad en los modelos de aprendizaje automático y cómo los Árboles de Decisión nos ayudan a lograrla?

%%%%%%%%%%%%%%%%%%%%%%%%%%%%%%%%%%%%%%%%
\section*{Desarrollo}
%%%%%%%%%%%%%%%%%%%%%%%%%%%%%%%%%%%%%%%%

%%%%%%%%%%%%%%%%%%%%%%%%%%%%%%%%%%%%%%%%  
\subsection*{Actividad 1: Retroalimentación de la clase invertida}  
%%%%%%%%%%%%%%%%%%%%%%%%%%%%%%%%%%%%%%%%  

Se conecta lo aprendido en casa con la utilidad práctica de los Árboles de Decisión en tareas de clasificación y regresión. Esta actividad busca reforzar los conceptos clave y vincularlos con aplicaciones prácticas para destacar la importancia de los Árboles de Decisión como modelos interpretables y efectivos.

\paragraph{¿Cómo lo haremos?}  
\begin{itemize}  
    \item \textbf{Resumen rápido:}  
    Breve recapitulación de conceptos clave:
    \begin{itemize}[leftmargin=*]  
        \item Interpretabilidad de los Árboles de Decisión mediante reglas basadas en características.  
        \item Entropía e información ganada como métricas para la construcción del árbol.  
    \end{itemize}  

    \item \textbf{Discusión:}  
    Reflexión grupal sobre aplicaciones reales y decisiones de diseño:  
    \begin{itemize}[leftmargin=*]  
        \item ¿Cuándo es preferible un Árbol de Decisión frente a otros modelos?  
        \item ¿Cómo afecta la poda al rendimiento y la interpretabilidad del árbol?  
        \item Identificación de ejemplos prácticos de su uso, como la clasificación de clientes o predicciones en salud.  
    \end{itemize}  
\end{itemize}  

\paragraph{Verificación de aprendizaje:}  
\begin{itemize}[leftmargin=*]  
    \item ¿Qué representa la ganancia de información en un Árbol de Decisión?  
    \item ¿Qué ventajas tiene la interpretabilidad de un Árbol de Decisión?  
\end{itemize}  

%%%%%%%%%%%%%%%%%%%%%%%%%%%%%%%%%%%%%%%%  
\subsection*{Actividad 2: Clase magistral sobre criterios y poda}  
%%%%%%%%%%%%%%%%%%%%%%%%%%%%%%%%%%%%%%%%  

\paragraph{¿Cómo lo haremos?}  
\begin{itemize}[leftmargin=*]  
    \item \textbf{Explicación teórica:}  
        \begin{itemize}[leftmargin=*]  
            \item Criterio de parada: ¿Cuándo detener la división?  
            \item Criterio de selección: Métricas como ganancia de información, índice Gini.  
            \item Criterio de clasificación: Cómo asignar etiquetas finales a las hojas.  
            \item Criterio de partición: Estrategias para dividir datos en subconjuntos.  
            % \item Poda del árbol: Métodos de poda pre-pruning y post-pruning para evitar sobreajuste.  
        \end{itemize}  
    \item \textbf{Discusión guiada:} Comparación de criterios según el tipo de problema (clasificación vs regresión).  
\end{itemize}  

\paragraph{Verificación de aprendizaje:}  
\begin{itemize}[leftmargin=*]  
    \item ¿Por qué es importante el criterio de parada?  
    % \item ¿Cuál es la ventaja de usar post-pruning frente a pre-pruning?  
\end{itemize}  

%%%%%%%%%%%%%%%%%%%%%%%%%%%%%%%%%%%%%%%%
\subsection*{Actividad 3: Implementación práctica de Árboles de Decisión}
%%%%%%%%%%%%%%%%%%%%%%%%%%%%%%%%%%%%%%%%

Esta actividad se enfoca en aplicar Árboles de Decisión utilizando Python en un cuaderno de Jupyter. Los estudiantes experimentarán con criterios de partición, poda y parámetros como la profundidad máxima, conectando teoría con aplicaciones prácticas en tareas de clasificación.

\paragraph{¿Cómo lo haremos?}  
\begin{itemize}[leftmargin=*]  
    \item \textbf{Implementación en Python:} Los estudiantes trabajarán en un cuaderno de Jupyter previamente preparado.  
    \begin{quote}  
        Enlace al cuaderno: \href{https://colab.research.google.com/github/andres-merino/AprendizajeAutomaticoInicial-05-N0105/blob/main/2-Notebooks/16-Arboles-Decision.ipynb}{16-Arboles-Decision.ipynb}.  
    \end{quote}  
    \item \textbf{Experimentación guiada:} Los estudiantes modificarán parámetros del Árbol de Decisión:  
        \begin{itemize}[leftmargin=*]  
            \item Cambiar el criterio de partición (\texttt{'gini'} a \texttt{'entropy'}).  
            \item Ajustar la profundidad máxima (\texttt{max\_depth}).  
            \item Probar con diferentes tamaños de conjuntos de datos.  
        \end{itemize}  
\end{itemize}  

\paragraph{Verificación de aprendizaje:}  
\begin{itemize}[leftmargin=*]  
    \item ¿Cómo afectan los diferentes criterios de partición a la estructura del árbol?  
    \item ¿Qué impacto tiene limitar la profundidad máxima en el modelo?  
    \item ¿Cómo podemos interpretar los resultados visuales del árbol?  
\end{itemize}  
%%%%%%%%%%%%%%%%%%%%%%%%%%%%%%%%%%%%%%%%  
\section*{Cierre}  
%%%%%%%%%%%%%%%%%%%%%%%%%%%%%%%%%%%%%%%%  

\paragraph{Tarea:}  
Implementar un Árbol de Decisión utilizando un conjunto de datos real y comparar los resultados al variar los criterios de partición, poda y profundidad máxima.  

\paragraph{Pregunta de investigación:}  
\begin{enumerate}[leftmargin=*]  
    \item ¿Qué significa podar un árbol de decisión? ¿Cómo se poda un árbol?
\end{enumerate}  

\paragraph{Para la próxima clase:}  
Visualizar el siguiente video sobre Bosques Aleatorios:
    \begin{quote}
        Enlace al video: \href{https://www.youtube.com/watch?v=J4Wdy0Wc_xQ}{StatQuest: Random Forests Part 1}.
    \end{quote}

\end{document} 