\documentclass[a4,11pt]{aleph-notas}
% Se puede ver la documentación aquí: 
% https://github.com/alephsub0/LaTeX_aleph-notas

% -- Paquetes adicionales 
\usepackage{enumitem}
\usepackage{url}
\usepackage{array}
\usepackage{booktabs}
\usepackage{longtable}
\usepackage{environ}  % Para definir nuevos entornos
\hypersetup{
    urlcolor=blue,
    linkcolor=blue,
}


% Definición del nuevo ambiente
\NewEnviron{materiales}{%
    \begin{minipage}{12cm}
        \vspace{0.1mm}
        \begin{itemize}[leftmargin=*]
            \BODY % Contenido de la lista
        \end{itemize}
        \vspace{1mm}
    \end{minipage}
}


% -- Datos 
\institucion{Escuela de Ciencias Físicas y Matemática}
\carrera{Ciencia de Datos}
\asignatura{Aprendizaje Automático Inicial}
\tema[Clase 06: Red. de Dimensionalidad y Extrac. de Características]{Clase 06: Reducción de Dimensionalidad y Extracción de Características}
\autor{Andrés Merino}
\fecha{Semestre 2025-1}

\logouno[0.14\textwidth]{Logos/logoPUCE_04_ac}
\definecolor{colortext}{HTML}{0030A1}
\definecolor{colordef}{HTML}{0030A1}
\fuente{montserrat}


% -- Comandos para tablas
\usepackage{listings}
\input{listings-python.prf}

\usepackage[spanish,onelanguage,vlined,linesnumbered]{algorithm2e}

% -- Comandos adicionales
\newtcolorbox{pscodigo}
    {icono=\faCogs,color=lightgray,postit,top=-1.5mm,bottom=-1.5mm}
    
\definecolor{colcod}{RGB}{174,218,255}
\newtcolorbox{pycodigo}
    {icono=\faKeyboardO, color=colcod, postit, 
    top=-2mm, bottom=-2mm, 
    extras first={bottom=0mm},
    extras last={top=0mm},
    extras middle={top=0mm,bottom=0mm},
    }

\lstloadlanguages{Python}
\lstset{
  language=Python,
  basicstyle=\small\sffamily,
  stringstyle=\color[HTML]{933797},
  commentstyle=\color[HTML]{228B22}\sffamily,
  emph={[2]from,import,pass,return}, emphstyle={[2]\color[HTML]{DD52F0}},
  emph={[3]range}, emphstyle={[3]\color[HTML]{D17032}},
  emph={[4]for,in,def}, emphstyle={[4]\color{blue}},
  showstringspaces=false,
  breaklines=true,
  prebreak=\mbox{{\color{gray}\tiny$\searrow$}},
  xleftmargin=3pt,
  inputencoding=utf8,
  extendedchars=true,
  columns=fullflexible,
  literate={á}{{\'a}}1 {é}{{\'e}}1 {í}{{\'i}}1 {ó}{{\'o}}1 {ú}{{\'u}}1,
}


\SetKwFunction{concat}{Concatenar}
\SetKwProg{Fn}{Función}{\string:}{}
\SetKwFunction{ult}{Ultimo}
\SetKwFunction{pri}{Primero}
\SetKwFunction{sinul}{SinUltimo}
\SetKw{Salir}{Salir}

\newcommand{\fuentecomentario}[1]{\scriptsize\ttfamily #1}
\SetCommentSty{fuentecomentario}
\SetAlFnt{\footnotesize}


\begin{document}

\encabezado


\section*{Resultado de Aprendizaje}

\subsection*{RdA de la asignatura:}
\begin{itemize}[leftmargin=*]
    \item \textbf{RdA 1:} 
    Plantear los conceptos fundamentales del aprendizaje automático, incluyendo los principios básicos, técnicas de preprocesado de datos, métodos de evaluación y ajuste de modelos, destacando su importancia en el análisis y resolución de problemas de datos.
\end{itemize}

\subsection*{RdA de la actividad:}
    \begin{itemize}[leftmargin=*]
        \item Comprender la importancia de la selección y extracción de atributos en la construcción de modelos eficientes.
        \item Implementar métodos básicos como PCA y SVD para reducir dimensionalidad.
        \item Analizar aplicaciones de factorización de matrices no negativas en aprendizaje automático.
    \end{itemize}

\section*{Introducción}

\paragraph{Pregunta inicial:} 
¿Por qué es importante reducir la dimensionalidad de un conjunto de datos al construir un modelo de aprendizaje automático?


\section*{Desarrollo}

\subsection*{Actividad 1: Selección de Atributos}

Se preparará un modelo GPT interactivo con el cual los estudiantes podrán explorar conceptos clave, ventajas y desventajas de técnicas de selección de atributos.

\paragraph{¿Cómo lo haremos?}  
\begin{itemize}[leftmargin=*]
    \item \textbf{Clase interactiva:}  
    Los estudiantes interactuarán con un modelo GPT diseñado para explicar las técnicas de selección de atributos y sus aplicaciones prácticas.
    \begin{quote}
        Enlace al GPT: \href{https://chatgpt.com/g/g-674f787b286c8191aee6a93bde4ede57-tutor-aa-seleccion-de-atributos}{Tutor AA - Selección de Atributos}
    \end{quote}
\end{itemize}

\paragraph{Verificación de aprendizaje:}  
Los estudiantes responderán las preguntas del GPT proporcionado.

\subsection*{Actividad 2: Análisis de Componentes Principales (PCA)}

Se utilizará un modelo GPT para repasar conceptos clave de PCA y guiar a los estudiantes en su implementación práctica.

\paragraph{¿Cómo lo haremos?}  
\begin{itemize}[leftmargin=*]
    \item \textbf{Clase interactiva:}  Los estudiantes interactuarán con un modelo GPT diseñado para sus conocimientos en PCA.
    \begin{quote}
        Enlace al GPT: \href{https://chatgpt.com/g/g-674f7cde85ac81918e2a05a692ae0ee9-evluador-aa-pca}{Evaluador AA - PCA}
    \end{quote}

    \item \textbf{Exploración del cuaderno de Jupyter:}  
       Se proporcionará un cuaderno de Jupyter con ejemplos prácticos de implementación del Análisis de Componentes Principales (PCA) en la reducción de dimensionalidad.
    \begin{quote}
        Enlace al cuaderno: \href{https://colab.research.google.com/github/andres-merino/AprendizajeAutomaticoInicial-05-N0105/blob/main/2-Notebooks/06_1-Reduccion-Dimensionalidad.ipynb}{06\_1-Reduccion-Dimensionalidad}.
    \end{quote}
     
\end{itemize}

\paragraph{Verificación de aprendizaje:}  
Los estudiantes responderán las preguntas del GPT proporcionado.

\subsection*{Actividad 3: Descomposición en Valores Singulares (SVD)}

Clase magistral para explicar los fundamentos de SVD, seguida de un ejemplo práctico de implementación en Python.

\paragraph{¿Cómo lo haremos?}  
\begin{itemize}[leftmargin=*]
    \item \textbf{Clase magistral:}  
    Introducción matemática a SVD y sus aplicaciones en reducción de dimensionalidad y sistemas de recomendación.

    \item \textbf{Ejemplo práctico en Python:}  
    \item \textbf{Exploración del cuaderno de Jupyter:}  
       Se proporcionará un cuaderno de Jupyter con ejemplos prácticos de implementación de la Descomposición en Valores Singulares (SVD) para el análisis de datos.
    \begin{quote}
        Enlace al cuaderno: \href{https://colab.research.google.com/github/andres-merino/AprendizajeAutomaticoInicial-05-N0105/blob/main/2-Notebooks/06_2-SVD.ipynb}{06\_2-SVD}.
    \end{quote}
    \item \textbf{Discusión guiada:}  
    Los estudiantes analizarán el significado de cada matriz resultante (U, S, VT) y cómo estas se relacionan con la matriz original.
\end{itemize}

\paragraph{Verificación de aprendizaje:}  
Los estudiantes responderán cómo se puede usar SVD para compresión de datos y reducción de ruido.

\subsection*{Actividad 4: Factorización de Matrices No Negativas}

Lectura asignada sobre aplicaciones prácticas de factorización de matrices no negativas (NMF).

\paragraph{¿Cómo lo haremos?}  
\begin{itemize}[leftmargin=*]
    \item \textbf{Lectura asignada:}  
    Artículo \href{https://mro.massey.ac.nz/server/api/core/bitstreams/7dbd6b5e-4d71-490a-b1b6-654e40181693/content}{“Non-negative Matrix Factorization: ASurvey”}.  
    \item \textbf{Análisis:}  
    Leer el resumen y la introducción del artículo para discutirlo la siguiente clase.
\end{itemize}

\section*{Cierre}

\paragraph{Tarea:} Culminar la Actividad 4.
    
    
\paragraph{Pregunta de investigación:} 
\begin{enumerate}
    \item ¿Puedo hacer PCA para datos categóricos?
    \item ¿Qué sistemas de recomendación usan SVD?

\end{enumerate}
    


\end{document} 