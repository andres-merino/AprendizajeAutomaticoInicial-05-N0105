\documentclass[a4,11pt]{aleph-notas}

% -- Paquetes adicionales 
\usepackage{enumitem}
\usepackage{url}
\usepackage{array}
\usepackage{booktabs}
\hypersetup{
    urlcolor=blue,
    linkcolor=blue,
}

% -- Datos 
\institucion{Facultad de Ciencias Exactas, Naturales y Ambientales}
\carrera{Ciencia de Datos}
\asignatura{Aprendizaje Automático Inicial}
\tema[Clase 06: Red. de Dimensionalidad y Extrac. de Características]{Clase 06: Reducción de Dimensionalidad y Extracción de Características}
\autor{Andrés Merino}
\fecha{Periodo 2025-2}

\logouno[0.14\textwidth]{Logos/logoPUCE_04_ac}
\definecolor{colortext}{HTML}{0030A1}
\definecolor{colordef}{HTML}{0030A1}
\fuente{montserrat}

\begin{document}

\encabezado

%%%%%%%%%%%%%%%%%%%%%%%%%%%%%%%%%%%%%%%%
\section*{Resultado de Aprendizaje}
%%%%%%%%%%%%%%%%%%%%%%%%%%%%%%%%%%%%%%%%

%%%%%%%%%%%%%%%%%%%%%%%%%%%%%%%%%%%%%%%%
\subsection*{RdA de la asignatura:}
\begin{itemize}[leftmargin=*]
    \item \textbf{RdA 1:} Plantear los conceptos fundamentales del aprendizaje automático, incluyendo los principios básicos, técnicas de preprocesado de datos, métodos de evaluación y ajuste de modelos, destacando su importancia en el análisis y resolución de problemas de datos.
\end{itemize}

%%%%%%%%%%%%%%%%%%%%%%%%%%%%%%%%%%%%%%%%
\subsection*{RdA de la clase:}
% Máximo 3 resultados
\begin{itemize}[leftmargin=*]
    \item Comprender la importancia de la selección y extracción de atributos en la construcción de modelos eficientes.
    \item Implementar métodos básicos como PCA y SVD para reducir dimensionalidad.
    \item Analizar aplicaciones de factorización de matrices no negativas en aprendizaje automático.
\end{itemize}

%%%%%%%%%%%%%%%%%%%%%%%%%%%%%%%%%%%%%%%%
\section*{Introducción}
%%%%%%%%%%%%%%%%%%%%%%%%%%%%%%%%%%%%%%%%

%%%%%%%%%%%%%%%%%%%%%%%%%%%%%%%%%%%%%%%%
\paragraph{Pregunta inicial:} 
¿Por qué es importante reducir la dimensionalidad de un conjunto de datos al construir un modelo de aprendizaje automático?

%%%%%%%%%%%%%%%%%%%%%%%%%%%%%%%%%%%%%%%%
\section*{Desarrollo}
%%%%%%%%%%%%%%%%%%%%%%%%%%%%%%%%%%%%%%%%

%%%%%%%%%%%%%%%%%%%%%%%%%%%%%%%%%%%%%%%%
\subsection*{Actividad 1: Selección de Atributos}

Se preparará un modelo GPT interactivo mediante clase interactiva con el cual los estudiantes podrán explorar conceptos clave, ventajas y desventajas de técnicas de selección de atributos.

\paragraph{¿Cómo lo haremos?}  
\begin{itemize}[leftmargin=*]
    \item \textbf{Clase interactiva:}  
    Los estudiantes interactuarán con un modelo GPT diseñado para explicar las técnicas de selección de atributos y sus aplicaciones prácticas.
    \begin{quote}
        Enlace al GPT: \href{https://chatgpt.com/g/g-674f787b286c8191aee6a93bde4ede57-tutor-aa-seleccion-de-atributos}{Tutor AA - Selección de Atributos}
    \end{quote}
    
\end{itemize}

%%%%%%%%%%%%%%%%%%%%%%%%%%%%%%%%%%%%%%%%
\subsection*{Actividad 2: Análisis de Componentes Principales (PCA)}

Se utilizará un modelo GPT para repasar conceptos clave de PCA mediante clase interactiva y se guiará a los estudiantes en su implementación práctica con cuaderno de Jupyter.

\paragraph{¿Cómo lo haremos?}  
\begin{itemize}[leftmargin=*]
    \item \textbf{Clase interactiva:}  
    Los estudiantes interactuarán con un modelo GPT diseñado para evaluar sus conocimientos en PCA.
    \begin{quote}
        Enlace al GPT: \href{https://chatgpt.com/g/g-674f7cde85ac81918e2a05a692ae0ee9-evluador-aa-pca}{Evaluador AA - PCA}
    \end{quote}
    
    \item \textbf{Implementación en Python:} Los estudiantes accederán a un cuaderno de Jupyter previamente preparado.
    \begin{quote}
        Enlace al cuaderno: \href{https://github.com/andres-merino/AprendizajeAutomaticoInicial-05-N0105/blob/main/2-Notebooks/06-1-Reduccion-Dimensionalidad.ipynb}{06-1-Reduccion-Dimensionalidad.ipynb}.
    \end{quote}
    
    \item \textbf{Experimentación:} Realiza un análisis PCA con un número diferente de componentes principales y observa cómo afecta a la carga de cada componente.
    
\end{itemize}

%%%%%%%%%%%%%%%%%%%%%%%%%%%%%%%%%%%%%%%%
\subsection*{Actividad 3: Descomposición en Valores Singulares (SVD)}

Clase magistral para explicar los fundamentos de SVD mediante introducción matemática, seguida de ejemplo práctico de implementación en cuaderno de Jupyter y discusión guiada sobre el significado de las matrices resultantes.

\paragraph{¿Cómo lo haremos?}  
\begin{itemize}[leftmargin=*]
    \item \textbf{Clase magistral:}  
    Deducción matemática a SVD y sus aplicaciones en reducción de dimensionalidad.

    \item \textbf{Materiales de apoyo:} 
    Se utilizará el documento \href{https://andres-merino.github.io/AprendizajeAutomaticoInicial-05-N0105/2-Resumenes/Resumen06.pdf}{Resumen06.pdf}
    
    \item \textbf{Implementación en Python:} Los estudiantes accederán a un cuaderno de Jupyter previamente preparado.
    \begin{quote}
        Enlace al cuaderno: \href{https://github.com/andres-merino/AprendizajeAutomaticoInicial-05-N0105/blob/main/2-Notebooks/06_2-SVD.ipynbb}{06-2-SVD.ipynb}.
    \end{quote}
    
    \item \textbf{Experimentación:} Amplía los datos del ejemplo agregando más usuarios y películas. Luego, implementa la SVD para realizar otra clasificación de conceptos.
    
\end{itemize}


%%%%%%%%%%%%%%%%%%%%%%%%%%%%%%%%%%%%%%%%
\section*{Cierre}
%%%%%%%%%%%%%%%%%%%%%%%%%%%%%%%%%%%%%%%%

\paragraph{Verificación de aprendizaje:} 
\begin{enumerate}[leftmargin=*]
    \item ¿Qué es PCA y cuál es su objetivo principal en la reducción de dimensionalidad?
    
    \item ¿Cuál es la diferencia entre selección de atributos y extracción de características?
    
    \item ¿Para qué se utiliza la descomposición SVD en aprendizaje automático?

    \item En el SVD, ¿qué representan las matrices $U$, $\Sigma$ y $V^T$?
\end{enumerate}

\paragraph{Preguntas tipo entrevista:} 
\begin{enumerate}[leftmargin=*]
    \item Tienes un dataset con 1000 características y solo 100 muestras. ¿Por qué esto es problemático y cómo usarías PCA para abordarlo? ¿Cuántos componentes principales deberías retener?
    
    \item Si PCA reduce las dimensiones de mi dataset de 100 a 10 características, ¿significa que las 90 características descartadas no eran importantes?
\end{enumerate}

\paragraph{Tarea:} Leer la introducción al artículo \href{https://mro.massey.ac.nz/server/api/core/bitstreams/7dbd6b5e-4d71-490a-b1b6-654e40181693/content}{Non-negative Matrix Factorization: A Survey}.

\paragraph{Pregunta de investigación:} 
\begin{enumerate}[leftmargin=*]
    \item ¿Puedo hacer PCA para datos categóricos?
    \item ¿Qué sistemas de recomendación usan SVD?
\end{enumerate}
    
\paragraph{Para la próxima clase:} Ver el video \href{https://youtu.be/WM7XaTyX7O8?si=gaJaezErTAaJLJp9}{QUÉ es el Aprendizaje No Supervisado}.


\end{document}