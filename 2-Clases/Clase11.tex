\documentclass[a4,11pt]{aleph-notas}

% -- Paquetes adicionales 
\usepackage{enumitem}
\usepackage{url}
\usepackage{array}
\usepackage{booktabs}
\hypersetup{
    urlcolor=blue,
    linkcolor=blue,
}

% -- Datos 
\institucion{Facultad de Ciencias Exactas, Naturales y Ambientales}
\carrera{Ciencia de Datos}
\asignatura{Aprendizaje Automático Inicial}
\tema{Clase 11: Máquinas de soporte vectorial}
\autor{Andrés Merino}
\fecha{Periodo 2025-2}

\logouno[0.14\textwidth]{Logos/logoPUCE_04_ac}
\definecolor{colortext}{HTML}{0030A1}
\definecolor{colordef}{HTML}{0030A1}
\fuente{montserrat}

\begin{document}

\encabezado
% En todo el documento, las indicaciones deben ser simples y directas, con una sola oración.

%%%%%%%%%%%%%%%%%%%%%%%%%%%%%%%%%%%%%%%%
\section*{Resultado de Aprendizaje}
%%%%%%%%%%%%%%%%%%%%%%%%%%%%%%%%%%%%%%%%

%%%%%%%%%%%%%%%%%%%%%%%%%%%%%%%%%%%%%%%%
\subsection*{RdA de la asignatura:}
% Se toma uno de los siguientes
\begin{itemize}[leftmargin=*]
    \item \textbf{RdA 2:} Aplicar modelos de aprendizaje automático supervisado y no supervisado, así como su validación y optimización, en la resolución de problemas tanto reales como simulados.
\end{itemize}

%%%%%%%%%%%%%%%%%%%%%%%%%%%%%%%%%%%%%%%%
\subsection*{RdA de la clase:}
% Máximo 3 resultados
\begin{itemize}[leftmargin=*]
    \item Comprender el funcionamiento del algoritmo de Máquinas de Soporte Vectorial (SVM) y sus fundamentos matemáticos.
    \item Identificar y comparar diferentes funciones de kernel en SVM.
    \item Implementar SVM en Python con Scikit-learn para resolver problemas de clasificación.
\end{itemize}

%%%%%%%%%%%%%%%%%%%%%%%%%%%%%%%%%%%%%%%%
\section*{Introducción}
%%%%%%%%%%%%%%%%%%%%%%%%%%%%%%%%%%%%%%%%

%%%%%%%%%%%%%%%%%%%%%%%%%%%%%%%%%%%%%%%%
\paragraph{Pregunta inicial:}
¿Cómo podríamos diseñar una frontera que no solo divida los datos correctamente, sino que lo haga de manera óptima para nuevos datos desconocidos?

%%%%%%%%%%%%%%%%%%%%%%%%%%%%%%%%%%%%%%%%
\section*{Desarrollo}
%%%%%%%%%%%%%%%%%%%%%%%%%%%%%%%%%%%%%%%%

%%%%%%%%%%%%%%%%%%%%%%%%%%%%%%%%%%%%%%%%
\subsection*{Actividad 1: Introducción a SVM}

En esta actividad los estudiantes conocerán las Máquinas de Soporte Vectorial (SVM) mediante visualización de video y clase magistral, explorando sus fundamentos teóricos sobre margen máximo, hiperplanos, vectores de soporte y optimización cuadrática.

\paragraph{¿Cómo lo haremos?}
\begin{itemize}[leftmargin=*]
    \item \textbf{Revisión de tarea:} Se realizará el siguiente cuestionario a estudiantes de manera aleatoria para revisar la lectura previas:
    \begin{quote}
        Enlace al cuestionario: \href{https://gemini.google.com/share/a88a74d2db79}{Cuestionario: Máquinas de Soporte Vectorial (SVM)}.
    \end{quote}
    \item \textbf{Video introductorio:}
    Ver el video que explica el concepto de margen máximo, hiperplanos y vectores de soporte.
    \begin{quote}
        Enlace al video: \href{https://www.youtube.com/watch?v=jo9slrXKKNo}{¿Qué es una SVM?}.
    \end{quote}
    
    \item \textbf{Clase magistral:}
    Presentación de los elementos teóricos principales:
    \begin{itemize}
        \item Margen máximo y vectores de soporte.
        \item Problema de optimización cuadrática.
    \end{itemize}
    
    \item \textbf{Materiales de apoyo:} 
    Se utilizará el documento \href{https://andres-merino.github.io/AprendizajeAutomaticoInicial-05-N0105/2-Resumenes/Resumen11.pdf}{Resumen11.pdf}
    
\end{itemize}

%%%%%%%%%%%%%%%%%%%%%%%%%%%%%%%%%%%%%%%%
\subsection*{Actividad 2: Implementación práctica de SVM lineal}

En esta actividad los estudiantes implementarán un modelo SVM con kernel lineal mediante exploración de cuaderno Jupyter para resolver un problema de clasificación, ajustando parámetros clave y analizando el impacto en la separación de las clases.

\paragraph{¿Cómo lo haremos?}
\begin{itemize}[leftmargin=*]
    \item \textbf{Explicación de parámetros:}
    Se explicará el parámetro C y su impacto en el margen y la clasificación (valores pequeños de C crean márgenes más amplios tolerando más errores, valores grandes de C crean márgenes más estrictos).
    
    \item \textbf{Implementación en Python:} Los estudiantes accederán a un cuaderno de Jupyter previamente preparado.
    \begin{quote}
        Enlace al cuaderno: \href{https://github.com/andres-merino/AprendizajeAutomaticoInicial-05-N0105/blob/main/2-Notebooks/11-SVM.ipynb}{11-SVM.ipynb}.
    \end{quote}
    
    \item \textbf{Experimentación:} 
    \begin{itemize}
        \item Pruébalo con un kernel polinómico de grado 2 y 3, y observa cómo cambia la visualización de las regiones de decisión.
        \item Genera datos con la función make\_moons y observa cómo cambia la visualización de las regiones de decisión con diferentes kernels.
        \item Para un conjunto de datos no linealmente separable, prueba un kernel polinomial con diferentes grados; realiza un gráfico de la precisión del modelo en función del grado del kernel.
    \end{itemize}
    
\end{itemize}

%%%%%%%%%%%%%%%%%%%%%%%%%%%%%%%%%%%%%%%%
\subsection*{Actividad 3: Introducción a los kernels y su aplicación}

En esta actividad los estudiantes explorarán mediante video, interacción con ChatGPT y clase magistral cómo los kernels permiten a las SVM abordar problemas de clasificación no lineales, analizando diferentes tipos de kernels y su efecto en la separación de datos.

\paragraph{¿Cómo lo haremos?}
\begin{itemize}[leftmargin=*]
    \item \textbf{Video sobre kernels:}
    Visualización del video que explica el kernel trick en SVM.
    \begin{quote}
        Enlace al video: \href{https://youtu.be/Q7vT0--5VII?si=lR6W66Cde65fEz1a}{The Kernel Trick in Support Vector Machine}.
    \end{quote}
    
    \item \textbf{Interacción con ChatGPT:}
    Los estudiantes formularán preguntas al modelo sobre términos nuevos relacionados con kernels y espacios de características.
    
    \item \textbf{Clase magistral:}
    Comparación de kernels comunes:
    \begin{itemize}
        \item Lineal.
        \item Polinomial.
        \item Radial (RBF).
        \item Sigmoide.
    \end{itemize}
       
\end{itemize}

%%%%%%%%%%%%%%%%%%%%%%%%%%%%%%%%%%%%%%%%
\section*{Cierre}
%%%%%%%%%%%%%%%%%%%%%%%%%%%%%%%%%%%%%%%%

\paragraph{Verificación de aprendizaje:}
\begin{enumerate}[leftmargin=*]
    \item ¿Qué es un margen máximo en SVM y por qué es importante maximizarlo?
    
    \item ¿Qué son los vectores de soporte y cuál es su rol en el algoritmo SVM?
    
    \item ¿Cómo funciona el kernel trick y por qué es útil en SVM?
\end{enumerate}

\paragraph{Preguntas tipo entrevista:}
\begin{enumerate}[leftmargin=*]
    \item Tienes un problema de clasificación binaria con 100,000 características y solo 500 muestras. ¿Por qué SVM podría ser una buena elección aquí?
    
    \item Has entrenado un SVM con kernel RBF y obtienes 100\% de accuracy en entrenamiento pero solo 60\% en test. ¿Qué está ocurriendo y cómo lo solucionarías?
\end{enumerate}

\paragraph{Tarea:}
Desarrollar los ejercicios planteados en el siguiente cuaderno:
\begin{quote}
    Enlace al cuaderno: \href{https://colab.research.google.com/github/andres-merino/AprendizajeAutomaticoInicial-05-N0105/blob/main/2-Ejercicios/06-SVM.ipynb}{06-SVM.ipynb}.
\end{quote}

\paragraph{Pregunta de investigación:}
\begin{enumerate}[leftmargin=*]
    \item ¿Cómo se utiliza SVM para regresión?
\end{enumerate}

\paragraph{Para la próxima clase:}
Realizar la Clase invertida: Redes Neuronales, disponible en el aula virtual y aquí: \href{https://andres-merino.github.io/AprendizajeAutomaticoInicial-05-N0105/2-ClaseInvertida/03Est-RedesNeuronales.pdf}{03Est-RedesNeuronales.pdf}.

\end{document}