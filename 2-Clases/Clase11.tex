\documentclass[a4,11pt]{aleph-notas}
% Se puede ver la documentación aquí: 
% https://github.com/alephsub0/LaTeX_aleph-notas

% -- Paquetes adicionales 
\usepackage{enumitem}
\usepackage{url}
\usepackage{array}
\usepackage{booktabs}
\usepackage{longtable}
\usepackage{environ}  % Para definir nuevos entornos
\hypersetup{
    urlcolor=blue,
    linkcolor=blue,
}

% Definición del nuevo ambiente
\NewEnviron{materiales}{%
    \begin{minipage}{12cm}
        \vspace{0.1mm}
        \begin{itemize}[leftmargin=*]
            \BODY % Contenido de la lista
        \end{itemize}
        \vspace{1mm}
    \end{minipage}
}


% -- Datos 
\institucion{Facultad de Ciencias Exactas, Naturales y Ambientales}
\carrera{Ciencia de Datos}
\asignatura{Aprendizaje Automático Inicial}
\tema{Clase 11: Máquinas de soporte vectorial}
\autor{Andrés Merino}
\fecha{Periodo 2025-2}

\logouno[0.14\textwidth]{Logos/logoPUCE_04_ac}
\definecolor{colortext}{HTML}{0030A1}
\definecolor{colordef}{HTML}{0030A1}
\fuente{montserrat}


% -- Comandos para tablas
\usepackage{listings}
\input{listings-python.prf}

\usepackage[spanish,onelanguage,vlined,linesnumbered]{algorithm2e}

% -- Comandos adicionales
\newtcolorbox{pscodigo}
    {icono=\faCogs,color=lightgray,postit,top=-1.5mm,bottom=-1.5mm}
    
\definecolor{colcod}{RGB}{174,218,255}
\newtcolorbox{pycodigo}
    {icono=\faKeyboardO, color=colcod, postit, 
    top=-2mm, bottom=-2mm, 
    extras first={bottom=0mm},
    extras last={top=0mm},
    extras middle={top=0mm,bottom=0mm},
    }

\lstloadlanguages{Python}
\lstset{
  language=Python,
  basicstyle=\small\sffamily,
  stringstyle=\color[HTML]{933797},
  commentstyle=\color[HTML]{228B22}\sffamily,
  emph={[2]from,import,pass,return}, emphstyle={[2]\color[HTML]{DD52F0}},
  emph={[3]range}, emphstyle={[3]\color[HTML]{D17032}},
  emph={[4]for,in,def}, emphstyle={[4]\color{blue}},
  showstringspaces=false,
  breaklines=true,
  prebreak=\mbox{{\color{gray}\tiny$\searrow$}},
  xleftmargin=3pt,
  inputencoding=utf8,
  extendedchars=true,
  columns=fullflexible,
  literate={á}{{\'a}}1 {é}{{\'e}}1 {í}{{\'i}}1 {ó}{{\'o}}1 {ú}{{\'u}}1,
}


\SetKwFunction{concat}{Concatenar}
\SetKwProg{Fn}{Función}{\string:}{}
\SetKwFunction{ult}{Ultimo}
\SetKwFunction{pri}{Primero}
\SetKwFunction{sinul}{SinUltimo}
\SetKw{Salir}{Salir}

\newcommand{\fuentecomentario}[1]{\scriptsize\ttfamily #1}
\SetCommentSty{fuentecomentario}
\SetAlFnt{\footnotesize}


\begin{document}

\encabezado


%%%%%%%%%%%%%%%%%%%%%%%%%%%%%%%%%%%%%%%%
\section*{Resultado de Aprendizaje}
%%%%%%%%%%%%%%%%%%%%%%%%%%%%%%%%%%%%%%%%

%%%%%%%%%%%%%%%%%%%%%%%%%%%%%%%%%%%%%%%%
\subsection*{RdA de la asignatura:}
\begin{itemize}[leftmargin=*]
    % \item \textbf{RdA 1:} Plantear los conceptos fundamentales del aprendizaje automático, incluyendo los principios básicos, técnicas de preprocesado de datos, métodos de evaluación y ajuste de modelos, destacando su importancia en el análisis y resolución de problemas de datos.
    \item \textbf{RdA 2:} 
    Aplicar modelos de aprendizaje automático supervisado y no supervisado, así como su validación y optimización, en la resolución de problemas tanto reales como simulados.
    % \item \textbf{RdA 3:} Resolver problemas prácticos mediante el uso de modelos de aprendizaje automático, ajustándolos para la mejora de su rendimiento y precisión.
\end{itemize}

%%%%%%%%%%%%%%%%%%%%%%%%%%%%%%%%%%%%%%%%
\subsection*{Resultados específicos:}  
\begin{itemize}[leftmargin=*]  
    \item Comprender el funcionamiento del algoritmo de Máquinas de Soporte Vectorial (SVM) y sus fundamentos matemáticos.  
    \item Identificar y comparar diferentes funciones de kernel en SVM.  
    \item Implementar SVM en Python con Scikit-learn para resolver problemas de clasificación.  
\end{itemize}  

%%%%%%%%%%%%%%%%%%%%%%%%%%%%%%%%%%%%%%%%  
\section*{Introducción}  
%%%%%%%%%%%%%%%%%%%%%%%%%%%%%%%%%%%%%%%%  

%%%%%%%%%%%%%%%%%%%%%%%%%%%%%%%%%%%%%%%%  
\paragraph{Pregunta inicial:}  
¿Cómo podríamos diseñar una frontera que no solo divida los datos correctamente, sino que lo haga de manera óptima para nuevos datos desconocidos?

%%%%%%%%%%%%%%%%%%%%%%%%%%%%%%%%%%%%%%%%  
\section*{Desarrollo}  
%%%%%%%%%%%%%%%%%%%%%%%%%%%%%%%%%%%%%%%%  

%%%%%%%%%%%%%%%%%%%%%%%%%%%%%%%%%%%%%%%%  
\subsection*{Actividad 1: Introducción a SVM}  
%%%%%%%%%%%%%%%%%%%%%%%%%%%%%%%%%%%%%%%%  

En esta actividad, los estudiantes conocerán las Máquinas de Soporte Vectorial (SVM), sus fundamentos teóricos, ventajas, desventajas y parámetros clave. Se iniciará con la visualización de un video introductorio, seguido de una explicación teórica y una discusión guiada para reflexionar sobre los conceptos más importantes del algoritmo.

\paragraph{¿Cómo lo haremos?}  
\begin{itemize}[leftmargin=*]  
    \item \textbf{Video introductorio:}  
    Ver el video \href{https://www.youtube.com/watch?v=jo9slrXKKNo}{¿Qué es una SVM?}, que explica el concepto de margen máximo, hiperplanos y vectores de soporte.  
    \item \textbf{Clase magistral:}  
    Presentación de los elementos teóricos principales:  
    \begin{itemize}  
        \item Margen máximo y vectores de soporte.  
        \item Problema de optimización cuadrática.  
    \end{itemize}  
\end{itemize}  

\paragraph{Verificación de aprendizaje:}  
\begin{itemize}[leftmargin=*]  
    \item ¿Qué es un margen máximo y por qué es importante?  
    \item ¿Cómo identifica SVM los vectores de soporte?  
\end{itemize}  

%%%%%%%%%%%%%%%%%%%%%%%%%%%%%%%%%%%%%%%%  
\subsection*{Actividad 2: Implementación práctica de SVM lineal}  
%%%%%%%%%%%%%%%%%%%%%%%%%%%%%%%%%%%%%%%%  
En esta actividad, los estudiantes implementarán un modelo SVM con un kernel lineal para resolver un problema de clasificación. Mediante el uso de un cuaderno Jupyter, explorarán cómo ajustar parámetros clave y analizarán el impacto de estos ajustes en la separación de las clases. Esto les permitirá comprender la relación entre teoría y práctica en el uso de SVM para datos linealmente separables.


\paragraph{¿Cómo lo haremos?}  
\begin{itemize}[leftmargin=*]  
    \item \textbf{Preparación:}  
    Los estudiantes accederán a un cuaderno Jupyter preparado para esta actividad. El cuaderno incluye una implementación básica de un SVM lineal con datos simulados.  
    \begin{quote}
        Enlace al cuaderno: \href{https://colab.research.google.com/github/andres-merino/AprendizajeAutomaticoInicial-05-N0105/blob/main/2-Notebooks/11-SVM.ipynb}{11-SVM.ipynb}.
    \end{quote}
    \item \textbf{Ejercicio guiado:}  
    Los estudiantes ejecutarán y modificarán el siguiente código base:  
\begin{pycodigo}  
\begin{lstlisting}[language=Python]  
from sklearn.datasets import make_blobs
from sklearn.model_selection import train_test_split  
from sklearn.svm import SVC  
from sklearn.metrics import accuracy_score  

# Generar datos de ejemplo  
X, y = make_blobs(n_samples=100, centers=2, cluster_std=4.0, random_state=42)

# Dividir los datos en entrenamiento y prueba
X_train, X_test, y_train, y_test = train_test_split(X, y, test_size=0.3, random_state=42)

# Entrenar un SVM lineal  
modelo = SVC(kernel='linear')  
modelo.fit(X, y)  

# Evaluar el modelo  
y_pred = modelo.predict(X_test)
print("Precisión del modelo:", accuracy_score(y_test, y_pred))
\end{lstlisting}  
\end{pycodigo}  

    \item \textbf{Experimentación:}  
    Los estudiantes ajustarán el valor de \( C \) para observar los cambios en la clasificación y en el margen.  
\end{itemize}  

\begin{ejer}  
Modifique el código para:  
\begin{enumerate}[leftmargin=*]  
    \item Cambiar el valor de \( C \) a \( 0.1 \) y \( 10 \). Compare los resultados.  
\end{enumerate}  
\end{ejer}  

%%%%%%%%%%%%%%%%%%%%%%%%%%%%%%%%%%%%%%%%  
\subsection*{Actividad 3: Introducción a los kernels y su aplicación}  
%%%%%%%%%%%%%%%%%%%%%%%%%%%%%%%%%%%%%%%%  

En esta actividad, los estudiantes explorarán cómo los kernels permiten a las SVM abordar problemas de clasificación no lineales. A través de un video introductorio, una interacción con ChatGPT y ejemplos prácticos, los estudiantes analizarán diferentes tipos de kernels, como el RBF y el polinomial, y experimentarán cómo estas transformaciones afectan la separación de los datos en espacios de mayor dimensionalidad.

\paragraph{¿Cómo lo haremos?}  
\begin{itemize}[leftmargin=*]  
    \item \textbf{Video sobre kernels:}  
    Visualización del video \href{https://youtu.be/Q7vT0--5VII?si=lR6W66Cde65fEz1a}{The Kernel Trick in Support Vector Machine}.  
    \item \textbf{Interacción con ChatGPT:}  
    Los estudiantes formularán preguntas al modelo sobre términos nuevos.
    \item \textbf{Clase magistral:}  
    Comparación de kernels comunes:  
    \begin{itemize}  
        \item Lineal.  
        \item Polinomial.  
        \item Radial (RBF).  
        \item Sigmoide.  
    \end{itemize}  
    \item \textbf{Ejemplo práctico:}  
    Implementación de un SVM con kernel RBF usando datos simulados.  
\end{itemize}  

%%%%%%%%%%%%%%%%%%%%%%%%%%%%%%%%%%%%%%%%  
\section*{Cierre}  
%%%%%%%%%%%%%%%%%%%%%%%%%%%%%%%%%%%%%%%%  

\paragraph{Tarea:}  
Desarrollar los ejercicios planteados en el siguiente cuaderno:  
\begin{quote}  
    Enlace al cuaderno: \href{https://example.com/svm_practice}{SVM Practice Notebook}.  
\end{quote}  

\paragraph{Pregunta de investigación:}  
\begin{enumerate}[leftmargin=*]  
    \item ¿Cómo elegir un kernel adecuado para un problema específico?  
    \item ¿Qué ventajas tiene el kernel RBF frente al kernel lineal?  
    \item ¿Cómo afecta la transformación de características en SVM a la capacidad del modelo para generalizar?  
\end{enumerate}  

\paragraph{Para la próxima clase:}  
Preparar la actividad invertida sobre Redes Neuronales, disponible aquí:  
\begin{quote}  
    \href{https://andres-merino.github.io/AprendizajeAutomaticoInicial-05-N0105/2-ClaseInvertida/03Est-RedesNeuronales.pdf}{03Est-RedesNeuronales.pdf}.  
\end{quote}  



\end{document} 