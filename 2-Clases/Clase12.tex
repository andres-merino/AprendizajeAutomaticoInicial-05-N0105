\documentclass[a4,11pt]{aleph-notas}
% Se puede ver la documentación aquí: 
% https://github.com/alephsub0/LaTeX_aleph-notas

% -- Paquetes adicionales 
\usepackage{enumitem}
\usepackage{url}
\hypersetup{
    urlcolor=blue,
    linkcolor=blue,
}

% -- Datos 
\institucion{Escuela de Ciencias Físicas y Matemática}
\carrera{Ciencia de Datos}
\asignatura{Aprendizaje Automático Inicial}
\tema{Clase 12: Redes neuronales}
\autor{Andrés Merino}
\fecha{Semestre 2024-2}

\logouno[0.14\textwidth]{Logos/logoPUCE_04_ac}
\definecolor{colortext}{HTML}{0030A1}
\definecolor{colordef}{HTML}{0030A1}
\fuente{montserrat}


% -- Comandos para tablas
\usepackage{listings}
\input{listings-python.prf}

\usepackage[spanish,onelanguage,vlined,linesnumbered]{algorithm2e}

% -- Comandos adicionales
\newtcolorbox{pscodigo}
    {icono=\faCogs,color=lightgray,postit,top=-1.5mm,bottom=-1.5mm}
    
\definecolor{colcod}{RGB}{174,218,255}
\newtcolorbox{pycodigo}
    {icono=\faKeyboardO, color=colcod, postit, 
    top=-2mm, bottom=-2mm, 
    extras first={bottom=0mm},
    extras last={top=0mm},
    extras middle={top=0mm,bottom=0mm},
    }

\lstloadlanguages{Python}
\lstset{
  language=Python,
  basicstyle=\small\sffamily,
  stringstyle=\color[HTML]{933797},
  commentstyle=\color[HTML]{228B22}\sffamily,
  emph={[2]from,import,pass,return}, emphstyle={[2]\color[HTML]{DD52F0}},
  emph={[3]range}, emphstyle={[3]\color[HTML]{D17032}},
  emph={[4]for,in,def}, emphstyle={[4]\color{blue}},
  showstringspaces=false,
  breaklines=true,
  prebreak=\mbox{{\color{gray}\tiny$\searrow$}},
  xleftmargin=3pt,
  inputencoding=utf8,
  extendedchars=true,
  columns=fullflexible,
  literate={á}{{\'a}}1 {é}{{\'e}}1 {í}{{\'i}}1 {ó}{{\'o}}1 {ú}{{\'u}}1,
}


\SetKwFunction{concat}{Concatenar}
\SetKwProg{Fn}{Función}{\string:}{}
\SetKwFunction{ult}{Ultimo}
\SetKwFunction{pri}{Primero}
\SetKwFunction{sinul}{SinUltimo}
\SetKw{Salir}{Salir}

\newcommand{\fuentecomentario}[1]{\scriptsize\ttfamily #1}
\SetCommentSty{fuentecomentario}
\SetAlFnt{\footnotesize}


\begin{document}

\encabezado


%%%%%%%%%%%%%%%%%%%%%%%%%%%%%%%%%%%%%%%%
\section*{Resultado de Aprendizaje}
%%%%%%%%%%%%%%%%%%%%%%%%%%%%%%%%%%%%%%%%

%%%%%%%%%%%%%%%%%%%%%%%%%%%%%%%%%%%%%%%%
\subsection*{RdA de la asignatura:}
\begin{itemize}[leftmargin=*]
    % \item \textbf{RdA 1:} Plantear los conceptos fundamentales del aprendizaje automático, incluyendo los principios básicos, técnicas de preprocesado de datos, métodos de evaluación y ajuste de modelos, destacando su importancia en el análisis y resolución de problemas de datos.
    \item \textbf{RdA 2:} Aplicar modelos de aprendizaje automático supervisado y no supervisado, así como su validación y optimización, en la resolución de problemas tanto reales como simulados.
    % \item \textbf{RdA 3:} Resolver problemas prácticos mediante el uso de modelos de aprendizaje automático, ajustándolos para la mejora de su rendimiento y precisión.
\end{itemize}

%%%%%%%%%%%%%%%%%%%%%%%%%%%%%%%%%%%%%%%%
\subsection*{RdA de la actividad:}
\begin{itemize}[leftmargin=*]
    \item Explicar los conceptos básicos de las redes neuronales y su inspiración biológica.
    \item Analizar las funciones de combinación y de activación como componentes esenciales del perceptrón.
    \item Implementar un perceptrón simple en un entorno práctico, interpretando resultados.
\end{itemize}

%%%%%%%%%%%%%%%%%%%%%%%%%%%%%%%%%%%%%%%%
\section*{Introducción}
%%%%%%%%%%%%%%%%%%%%%%%%%%%%%%%%%%%%%%%%

%%%%%%%%%%%%%%%%%%%%%%%%%%%%%%%%%%%%%%%%
\paragraph{Pregunta inicial:} 
¿Crees que las máquinas pueden aprender como un cerebro humano? ¿Qué elementos necesitarían para lograrlo?

%%%%%%%%%%%%%%%%%%%%%%%%%%%%%%%%%%%%%%%%
\section*{Desarrollo}
%%%%%%%%%%%%%%%%%%%%%%%%%%%%%%%%%%%%%%%%

%%%%%%%%%%%%%%%%%%%%%%%%%%%%%%%%%%%%%%%%
\subsection*{Actividad 1: Retroalimentación de la Clase Invertida}
En esta actividad, los estudiantes revisarán los conceptos clave de las redes neuronales, comenzando con su inspiración biológica y el funcionamiento del perceptrón simple. A través de una discusión guiada, se profundizará en el papel de las funciones de activación y el descenso del gradiente como fundamentos esenciales para el aprendizaje automático.

\paragraph{¿Cómo lo haremos?}  
\begin{itemize}[leftmargin=*]
    \item Discusión inicial para resolver dudas y profundizar en los conceptos de la clase invertida:
    \begin{itemize}
        \item Inspiración biológica de las redes neuronales.
        \item Funcionamiento básico del perceptrón simple.
        \item Funciones de activación.
        \item Descenso del gradiente.
    \end{itemize}
    \item Reflexión grupal para conectar la teoría con casos prácticos.
\end{itemize}

\paragraph{Verificación de aprendizaje:}  
\begin{itemize}[leftmargin=*]
    \item ¿Qué conceptos clave destacan en la inspiración biológica de las redes neuronales?
    \item ¿Cómo funciona un perceptrón simple?
    \item ¿Qué papel juegan las funciones de activación en una red neuronal?
\end{itemize}

%%%%%%%%%%%%%%%%%%%%%%%%%%%%%%%%%%%%%%%%
\subsection*{Actividad 2: Clase Magistral - Fundamentos de Redes Neuronales}
%%%%%%%%%%%%%%%%%%%%%%%%%%%%%%%%%%%%%%%%

En esta actividad, los estudiantes explorarán los fundamentos teóricos de las redes neuronales. Se abordará la historia de su desarrollo, las funciones de combinación y activación, así como el funcionamiento del perceptrón. La clase magistral permitirá conectar estos conceptos con aplicaciones prácticas y preparar a los estudiantes para la implementación práctica.

\paragraph{¿Cómo lo haremos?}  
\begin{itemize}[leftmargin=*]
    \item \textbf{Historia:} Breve revisión histórica del desarrollo de las redes neuronales, destacando el perceptrón.
    \item \textbf{Fundamentos:} Presentación teórica sobre:
    \begin{itemize}
        \item Funciones de combinación.
        \item Funciones de activación (ejemplos: sigmoide, ReLU, etc.).
    \end{itemize}
    \item \textbf{Introducción al Perceptrón:} Explicación detallada de su estructura y funcionamiento.
\end{itemize}

\paragraph{Verificación de aprendizaje:}  
\begin{itemize}[leftmargin=*]
    \item ¿Cómo se combinan las entradas en un perceptrón simple?
    \item ¿Qué tipos de funciones de activación existen y cuándo son útiles?
\end{itemize}

%%%%%%%%%%%%%%%%%%%%%%%%%%%%%%%%%%%%%%%%
\subsection*{Actividad 3: Implementación Práctica - Cuaderno de Jupyter}
%%%%%%%%%%%%%%%%%%%%%%%%%%%%%%%%%%%%%%%%

En esta actividad, los estudiantes pondrán en práctica los conceptos aprendidos mediante la implementación de un perceptrón simple en un cuaderno de Jupyter. Experimentarán con diferentes configuraciones, como tasas de aprendizaje y pesos iniciales, para observar su impacto en el entrenamiento y el rendimiento, desarrollando así una comprensión más profunda del modelo.

\paragraph{¿Cómo lo haremos?}  
\begin{itemize}[leftmargin=*]
    \item \textbf{Preparación:} Los estudiantes accederán a un cuaderno de Jupyter previamente preparado.
    \begin{quote}
        Enlace al cuaderno: \href{https://colab.research.google.com/github/andres-merino/AprendizajeAutomaticoInicial-05-N0105/blob/main/2-Notebooks/11-Perceptron-I.ipynb}{11-Perceptron-I.ipynb}.
    \end{quote}
    \item \textbf{Exploración guiada:} Se implementará un perceptrón simple siguiendo instrucciones detalladas en el cuaderno.
    \item \textbf{Experimentación:} Los estudiantes modificarán parámetros como tasa de aprendizaje para observar el impacto en el rendimiento.
\end{itemize}

\paragraph{Verificación de aprendizaje:}  
\begin{itemize}[leftmargin=*]
    \item ¿Cómo afecta la tasa de aprendizaje al entrenamiento del perceptrón?
\end{itemize}

%%%%%%%%%%%%%%%%%%%%%%%%%%%%%%%%%%%%%%%%
\section*{Cierre}
%%%%%%%%%%%%%%%%%%%%%%%%%%%%%%%%%%%%%%%%

\paragraph{Tarea:}
    No queda tarea.

\paragraph{Pregunta de investigación:}  
\begin{enumerate}[leftmargin=*]
    \item ¿Qué limitaciones presenta un perceptrón simple y cómo se pueden superar con redes más complejas?
    \item ¿Qué ventajas tiene el uso de funciones de activación no lineales?
\end{enumerate}

\paragraph{Para la próxima clase:}  
Preparar la actividad invertida sobre Redes Neuronales, disponible aquí:  
\begin{quote}  
    \href{https://andres-merino.github.io/AprendizajeAutomaticoInicial-05-N0105/2-ClaseInvertida/04Est-RedesNeuronales-II.pdf}{04Est-RedesNeuronales-II.pdf}.  
\end{quote}  

\end{document} 