\documentclass[a4,11pt]{aleph-notas}

% -- Paquetes adicionales 
\usepackage{enumitem}
\usepackage{url}
\usepackage{array}
\usepackage{booktabs}
\hypersetup{
    urlcolor=blue,
    linkcolor=blue,
}

% -- Datos 
\institucion{Facultad de Ciencias Exactas, Naturales y Ambientales}
\carrera{Ciencia de Datos}
\asignatura{Aprendizaje Automático Inicial}
\tema{Clase 12: Introducción a las redes neuronales: perceptrón lineal}
\autor{Andrés Merino}
\fecha{Periodo 2025-2}

\logouno[0.14\textwidth]{Logos/logoPUCE_04_ac}
\definecolor{colortext}{HTML}{0030A1}
\definecolor{colordef}{HTML}{0030A1}
\fuente{montserrat}

\begin{document}

\encabezado
% En todo el documento, las indicaciones deben ser simples y directas, con una sola oración.

%%%%%%%%%%%%%%%%%%%%%%%%%%%%%%%%%%%%%%%%
\section*{Resultado de Aprendizaje}
%%%%%%%%%%%%%%%%%%%%%%%%%%%%%%%%%%%%%%%%

%%%%%%%%%%%%%%%%%%%%%%%%%%%%%%%%%%%%%%%%
\subsection*{RdA de la asignatura:}
% Se toma uno de los siguientes
\begin{itemize}[leftmargin=*]
    \item \textbf{RdA 2:} Aplicar modelos de aprendizaje automático supervisado y no supervisado, así como su validación y optimización, en la resolución de problemas tanto reales como simulados.
\end{itemize}

%%%%%%%%%%%%%%%%%%%%%%%%%%%%%%%%%%%%%%%%
\subsection*{RdA de la clase:}
% Máximo 3 resultados
\begin{itemize}[leftmargin=*]
    \item Explicar los conceptos básicos de las redes neuronales y su inspiración biológica.
    \item Analizar las funciones de combinación y de activación como componentes esenciales del perceptrón.
    \item Implementar un perceptrón simple en un entorno práctico, interpretando resultados.
\end{itemize}

%%%%%%%%%%%%%%%%%%%%%%%%%%%%%%%%%%%%%%%%
\section*{Introducción}
%%%%%%%%%%%%%%%%%%%%%%%%%%%%%%%%%%%%%%%%

%%%%%%%%%%%%%%%%%%%%%%%%%%%%%%%%%%%%%%%%
\paragraph{Pregunta inicial:} 
Si una neurona biológica responde con mayor o menor intensidad a los estímulos que recibe, ¿cómo podría una neurona artificial decidir cuándo “activarse” y cuándo no?

%%%%%%%%%%%%%%%%%%%%%%%%%%%%%%%%%%%%%%%%
\section*{Desarrollo}
%%%%%%%%%%%%%%%%%%%%%%%%%%%%%%%%%%%%%%%%

%%%%%%%%%%%%%%%%%%%%%%%%%%%%%%%%%%%%%%%%
\subsection*{Actividad 1: Retroalimentación de la Clase Invertida}

En esta actividad los estudiantes revisarán mediante discusión guiada los conceptos clave de las redes neuronales, comenzando con su inspiración biológica, el funcionamiento del perceptrón simple, las funciones de activación y el descenso del gradiente.

\paragraph{¿Cómo lo haremos?}  
\begin{itemize}[leftmargin=*]
    \item \textbf{Discusión inicial:}
    Se resolverán dudas y se profundizará en los conceptos de la clase invertida:
    \begin{itemize}
        \item Inspiración biológica de las redes neuronales.
        \item Funcionamiento básico del perceptrón simple.
        \item Funciones de activación.
        \item Descenso del gradiente.
    \end{itemize}
   
\end{itemize}

%%%%%%%%%%%%%%%%%%%%%%%%%%%%%%%%%%%%%%%%
\subsection*{Actividad 2: Clase Magistral - Fundamentos de Redes Neuronales}

En esta actividad los estudiantes explorarán mediante clase magistral los fundamentos teóricos de las redes neuronales, abordando su historia, las funciones de combinación y activación, y el funcionamiento del perceptrón.

\paragraph{¿Cómo lo haremos?}  
\begin{itemize}[leftmargin=*]
    \item \textbf{Historia:}
    Breve revisión histórica del desarrollo de las redes neuronales, destacando el perceptrón de Rosenblatt y su evolución.
    
    \item \textbf{Fundamentos:}
    Presentación teórica sobre:
    \begin{itemize}
        \item Funciones de combinación (suma ponderada de entradas).
        \item Funciones de activación (sigmoide, ReLU, tanh, escalón).
    \end{itemize}
    
    \item \textbf{Introducción al Perceptrón:}
    Explicación detallada de su estructura (entradas, pesos, sesgo, función de activación, salida) y funcionamiento.
    
        \item \textbf{Materiales de apoyo:} 
    Se utilizará el documento \href{https://andres-merino.github.io/AprendizajeAutomaticoInicial-05-N0105/2-Resumenes/Resumen12.pdf}{Resumen12.pdf}
    
\end{itemize}

%%%%%%%%%%%%%%%%%%%%%%%%%%%%%%%%%%%%%%%%
\subsection*{Actividad 3: Implementación Práctica - Cuaderno de Jupyter}

En esta actividad los estudiantes pondrán en práctica mediante exploración de cuaderno de Jupyter los conceptos aprendidos implementando un perceptrón simple, experimentando con diferentes configuraciones de tasas de aprendizaje y pesos iniciales.

\paragraph{¿Cómo lo haremos?}  
\begin{itemize}[leftmargin=*]
    \item \textbf{Exploración guiada:}
    Se implementará un perceptrón simple siguiendo instrucciones detalladas en el cuaderno, observando cómo aprende realizar una regresión lineal.
    
    \item \textbf{Implementación en Python:} Los estudiantes accederán a un cuaderno de Jupyter previamente preparado.
    \begin{quote}
        Enlace al cuaderno: \href{https://github.com/andres-merino/AprendizajeAutomaticoInicial-05-N0105/blob/main/2-Notebooks/12-Perceptron-Lineal.ipynb}{12-Perceptron-Lineal.ipynb}.
    \end{quote}

    \item \textbf{Experimentación:} Los estudiantes modificarán parámetros como la tasa de aprendizaje y los pesos iniciales para observar su impacto en el proceso de aprendizaje del perceptrón.
    
\end{itemize}

%%%%%%%%%%%%%%%%%%%%%%%%%%%%%%%%%%%%%%%%
\section*{Cierre}
%%%%%%%%%%%%%%%%%%%%%%%%%%%%%%%%%%%%%%%%

\paragraph{Verificación de aprendizaje:}
\begin{enumerate}[leftmargin=*]
    \item ¿Qué es un perceptrón y cuáles son sus componentes principales?
    
    \item ¿Cómo afecta la tasa de aprendizaje al entrenamiento del perceptrón?
\end{enumerate}

\paragraph{Tarea:}
No queda tarea.

\paragraph{Pregunta de investigación:}  
\begin{enumerate}[leftmargin=*]
    \item ¿Qué limitaciones presenta un perceptrón simple y cómo se pueden superar con redes más complejas?
    \item ¿Una red neuronal puede aproximar cualquier tipo de funciones?
\end{enumerate}

\paragraph{Para la próxima clase:}  
\begin{itemize}
    \item Visualizar el video \href{https://youtu.be/_0wdproot34?si=z6FHp27sh5b4CH32}{Funciones de activación a detalle}.
    \item Visualizar el video \href{https://youtu.be/wzMMyTx7DKk?si=NfsF9j1sd13IsXqO}{Funciones de Activación de Redes Neuronales}.
\end{itemize}

\end{document}