\documentclass[a4,11pt]{aleph-notas}

% -- Paquetes adicionales 
\usepackage{enumitem}
\usepackage{url}
\usepackage{array}
\usepackage{booktabs}
\hypersetup{
    urlcolor=blue,
    linkcolor=blue,
}

% -- Datos 
\institucion{Facultad de Ciencias Exactas, Naturales y Ambientales}
\carrera{Ciencia de Datos}
\asignatura{Aprendizaje Automático Inicial}
\tema{Clase 04: Mi primer modelo}
\autor{Andrés Merino}
\fecha{Periodo 2025-2}

\logouno[0.14\textwidth]{Logos/logoPUCE_04_ac}
\definecolor{colortext}{HTML}{0030A1}
\definecolor{colordef}{HTML}{0030A1}
\fuente{montserrat}

\begin{document}

\encabezado
%%%%%%%%%%%%%%%%%%%%%%%%%%%%%%%%%%%%%%%%
\section*{Actividad previa}
%%%%%%%%%%%%%%%%%%%%%%%%%%%%%%%%%%%%%%%%

\begin{itemize}
    \item Solicitar llenar el foro «¿Qué aprendí esta semana?» del aula virtual.
    \item Cuestionar sobre la colinealidad al aplicar one-hot encoding.
\end{itemize}


%%%%%%%%%%%%%%%%%%%%%%%%%%%%%%%%%%%%%%%%
\section*{Resultado de Aprendizaje}
%%%%%%%%%%%%%%%%%%%%%%%%%%%%%%%%%%%%%%%%

%%%%%%%%%%%%%%%%%%%%%%%%%%%%%%%%%%%%%%%%
\subsection*{RdA de la asignatura:}
%%%%%%%%%%%%%%%%%%%%%%%%%%%%%%%%%%%%%%%%

\begin{itemize}[leftmargin=*]
    \item \textbf{RdA 2:} Aplicar modelos de aprendizaje automático supervisado y no supervisado, así como su validación y optimización, en la resolución de problemas tanto reales como simulados.
    \item \textbf{RdA 3:} Resolver problemas prácticos mediante el uso de modelos de aprendizaje automático, ajustándolos para la mejora de su rendimiento y precisión.
\end{itemize}

%%%%%%%%%%%%%%%%%%%%%%%%%%%%%%%%%%%%%%%%
\subsection*{RdA de la clase:}
%%%%%%%%%%%%%%%%%%%%%%%%%%%%%%%%%%%%%%%%

\begin{itemize}[leftmargin=*]
\item Identificar las etapas esenciales del flujo de trabajo del entrenamiento de un modelo supervisado.
\item Implementar la división de un conjunto de datos en entrenamiento y prueba mediante Python.
\item Entrenar un modelo de regresión lineal y evaluar su desempeño.
\end{itemize}

%%%%%%%%%%%%%%%%%%%%%%%%%%%%%%%%%%%%%%%%
\section*{Introducción}
%%%%%%%%%%%%%%%%%%%%%%%%%%%%%%%%%%%%%%%%

\paragraph{Pregunta inicial:}
¿Cómo se aplica el aprendizaje supervisado para predecir valores continuos en un conjunto de datos usando Python?

%%%%%%%%%%%%%%%%%%%%%%%%%%%%%%%%%%%%%%%%
\section*{Desarrollo}
%%%%%%%%%%%%%%%%%%%%%%%%%%%%%%%%%%%%%%%%

%%%%%%%%%%%%%%%%%%%%%%%%%%%%%%%%%%%%%%%%
\subsection*{Actividad 1: El flujo de trabajo del aprendizaje supervisado}
%%%%%%%%%%%%%%%%%%%%%%%%%%%%%%%%%%%%%%%%

Esta actividad presenta los elementos fundamentales del flujo de trabajo: generación de datos, separación en conjuntos, entrenamiento y evaluación. Se desarrollará mediante clase magistral apoyada con ejemplos y discusión guiada.

\paragraph{¿Cómo lo haremos?}
\begin{itemize}[leftmargin=*]
\item \textbf{Presentación del flujo:} Se revisarán las etapas: definición del problema, construcción del conjunto de datos, división en entrenamiento y prueba, entrenamiento del modelo, predicción y métricas.
\item \textbf{Discusión guiada:} Se plantearán preguntas para reflexionar sobre la importancia de cada etapa.

\end{itemize}

%%%%%%%%%%%%%%%%%%%%%%%%%%%%%%%%%%%%%%%%
\subsection*{Actividad 2: Implementación en Python de un modelo de regresión lineal}
%%%%%%%%%%%%%%%%%%%%%%%%%%%%%%%%%%%%%%%%

La actividad se desarrollará mediante trabajo práctico guiado en Python, utilizando un conjunto de datos simulado.

\paragraph{¿Cómo lo haremos?}
\begin{itemize}[leftmargin=*]
\item \textbf{Implementación en Python:} Los estudiantes accederán a un cuaderno de Jupyter previamente preparado.
\begin{quote}
Enlace al cuaderno: \href{https://github.com/andres-merino/AprendizajeAutomaticoInicial-05-N0105/blob/main/2-Notebooks/04-Mi-primer-modelo.ipynb}{04-Mi-primer-modelo.ipynb}.
\end{quote}
\item \textbf{Entrenamiento y evaluación:} Se ejecutará el modelo de regresión lineal, se obtendrán predicciones y se calcularán métricas como el error cuadrático medio.
\item \textbf{Experimentación:} 
\begin{itemize}
    \item Cambiar el tamaño del conjunto de entrenamiento y observar el impacto en las métricas.
\end{itemize}
\end{itemize}

%%%%%%%%%%%%%%%%%%%%%%%%%%%%%%%%%%%%%%%%
\subsection*{Actividad 3: Taller de implementación}
%%%%%%%%%%%%%%%%%%%%%%%%%%%%%%%%%%%%%%%%

Los estudiantes trabajarán para implementar un modelo de regresión lineal desde cero, siguiendo los pasos aprendidos, sobre el conjunto de datos \texttt{kingCountyHouseData}.

\paragraph{¿Cómo lo haremos?}
\begin{itemize}[leftmargin=*]
    \item \textbf{Desarrollo del taller:} Los estudiantes aplicarán los conceptos aprendidos para dividir los datos, entrenar el modelo y evaluar su desempeño. Se utilizará la siguiente plantilla:
    \begin{quote}
    Enlace al cuaderno: \href{https://github.com/andres-merino/AprendizajeAutomaticoInicial-05-N0105/blob/main/2-Ejercicios/01-Mi-segundo-modelo-Plantilla.ipynb}{01-Mi-segundo-modelo-Plantilla.ipynb}.
    \end{quote}
    \item \textbf{Soporte y guía:} El instructor estará disponible para resolver dudas y proporcionar retroalimentación durante la actividad.
\end{itemize}

%%%%%%%%%%%%%%%%%%%%%%%%%%%%%%%%%%%%%%%%
\section*{Cierre}
%%%%%%%%%%%%%%%%%%%%%%%%%%%%%%%%%%%%%%%%

\paragraph{Verificación de aprendizaje:}
\begin{itemize}[leftmargin=*]
\item ¿Qué etapas conforman el flujo de trabajo del aprendizaje supervisado?
\item ¿Cómo se divide un conjunto de datos en entrenamiento y prueba utilizando Python?
\item ¿Qué hacen los métodos \texttt{fit()}, \texttt{transform()} y \texttt{predict()} en scikit-learn?
\end{itemize}

\paragraph{Tarea:}
    Completar la tareas «Repositorio de GitHub para tareas» y «Mi segundo modelo», del aula virtual.


\end{document} 