\documentclass[a4,11pt]{aleph-notas}
% Se puede ver la documentación aquí: 
% https://github.com/alephsub0/LaTeX_aleph-notas

% -- Paquetes adicionales 
\usepackage{enumitem}
\usepackage{url}
\usepackage{array}
\usepackage{booktabs}
\usepackage{longtable}
\usepackage{environ}  % Para definir nuevos entornos
\hypersetup{
    urlcolor=blue,
    linkcolor=blue,
}

% Definición del nuevo ambiente
\NewEnviron{materiales}{%
    \begin{minipage}{12cm}
        \vspace{0.1mm}
        \begin{itemize}[leftmargin=*]
            \BODY % Contenido de la lista
        \end{itemize}
        \vspace{1mm}
    \end{minipage}
}

% -- Datos 
\institucion{Escuela de Ciencias Físicas y Matemática}
\carrera{Ciencia de Datos}
\asignatura{Aprendizaje Automático Inicial}
\tema{Clase 10: Algoritmo k-Nearest Neighbors}
\autor{Andrés Merino}
\fecha{Semestre 2025-1}

\logouno[0.14\textwidth]{Logos/logoPUCE_04_ac}
\definecolor{colortext}{HTML}{0030A1}
\definecolor{colordef}{HTML}{0030A1}
\fuente{montserrat}


% -- Comandos para tablas
\usepackage{listings}
\input{listings-python.prf}

\usepackage[spanish,onelanguage,vlined,linesnumbered]{algorithm2e}

% -- Comandos adicionales
\newtcolorbox{pscodigo}
    {icono=\faCogs,color=lightgray,postit,top=-1.5mm,bottom=-1.5mm}
    
\definecolor{colcod}{RGB}{174,218,255}
\newtcolorbox{pycodigo}
    {icono=\faKeyboardO, color=colcod, postit, 
    top=-2mm, bottom=-2mm, 
    extras first={bottom=0mm},
    extras last={top=0mm},
    extras middle={top=0mm,bottom=0mm},
    }

\lstloadlanguages{Python}
\lstset{
  language=Python,
  basicstyle=\small\sffamily,
  stringstyle=\color[HTML]{933797},
  commentstyle=\color[HTML]{228B22}\sffamily,
  emph={[2]from,import,pass,return}, emphstyle={[2]\color[HTML]{DD52F0}},
  emph={[3]range}, emphstyle={[3]\color[HTML]{D17032}},
  emph={[4]for,in,def}, emphstyle={[4]\color{blue}},
  showstringspaces=false,
  breaklines=true,
  prebreak=\mbox{{\color{gray}\tiny$\searrow$}},
  xleftmargin=3pt,
  inputencoding=utf8,
  extendedchars=true,
  columns=fullflexible,
  literate={á}{{\'a}}1 {é}{{\'e}}1 {í}{{\'i}}1 {ó}{{\'o}}1 {ú}{{\'u}}1,
}


\SetKwFunction{concat}{Concatenar}
\SetKwProg{Fn}{Función}{\string:}{}
\SetKwFunction{ult}{Ultimo}
\SetKwFunction{pri}{Primero}
\SetKwFunction{sinul}{SinUltimo}
\SetKw{Salir}{Salir}

\newcommand{\fuentecomentario}[1]{\scriptsize\ttfamily #1}
\SetCommentSty{fuentecomentario}
\SetAlFnt{\footnotesize}


\begin{document}

\encabezado


%%%%%%%%%%%%%%%%%%%%%%%%%%%%%%%%%%%%%%%%
\section*{Resultado de Aprendizaje}
%%%%%%%%%%%%%%%%%%%%%%%%%%%%%%%%%%%%%%%%

%%%%%%%%%%%%%%%%%%%%%%%%%%%%%%%%%%%%%%%%
\subsection*{RdA de la asignatura:}
\begin{itemize}[leftmargin=*]
    % \item \textbf{RdA 1:} Plantear los conceptos fundamentales del aprendizaje automático, incluyendo los principios básicos, técnicas de preprocesado de datos, métodos de evaluación y ajuste de modelos, destacando su importancia en el análisis y resolución de problemas de datos.
    \item \textbf{RdA 2:} 
    Aplicar modelos de aprendizaje automático supervisado y no supervisado, así como su validación y optimización, en la resolución de problemas tanto reales como simulados.
    % \item \textbf{RdA 3:} Resolver problemas prácticos mediante el uso de modelos de aprendizaje automático, ajustándolos para la mejora de su rendimiento y precisión.
\end{itemize}

%%%%%%%%%%%%%%%%%%%%%%%%%%%%%%%%%%%%%%%%
\subsection*{RdA de la actividad:}
%%%%%%%%%%%%%%%%%%%%%%%%%%%%%%%%%%%%%%%%
\begin{itemize}[leftmargin=*]
    \item Comprender los fundamentos del algoritmo kNN, sus ventajas, desventajas y parámetros principales.
    \item Implementar el algoritmo kNN en Python y evaluar su desempeño en un problema práctico.
\end{itemize}

%%%%%%%%%%%%%%%%%%%%%%%%%%%%%%%%%%%%%%%%
\section*{Introducción}
%%%%%%%%%%%%%%%%%%%%%%%%%%%%%%%%%%%%%%%%

%%%%%%%%%%%%%%%%%%%%%%%%%%%%%%%%%%%%%%%%
\paragraph{Pregunta inicial:} 
Imagina que estás organizando una fiesta y no sabes si alguien prefiere pizza o sushi. Si conoces las preferencias de sus amigos más cercanos, ¿podrías predecir su elección? ¿Cómo lo harías?

%%%%%%%%%%%%%%%%%%%%%%%%%%%%%%%%%%%%%%%%
\section*{Desarrollo}
%%%%%%%%%%%%%%%%%%%%%%%%%%%%%%%%%%%%%%%%

%%%%%%%%%%%%%%%%%%%%%%%%%%%%%%%%%%%%%%%%
\subsection*{Actividad 1: Introducción al algoritmo kNN}
%%%%%%%%%%%%%%%%%%%%%%%%%%%%%%%%%%%%%%%%

En esta actividad, los estudiantes conocerán el algoritmo kNN, sus fundamentos, ventajas, desventajas y parámetros clave. Se iniciará con la visualización de un video y se continuará con una discusión guiada sobre los conceptos más importantes.

\paragraph{¿Cómo lo haremos?}  
\begin{itemize}[leftmargin=*]
    \item \textbf{Proyección de video:}  
    Se presentará un video educativo que explique qué es el aprendizaje no supervisado.
    \begin{quote}
        Enlace al video: \href{https://www.youtube.com/watch?v=0p0o5cmgLdE}{K Nearest Neighbors}.
    \end{quote}
    \item \textbf{Interacción con ChatGPT:}  
    Después del video, cada estudiante formulará preguntas a ChatGPT relacionadas con los términos o conceptos nuevos.
    \item \textbf{Clase magistral:} Se explicará cómo funciona kNN en detalle: 
        \begin{itemize}
            \item Concepto de "vecinos más cercanos".
            \item Parámetros importantes (valor de k, peso por distancia).
            \item Ventajas y desventajas del algoritmo.
        \end{itemize}
\end{itemize}

\paragraph{Verificación de aprendizaje:}  
Se plantearán preguntas rápidas para los estudiantes:
\begin{itemize}[leftmargin=*]
    \item ¿Cómo decide kNN la clase de un punto nuevo cuando $k=3$?
    \item Si $k=1$, ¿cómo afecta esto al modelo en términos de sensibilidad al ruido?
\end{itemize}

%%%%%%%%%%%%%%%%%%%%%%%%%%%%%%%%%%%%%%%%
\subsection*{Actividad 2: Implementación práctica del algoritmo kNN}
%%%%%%%%%%%%%%%%%%%%%%%%%%%%%%%%%%%%%%%%

En esta actividad, los estudiantes implementarán el algoritmo kNN utilizando Python. Se trabajará con un cuaderno de Jupyter que incluye un ejemplo guiado y un ejercicio práctico.

\paragraph{¿Cómo lo haremos?}  
\begin{itemize}[leftmargin=*]
    \item \textbf{Preparación:}  
    Los estudiantes accederán a un cuaderno de Google Colab preparado para esta actividad. El cuaderno contiene un ejemplo básico de clasificación utilizando kNN.
    \begin{quote}
        Enlace al cuaderno: \href{https://colab.research.google.com/github/andres-merino/AprendizajeAutomaticoInicial-05-N0105/blob/main/2-Notebooks/10-kNN.ipynb}{10-kNN.ipynb}.
    \end{quote}

    \item \textbf{Ejercicio guiado:}  
    Los estudiantes ejecutarán y modificarán el siguiente código base:
\begin{pycodigo}
\begin{lstlisting}[language=Python]
# Importar librerías necesarias
from sklearn.datasets import load_iris
from sklearn.model_selection import train_test_split
from sklearn.neighbors import KNeighborsClassifier
from sklearn.metrics import accuracy_score

# Cargar el dataset Iris
iris = load_iris()
X, y = iris.data, iris.target

# Dividir los datos en entrenamiento y prueba
X_train, X_test, y_train, y_test = train_test_split(X, y, test_size=0.3, random_state=42)

# Crear y entrenar el modelo kNN con k=3
modelo = KNeighborsClassifier(n_neighbors=3)
modelo.fit(X_train, y_train)

# Realizar predicciones y evaluar el modelo
y_pred = modelo.predict(X_test)
print("Precisión del modelo:", accuracy_score(y_test, y_pred))
\end{lstlisting}
\end{pycodigo}

    \item \textbf{Experimentación:}  
    Los estudiantes modificarán el código para experimentar con:
    \begin{itemize}[leftmargin=*]
        \item Cambiar el valor de \(k\) y observar cómo afecta la precisión.
        \item Probar diferentes métricas de distancia, como \texttt{'manhattan'} o \texttt{'euclidean'}.
        \item Usar una proporción diferente en el conjunto de prueba (\texttt{test\_size}).
    \end{itemize}
\end{itemize}

\begin{ejer}
Modifique el código para:
\begin{enumerate}[leftmargin=*]
    \item Establecer \(k = 5\) y comparar los resultados con \(k = 3\).
    \item Utilizar la métrica de distancia \texttt{'manhattan'} en lugar de \texttt{'euclidean'}.
    \item Cambiar el tamaño del conjunto de prueba a \(40\%\) y analizar el impacto en la precisión.
\end{enumerate}
\end{ejer}

\paragraph{Verificación de aprendizaje:}  
\begin{itemize}[leftmargin=*]
    \item ¿Cómo afecta el valor de \(k\) la precisión del modelo?
    \item ¿Qué diferencias observaste al usar la métrica \texttt{'manhattan'} en comparación con \texttt{'euclidean'}?
    \item ¿Cómo influye el tamaño del conjunto de prueba en los resultados del modelo?
\end{itemize}

%%%%%%%%%%%%%%%%%%%%%%%%%%%%%%%%%%%%%%%%
\section*{Cierre}
%%%%%%%%%%%%%%%%%%%%%%%%%%%%%%%%%%%%%%%%

\paragraph{Tarea:}
    Desarrollar los ejercicios planteados en el siguiente cuaderno y entregarlo por el aula virtual:
    \begin{quote}
        Enlace al cuaderno: \href{https://colab.research.google.com/github/andres-merino/AprendizajeAutomaticoInicial-05-N0105/blob/main/2-Ejercicios/05-kNN.ipynb}{05-kNN.ipynb}.
    \end{quote}

\paragraph{Pregunta de investigación:}  
\begin{enumerate}[leftmargin=*]
    \item ¿Qué sucede con el rendimiento de kNN cuando el tamaño del dataset crece significativamente, y cómo podrías abordar este problema?
    \item ¿Qué tan sensible es kNN a datos con ruido o valores atípicos, y cómo podrías preprocesar los datos para mitigar estos efectos?
\end{enumerate}

\paragraph{Para la próxima clase:}  
Visualizar el siguiente video sobre SVM:
    \begin{quote}
        Enlace al video: \href{https://www.youtube.com/watch?v=jo9slrXKKNo}{¿Qué es una SVM?}.
    \end{quote}



\end{document} 