\documentclass[a4,11pt]{aleph-notas}

% -- Paquetes adicionales 
\usepackage{enumitem}
\usepackage{url}
\usepackage{array}
\usepackage{booktabs}
\hypersetup{
    urlcolor=blue,
    linkcolor=blue,
}


% -- Datos 
\institucion{Facultad de Ciencias Exactas, Naturales y Ambientales}
\carrera{Ciencia de Datos}
\asignatura{Aprendizaje Automático Inicial}
\tema{Clase 01: Introducción al Aprendizaje Automático}
\autor{Andrés Merino}
\fecha{Periodo 2025-2}

\logouno[0.14\textwidth]{Logos/logoPUCE_04_ac}
\definecolor{colortext}{HTML}{0030A1}
\definecolor{colordef}{HTML}{0030A1}
\fuente{montserrat}

% -- Comandos adicionales (comentados si no se usan)
% \usepackage{listings}
\input{listings-python.prf}

\usepackage[spanish,onelanguage,vlined,linesnumbered]{algorithm2e}

% -- Comandos adicionales
\newtcolorbox{pscodigo}
    {icono=\faCogs,color=lightgray,postit,top=-1.5mm,bottom=-1.5mm}
    
\definecolor{colcod}{RGB}{174,218,255}
\newtcolorbox{pycodigo}
    {icono=\faKeyboardO, color=colcod, postit, 
    top=-2mm, bottom=-2mm, 
    extras first={bottom=0mm},
    extras last={top=0mm},
    extras middle={top=0mm,bottom=0mm},
    }

\lstloadlanguages{Python}
\lstset{
  language=Python,
  basicstyle=\small\sffamily,
  stringstyle=\color[HTML]{933797},
  commentstyle=\color[HTML]{228B22}\sffamily,
  emph={[2]from,import,pass,return}, emphstyle={[2]\color[HTML]{DD52F0}},
  emph={[3]range}, emphstyle={[3]\color[HTML]{D17032}},
  emph={[4]for,in,def}, emphstyle={[4]\color{blue}},
  showstringspaces=false,
  breaklines=true,
  prebreak=\mbox{{\color{gray}\tiny$\searrow$}},
  xleftmargin=3pt,
  inputencoding=utf8,
  extendedchars=true,
  columns=fullflexible,
  literate={á}{{\'a}}1 {é}{{\'e}}1 {í}{{\'i}}1 {ó}{{\'o}}1 {ú}{{\'u}}1,
}


\SetKwFunction{concat}{Concatenar}
\SetKwProg{Fn}{Función}{\string:}{}
\SetKwFunction{ult}{Ultimo}
\SetKwFunction{pri}{Primero}
\SetKwFunction{sinul}{SinUltimo}
\SetKw{Salir}{Salir}

\newcommand{\fuentecomentario}[1]{\scriptsize\ttfamily #1}
\SetCommentSty{fuentecomentario}
\SetAlFnt{\footnotesize}


\begin{document}

\encabezado
% En todo el documento, las indicaciones deben ser simples y directas, con una sola oración.

%%%%%%%%%%%%%%%%%%%%%%%%%%%%%%%%%%%%%%%%
\section*{Resultado de Aprendizaje}
%%%%%%%%%%%%%%%%%%%%%%%%%%%%%%%%%%%%%%%%

%%%%%%%%%%%%%%%%%%%%%%%%%%%%%%%%%%%%%%%%
\subsection*{RdA de la asignatura:}
% Se toma uno de los siguientes
\begin{itemize}[leftmargin=*]
    \item \textbf{RdA 1:} Plantear los conceptos fundamentales del aprendizaje automático, incluyendo los principios básicos, técnicas de preprocesado de datos, métodos de evaluación y ajuste de modelos, destacando su importancia en el análisis y resolución de problemas de datos.
\end{itemize}

%%%%%%%%%%%%%%%%%%%%%%%%%%%%%%%%%%%%%%%%
\subsection*{RdA de la clase:}
% Máximo 3 resultados
\begin{itemize}[leftmargin=*]
    \item Identificar la estructura general del curso y sus componentes.
    \item Reconocer los contenidos, cronograma y mecanismos de evaluación.
    \item Comprender el uso de Git, GitHub y lineamientos de IA para el desarrollo del curso.
\end{itemize}

%%%%%%%%%%%%%%%%%%%%%%%%%%%%%%%%%%%%%%%%
\section*{Introducción}
%%%%%%%%%%%%%%%%%%%%%%%%%%%%%%%%%%%%%%%%

%%%%%%%%%%%%%%%%%%%%%%%%%%%%%%%%%%%%%%%%
\paragraph{Pregunta inicial:} 
Si mañana tuvieras que construir desde cero un sistema inteligente, ¿qué conocimientos previos realmente necesitarías?


%%%%%%%%%%%%%%%%%%%%%%%%%%%%%%%%%%%%%%%%
\section*{Desarrollo}
%%%%%%%%%%%%%%%%%%%%%%%%%%%%%%%%%%%%%%%%

%%%%%%%%%%%%%%%%%%%%%%%%%%%%%%%%%%%%%%%%
\subsection*{Actividad 1: Presentación del curso y estructura académica}

Esta actividad presenta el cronograma, los contenidos, la forma de evaluación, los requisitos previos y las interrelaciones con otras asignaturas de la malla.

\paragraph{¿Cómo lo haremos?}  
\begin{itemize}[leftmargin=*]
    \item \textbf{Revisión del programa:} Se explicará el cronograma, los contenidos, el peso de cada evaluación y la secuencia lógica del curso.
    \item \textbf{Conceptos previos:} Se revisarán los conocimientos mínimos esperados y su conexión con asignaturas de la carrera.
    \item \textbf{Reglas del curso:} Se explicarán normas de trabajo, pautas de uso de IA y lineamientos éticos.
\end{itemize}

%%%%%%%%%%%%%%%%%%%%%%%%%%%%%%%%%%%%%%%%
\subsection*{Actividad 2: Inteligencia Artificial, Git y GitHub}

Esta actividad presenta el rol actual de la inteligencia artificial en el curso y enseña el uso básico de Git y GitHub para organizar el trabajo académico.

\paragraph{¿Cómo lo haremos?}  
\begin{itemize}[leftmargin=*]
    \item \textbf{Discusión guiada:} Se explicará el papel de la IA en el aprendizaje y el uso responsable de herramientas como ChatGPT.
    \item \textbf{Demostración técnica:} Se mostrará cómo crear un repositorio, clonar, hacer commits y sincronizar cambios.
    \item \textbf{Práctica guiada:} Los estudiantes iniciarán su repositorio personal para guardar códigos del curso. Se basarán en esta plantilla: \href{https://github.com/andres-merino/FormatoBaseProyectos}{FormatoBaseProyectos}.
\end{itemize}

%%%%%%%%%%%%%%%%%%%%%%%%%%%%%%%%%%%%%%%%
\section*{Cierre}
%%%%%%%%%%%%%%%%%%%%%%%%%%%%%%%%%%%%%%%%

\paragraph{Verificación de aprendizaje:}  
\begin{itemize}[leftmargin=*]
    \item ¿Cuál es la estructura de evaluación del curso?
    \item ¿Qué conocimientos previos se requieren para aprobar el curso?
    \item ¿Qué pasos básicos se siguen para crear y actualizar un repositorio en GitHub?
\end{itemize}

\paragraph{Preguntas tipo entrevista:} 
\begin{enumerate}[leftmargin=*]
    \item ¿Por qué es importante mantener un repositorio ordenado para un proyecto técnico? 
    \item Si GitHub guarda automáticamente tus avances sin intervención manual, ¿por qué sería necesario hacer commits frecuentemente?
\end{enumerate}

\paragraph{Tarea:}  
Crear un repositorio en GitHub para el curso y subir un archivo \texttt{README.md} con una breve descripción del estudiante y su motivación para aprender aprendizaje automático.

\paragraph{Para la próxima clase:}  
Iniciar el Curso online gratuito \href{https://www.elementsofai.com/es}{«Elementos de IA»}, además, realizar el control de lectura «¿Cómo deberíamos definir IA?» del aula virtual. Por otro lado, 

\end{document}