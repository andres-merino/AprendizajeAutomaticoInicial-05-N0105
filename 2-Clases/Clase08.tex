\documentclass[a4,11pt]{aleph-notas}
% Se puede ver la documentación aquí: 
% https://github.com/alephsub0/LaTeX_aleph-notas

% -- Paquetes adicionales 
\usepackage{enumitem}
\usepackage{url}
\usepackage{array}
\usepackage{booktabs}
\usepackage{longtable}
\usepackage{environ}  % Para definir nuevos entornos
\hypersetup{
    urlcolor=blue,
    linkcolor=blue,
}

% Definición del nuevo ambiente
\NewEnviron{materiales}{%
    \begin{minipage}{12cm}
        \vspace{0.1mm}
        \begin{itemize}[leftmargin=*]
            \BODY % Contenido de la lista
        \end{itemize}
        \vspace{1mm}
    \end{minipage}
}


% -- Datos 
\institucion{Facultad de Ciencias Exactas, Naturales y Ambientales}
\carrera{Ciencia de Datos}
\asignatura{Aprendizaje Automático Inicial}
\tema{Clase 08: Algoritmos de Agrupamiento Jerárquico}
\autor{Andrés Merino}
\fecha{Periodo 2025-2}

\logouno[0.14\textwidth]{Logos/logoPUCE_04_ac}
\definecolor{colortext}{HTML}{0030A1}
\definecolor{colordef}{HTML}{0030A1}
\fuente{montserrat}


% -- Comandos para tablas
\usepackage{listings}
\input{listings-python.prf}

\usepackage[spanish,onelanguage,vlined,linesnumbered]{algorithm2e}

% -- Comandos adicionales
\newtcolorbox{pscodigo}
    {icono=\faCogs,color=lightgray,postit,top=-1.5mm,bottom=-1.5mm}
    
\definecolor{colcod}{RGB}{174,218,255}
\newtcolorbox{pycodigo}
    {icono=\faKeyboardO, color=colcod, postit, 
    top=-2mm, bottom=-2mm, 
    extras first={bottom=0mm},
    extras last={top=0mm},
    extras middle={top=0mm,bottom=0mm},
    }

\lstloadlanguages{Python}
\lstset{
  language=Python,
  basicstyle=\small\sffamily,
  stringstyle=\color[HTML]{933797},
  commentstyle=\color[HTML]{228B22}\sffamily,
  emph={[2]from,import,pass,return}, emphstyle={[2]\color[HTML]{DD52F0}},
  emph={[3]range}, emphstyle={[3]\color[HTML]{D17032}},
  emph={[4]for,in,def}, emphstyle={[4]\color{blue}},
  showstringspaces=false,
  breaklines=true,
  prebreak=\mbox{{\color{gray}\tiny$\searrow$}},
  xleftmargin=3pt,
  inputencoding=utf8,
  extendedchars=true,
  columns=fullflexible,
  literate={á}{{\'a}}1 {é}{{\'e}}1 {í}{{\'i}}1 {ó}{{\'o}}1 {ú}{{\'u}}1,
}


\SetKwFunction{concat}{Concatenar}
\SetKwProg{Fn}{Función}{\string:}{}
\SetKwFunction{ult}{Ultimo}
\SetKwFunction{pri}{Primero}
\SetKwFunction{sinul}{SinUltimo}
\SetKw{Salir}{Salir}

\newcommand{\fuentecomentario}[1]{\scriptsize\ttfamily #1}
\SetCommentSty{fuentecomentario}
\SetAlFnt{\footnotesize}


\begin{document}

\encabezado

%%%%%%%%%%%%%%%%%%%%%%%%%%%%%%%%%%%%%%%%
\section*{Resultado de Aprendizaje}

\subsection*{RdA de la asignatura:}
\begin{itemize}[leftmargin=*]
    \item \textbf{RdA 2:} 
    Aplicar modelos de aprendizaje automático supervisado y no supervisado, así como su validación y optimización, en la resolución de problemas tanto reales como simulados.
\end{itemize}

%%%%%%%%%%%%%%%%%%%%%%%%%%%%%%%%%%%%%%%%
\subsection*{RdA de la clase:}

\begin{itemize}[leftmargin=*]
    \item Describir y diferenciar los algoritmos jerárquicos aglomerativos y divisivos, junto con sus ventajas y desventajas.
    \item Analizar y construir dendogramas para entender los resultados de un agrupamiento jerárquico.
    \item Aplicar diferentes criterios de enlace en algoritmos de agrupamiento jerárquico.
    \item Implementar algoritmos de agrupamiento jerárquico en un conjunto de datos utilizando Python.

\end{itemize}


%%%%%%%%%%%%%%%%%%%%%%%%%%%%%%%%%%%%%%%%
\section*{Introducción}

%%%%%%%%%%%%%%%%%%%%%%%%%%%%%%%%%%%%%%%%
\paragraph{Pregunta inicial:} 
 Si tuvieras que organizar toda una biblioteca sin etiquetas, solo viendo los libros, ¿cómo agruparías los más parecidos entre sí?


%%%%%%%%%%%%%%%%%%%%%%%%%%%%%%%%%%%%%%%%
\section*{Desarrollo}

%%%%%%%%%%%%%%%%%%%%%%%%%%%%%%%%%%%%%%%%
\subsection*{Actividad 1: Presentación teórica sobre algoritmos de agrupamiento jerárquico}

Se introducirá a los estudiantes los fundamentos del agrupamiento jerárquico, incluyendo los métodos aglomerativos y divisivos, y los criterios de enlace para calcular las distancias entre grupos. También se explicará el uso de los dendogramas como herramienta visual para entender las jerarquías de los grupos.

\paragraph{¿Cómo lo haremos?}  
\begin{itemize}[leftmargin=*]
    \item \textbf{Introducción a los algoritmos jerárquicos:}  
    Explicaremos los principios de los algoritmos aglomerativos y divisivos. Se destacará cómo los métodos aglomerativos comienzan con cada punto como un clúster individual y los unen iterativamente, mientras que los divisivos parten de un clúster único y lo dividen sucesivamente.
    
    \item \textbf{Dendogramas:}  
    Introduciremos los dendogramas como una herramienta para visualizar la jerarquía de los agrupamientos. Analizaremos cómo interpretar las uniones y las distancias representadas en el gráfico.
    
    \item \textbf{Criterios de enlace:}  
    Explicaremos los tres criterios principales para determinar las distancias entre clústeres:
    \begin{itemize}
        \item \textbf{Enlace sencillo (Single Linkage):} Distancia mínima entre dos clústeres.
        \item \textbf{Enlace completo (Complete Linkage):} Distancia máxima entre dos clústeres.
        \item \textbf{Enlace promedio (Average Linkage):} Promedio de todas las distancias entre puntos de los dos clústeres.
    \end{itemize}

\end{itemize}

\paragraph{Verificación de aprendizaje:}  
Los estudiantes deberán:
\begin{itemize}[leftmargin=*]
    \item Identificar correctamente las diferencias entre los métodos aglomerativos y divisivos.
    \item Explicar qué representa un dendograma y cómo se interpreta.
    \item Comparar los efectos de los diferentes criterios de enlace en un ejemplo dado.
\end{itemize}

%%%%%%%%%%%%%%%%%%%%%%%%%%%%%%%%%%%%%%%%
\subsection*{Actividad 2 : Agrupamiento jerárquico con una lista de números}

Se realizará un ejercicio práctico para agrupar una lista de números utilizando un enfoque jerárquico. Se construirán grupos utilizando diferentes criterios de enlace y se visualizará el proceso en un dendograma manual.

\paragraph{¿Cómo lo haremos?}  
\begin{itemize}[leftmargin=*]
    \item \textbf{Datos iniciales:}  
    Se trabajará con la siguiente lista de números: \\[1mm]
    \texttt{[10, 4, 20, 30, 38, 87, 82, 56, 66, 70]}
    
    \item \textbf{Cálculo de distancias:}  
    Se calcularán las distancias absolutas entre todos los pares de números para construir una matriz de distancias inicial. La distancia entre dos números será la diferencia absoluta entre ellos.
    
    \item \textbf{Construcción del dendograma:}  
    Se fusionarán los números o grupos más cercanos según el criterio de enlace seleccionado:
    \begin{itemize}
        \item \textbf{Enlace sencillo:} Se tomará la distancia mínima entre elementos de los dos grupos.
        \item \textbf{Enlace completo:} Se tomará la distancia máxima entre elementos de los dos grupos.
        \item \textbf{Enlace promedio:} Se calculará el promedio de todas las distancias entre elementos de los dos grupos.
    \end{itemize}
    
    \item \textbf{Representación manual:}  
    A medida que se fusionan los grupos, se registrarán los pasos y se dibujará un dendograma manualmente en el pizarrón o en hojas de trabajo distribuidas a los estudiantes.
\end{itemize}

\paragraph{Verificación de aprendizaje:}  
Replicar el proceso con otro conjunto de números.

%%%%%%%%%%%%%%%%%%%%%%%%%%%%%%%%%%%%%%%%
\subsection*{Actividad 3 : Visualización de video sobre algoritmos de agrupamiento jerárquico}

Se utilizará un video educativo para que los estudiantes comprendan los conceptos fundamentales del agrupamiento jerárquico, incluyendo los métodos aglomerativos y divisivos, los criterios de enlace, y la interpretación de dendogramas.

\paragraph{¿Cómo lo haremos?}  
\begin{itemize}[leftmargin=*]
    \item \textbf{Proyección de video:}  
    Se presentará un video educativo que explique los fundamentos del agrupamiento jerárquico.
    \begin{quote}
        Enlace al video: \href{https://youtu.be/8QCBl-xdeZI?si=TYfeBv0pZGRS_0zS}{Hierarchical Cluster Analysis}.
    \end{quote}
    \item \textbf{Interacción con ChatGPT:}  
    Después del video, cada estudiante formulará preguntas a ChatGPT relacionadas con términos o conceptos que no comprendieron durante la visualización.
\end{itemize}

\paragraph{Verificación de aprendizaje:}  
Se realizará una discusión grupal donde los estudiantes compartirán:
\begin{itemize}[leftmargin=*]
    \item Las preguntas que realizaron a ChatGPT.
    \item Las respuestas que obtuvieron y cómo estas les ayudaron a comprender mejor el tema.
\end{itemize}

%%%%%%%%%%%%%%%%%%%%%%%%%%%%%%%%%%%%%%%%
\subsection*{Actividad 4: Implementación del agrupamiento jerárquico}

Los estudiantes utilizarán un cuaderno de Jupyter para implementar y experimentar con algoritmos de agrupamiento jerárquico sobre varios conjuntos de datos, observando cómo las métricas y criterios de enlace afectan los resultados.

\paragraph{¿Cómo lo haremos?}  
\begin{itemize}[leftmargin=*]
    \item \textbf{Exploración del cuaderno de Jupyter:}  
    Se proporcionará un cuaderno de Jupyter con ejemplos iniciales de implementación de agrupamiento jerárquico.
    \begin{quote}
        Enlace al cuaderno: \href{https://colab.research.google.com/github/andres-merino/AprendizajeAutomaticoInicial-05-N0105/blob/main/2-Notebooks/08-Agrupamiento-Jerarquico.ipynb}{08-Agrupamiento-Jerarquico.ipynb}.
    \end{quote}

    \item \textbf{Modificación del código:}  
    Los estudiantes experimentarán modificando las siguientes variables y funciones:
    \begin{itemize}
        \item \textbf{Criterios de enlace:} Cambiar entre «single», «complete» y «average».
        \item \textbf{Métricas de distancia:} Probar métricas como «euclidean», «manhattan» o «cosine».
        \item \textbf{Conjuntos de datos:} Aplicar el agrupamiento jerárquico sobre al menos tres datasets diferentes provistos en el cuaderno.
    \end{itemize}

    \item \textbf{Visualización de resultados:}  
    Utilizar los dendogramas generados para interpretar las estructuras jerárquicas de los datos.
\end{itemize}

\paragraph{Verificación de aprendizaje:}  
Los estudiantes deberán completar los ejercicios del cuaderno y responder a las siguientes preguntas:
    \begin{itemize}
        \item ¿Qué impacto tienen las métricas de distancia y los criterios de enlace en la forma final de los dendogramas?
        \item ¿Qué criterio de enlace produjo resultados más consistentes para los datasets analizados?
        \item ¿Qué observaciones se pueden hacer sobre la jerarquía en los datos de cada conjunto?
    \end{itemize}
%%%%%%%%%%%%%%%%%%%%%%%%%%%%%%%%%%%%%%%%






%%%%%%%%%%%%%%%%%%%%%%%%%%%%%%%%%%%%%%%%
\section*{Cierre}

\paragraph{Tarea:}  
    Desarrollar los ejercicios planteados en el siguiente cuaderno y entregarlo por el aula virtual:
    \begin{quote}
        Enlace al cuaderno: \href{https://colab.research.google.com/github/andres-merino/AprendizajeAutomaticoInicial-05-N0105/blob/main/2-Ejercicios/03-Agrupamiento-Jerarquico.ipynb}{03-Agrupamiento-Jerarquico.ipynb}.
    \end{quote}

\paragraph{Pregunta de investigación:}  
\begin{enumerate}[leftmargin=*]
    \item ¿Qué información podemos obtener de un dendograma que no podemos obtener directamente de otros modelos?
    \item ¿Cómo afecta el criterio de enlace (sencillo, completo o promedio) a la forma final de los grupos en el agrupamiento jerárquico?
    \item ¿En qué casos prácticos sería más útil un método divisivo en lugar de uno aglomerativo?
\end{enumerate}
    


\end{document} 