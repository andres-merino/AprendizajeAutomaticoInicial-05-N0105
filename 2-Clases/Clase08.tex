\documentclass[a4,11pt]{aleph-notas}

% -- Paquetes adicionales 
\usepackage{enumitem}
\usepackage{url}
\usepackage{array}
\usepackage{booktabs}
\hypersetup{
    urlcolor=blue,
    linkcolor=blue,
}

% -- Datos 
\institucion{Facultad de Ciencias Exactas, Naturales y Ambientales}
\carrera{Ciencia de Datos}
\asignatura{Aprendizaje Automático Inicial}
\tema{Clase 08: Algoritmos de Agrupamiento Jerárquico}
\autor{Andrés Merino}
\fecha{Periodo 2025-2}

\logouno[0.14\textwidth]{Logos/logoPUCE_04_ac}
\definecolor{colortext}{HTML}{0030A1}
\definecolor{colordef}{HTML}{0030A1}
\fuente{montserrat}

% -- Comandos adicionales (comentados si no se usan)
% \usepackage{listings}
\input{listings-python.prf}

\usepackage[spanish,onelanguage,vlined,linesnumbered]{algorithm2e}

% -- Comandos adicionales
\newtcolorbox{pscodigo}
    {icono=\faCogs,color=lightgray,postit,top=-1.5mm,bottom=-1.5mm}
    
\definecolor{colcod}{RGB}{174,218,255}
\newtcolorbox{pycodigo}
    {icono=\faKeyboardO, color=colcod, postit, 
    top=-2mm, bottom=-2mm, 
    extras first={bottom=0mm},
    extras last={top=0mm},
    extras middle={top=0mm,bottom=0mm},
    }

\lstloadlanguages{Python}
\lstset{
  language=Python,
  basicstyle=\small\sffamily,
  stringstyle=\color[HTML]{933797},
  commentstyle=\color[HTML]{228B22}\sffamily,
  emph={[2]from,import,pass,return}, emphstyle={[2]\color[HTML]{DD52F0}},
  emph={[3]range}, emphstyle={[3]\color[HTML]{D17032}},
  emph={[4]for,in,def}, emphstyle={[4]\color{blue}},
  showstringspaces=false,
  breaklines=true,
  prebreak=\mbox{{\color{gray}\tiny$\searrow$}},
  xleftmargin=3pt,
  inputencoding=utf8,
  extendedchars=true,
  columns=fullflexible,
  literate={á}{{\'a}}1 {é}{{\'e}}1 {í}{{\'i}}1 {ó}{{\'o}}1 {ú}{{\'u}}1,
}


\SetKwFunction{concat}{Concatenar}
\SetKwProg{Fn}{Función}{\string:}{}
\SetKwFunction{ult}{Ultimo}
\SetKwFunction{pri}{Primero}
\SetKwFunction{sinul}{SinUltimo}
\SetKw{Salir}{Salir}

\newcommand{\fuentecomentario}[1]{\scriptsize\ttfamily #1}
\SetCommentSty{fuentecomentario}
\SetAlFnt{\footnotesize}


\begin{document}

\encabezado
% En todo el documento, las indicaciones deben ser simples y directas, con una sola oración.

%%%%%%%%%%%%%%%%%%%%%%%%%%%%%%%%%%%%%%%%
\section*{Resultado de Aprendizaje}
%%%%%%%%%%%%%%%%%%%%%%%%%%%%%%%%%%%%%%%%

%%%%%%%%%%%%%%%%%%%%%%%%%%%%%%%%%%%%%%%%
\subsection*{RdA de la asignatura:}
% Se toma uno de los siguientes
\begin{itemize}[leftmargin=*]
    \item \textbf{RdA 2:} Aplicar modelos de aprendizaje automático supervisado y no supervisado, así como su validación y optimización, en la resolución de problemas tanto reales como simulados.
\end{itemize}

%%%%%%%%%%%%%%%%%%%%%%%%%%%%%%%%%%%%%%%%
\subsection*{RdA de la clase:}
% Máximo 3 resultados
\begin{itemize}[leftmargin=*]
    \item Describir y diferenciar los algoritmos jerárquicos aglomerativos y divisivos, junto con sus criterios de enlace.
    \item Analizar y construir dendogramas para entender los resultados de un agrupamiento jerárquico.
    \item Implementar algoritmos de agrupamiento jerárquico en un conjunto de datos utilizando Python.
\end{itemize}

%%%%%%%%%%%%%%%%%%%%%%%%%%%%%%%%%%%%%%%%
\section*{Introducción}
%%%%%%%%%%%%%%%%%%%%%%%%%%%%%%%%%%%%%%%%

%%%%%%%%%%%%%%%%%%%%%%%%%%%%%%%%%%%%%%%%
\paragraph{Pregunta inicial:} 
A veces queremos agrupar a los Pokémon solo en «Legendarios vs. Comunes» (2 grupos), pero otras veces necesitamos dividirlos por sus 18 Tipos elementales. ¿Existe algún algoritmo que no nos obligue a decidir el número de grupos desde el principio, sino que nos permita «cortar» la clasificación al nivel de detalle que nosotros queramos después de ver los datos?

%%%%%%%%%%%%%%%%%%%%%%%%%%%%%%%%%%%%%%%%
\section*{Desarrollo}
%%%%%%%%%%%%%%%%%%%%%%%%%%%%%%%%%%%%%%%%

%%%%%%%%%%%%%%%%%%%%%%%%%%%%%%%%%%%%%%%%
\subsection*{Actividad 1: Presentación teórica sobre algoritmos de agrupamiento jerárquico}

Se introducirá a los estudiantes los fundamentos del agrupamiento jerárquico mediante clase magistral, incluyendo los métodos aglomerativos y divisivos, los criterios de enlace para calcular distancias entre grupos, y el uso de dendogramas como herramienta visual.

\paragraph{¿Cómo lo haremos?}  
\begin{itemize}[leftmargin=*]
    \item \textbf{Clase magistral:}  
    Se explicarán los conceptos teóricos del agrupamiento jerárquico, incluyendo:
    \begin{itemize}
        \item Métodos aglomerativos vs. divisivos.
        \item Criterios de enlace: simple, completo y promedio.
        \item Interpretación de dendogramas.
    \end{itemize}

    \item \textbf{Materiales de apoyo:} 
    Se utilizará el documento \href{https://andres-merino.github.io/AprendizajeAutomaticoInicial-05-N0105/2-Resumenes/Resumen08.pdf}{Resumen08.pdf}
    
    \item \textbf{Implementación en Python:} Los estudiantes accederán a un cuaderno de Jupyter previamente preparado.
    \begin{quote}
        Enlace al cuaderno: \href{https://github.com/andres-merino/AprendizajeAutomaticoInicial-05-N0105/blob/main/2-Notebooks/08-Agrupamiento-Jerarquico.ipynb}{08-Agrupamiento-Jerarquico.ipynb}.
    \end{quote}
    
\end{itemize}

%%%%%%%%%%%%%%%%%%%%%%%%%%%%%%%%%%%%%%%%
\subsection*{Actividad 2: Agrupamiento jerárquico con una lista de números}

Se realizará un ejercicio práctico manual para agrupar una lista de números utilizando un enfoque jerárquico, construyendo grupos con diferentes criterios de enlace y visualizando el proceso en un dendograma.

\paragraph{¿Cómo lo haremos?}  
\begin{itemize}[leftmargin=*]
    \item \textbf{Datos iniciales:}  
    Se trabajará con la siguiente lista de números: \texttt{[10, 4, 20, 30, 38, 87, 82, 56, 66, 70]}.
    
    \item \textbf{Representación manual:}  
    A medida que se fusionan los grupos, se registrarán los pasos y se dibujará un dendograma manualmente en el pizarrón o en hojas de trabajo distribuidas a los estudiantes.
    
\end{itemize}

%%%%%%%%%%%%%%%%%%%%%%%%%%%%%%%%%%%%%%%%
\subsection*{Actividad 3: Visualización de video sobre algoritmos de agrupamiento jerárquico}

Se utilizará un video educativo mediante proyección y discusión para que los estudiantes comprendan los conceptos fundamentales del agrupamiento jerárquico, incluyendo los métodos aglomerativos y divisivos, los criterios de enlace, y la interpretación de dendogramas.

\paragraph{¿Cómo lo haremos?}  
\begin{itemize}[leftmargin=*]
    \item \textbf{Proyección de video:}  
    Se presentará un video educativo que explique los fundamentos del agrupamiento jerárquico.
    \begin{quote}
        Enlace al video: \href{https://youtu.be/8QCBl-xdeZI?si=TYfeBv0pZGRS_0zS}{Hierarchical Cluster Analysis}.
    \end{quote}
    
    \item \textbf{Interacción con ChatGPT:}  
    Después del video, cada estudiante formulará preguntas a ChatGPT relacionadas con términos o conceptos que no comprendieron durante la visualización.
    
    \item \textbf{Discusión grupal:}
    Los estudiantes compartirán las preguntas que realizaron a ChatGPT y las respuestas que obtuvieron.
    
\end{itemize}

%%%%%%%%%%%%%%%%%%%%%%%%%%%%%%%%%%%%%%%%
\subsection*{Actividad 4: Implementación del agrupamiento jerárquico}

Los estudiantes utilizarán un cuaderno de Jupyter mediante exploración práctica para implementar y experimentar con algoritmos de agrupamiento jerárquico sobre varios conjuntos de datos, observando cómo las métricas y criterios de enlace afectan los resultados.

\paragraph{¿Cómo lo haremos?}  
\begin{itemize}[leftmargin=*]
    \item \textbf{Explicación de la experimentación:}  
    Los estudiantes experimentarán modificando criterios de enlace (single, complete, average), métricas de distancia (euclidean, manhattan, cosine) y aplicando el agrupamiento sobre diferentes datasets.
    
    \item \textbf{Implementación en Python:} Los estudiantes accederán a un cuaderno de Jupyter previamente preparado.
    \begin{quote}
        Enlace al cuaderno: \href{https://github.com/andres-merino/AprendizajeAutomaticoInicial-05-N0105/blob/main/2-Notebooks/08-Agrupamiento-Jerarquico.ipynb}{08-Agrupamiento-Jerarquico.ipynb}.
    \end{quote}
    
    \item \textbf{Experimentación:} Genera el agrupamiento con diferentes métricas y métodos de enlaces. ¿Qué diferencias observas?
    
\end{itemize}

%%%%%%%%%%%%%%%%%%%%%%%%%%%%%%%%%%%%%%%%
\section*{Cierre}
%%%%%%%%%%%%%%%%%%%%%%%%%%%%%%%%%%%%%%%%

\paragraph{Verificación de aprendizaje:} 
\begin{enumerate}[leftmargin=*]
    \item ¿Cuál es la diferencia principal entre los métodos aglomerativos y divisivos en agrupamiento jerárquico?
    
    \item ¿Qué información proporciona un dendograma y cómo se interpreta?
    
    \item ¿Cómo difieren los tres criterios de enlace principales (simple, completo y promedio) y cuándo usaría cada uno?
\end{enumerate}

\paragraph{Preguntas tipo entrevista:} 
\begin{enumerate}[leftmargin=*]
    \item Tienes un dataset de clientes con múltiples características. Aplicaste clustering jerárquico con enlace simple y obtuviste un cluster gigante con casi todos los datos. ¿Qué problema está ocurriendo y cómo lo resolverías?
    
    \item ¿Qué ventajas tiene el clustering jerárquico divisivo sobre el aglomerativo en términos de eficiencia computacional y calidad de los clusters formados?
\end{enumerate}

\paragraph{Tarea:}  
Desarrollar los ejercicios planteados en el siguiente cuaderno y entregarlo por el aula virtual:
\begin{quote}
    Enlace al cuaderno: \href{https://colab.research.google.com/github/andres-merino/AprendizajeAutomaticoInicial-05-N0105/blob/main/2-Ejercicios/03-Agrupamiento-Jerarquico.ipynb}{03-Agrupamiento-Jerarquico.ipynb}.
\end{quote}

\paragraph{Pregunta de investigación:}  
\begin{enumerate}[leftmargin=*]
    \item ¿En qué casos prácticos sería más útil un método divisivo en lugar de uno aglomerativo?
    \item ¿Cómo afecta la elección de la métrica de distancia (euclidiana, Manhattan, coseno) en los resultados del agrupamiento jerárquico?
\end{enumerate}
    
\paragraph{Para la próxima clase:}  Leer la sección 12.1 del artículo \href{https://link.springer.com/article/10.1140/epjs/s11734-021-00209-7}{Beginning with machine learning: a comprehensive primer}.


\end{document}