\documentclass[a4,11pt]{aleph-notas}
% Se puede ver la documentación aquí: 
% https://github.com/alephsub0/LaTeX_aleph-notas

% -- Paquetes adicionales 
\usepackage{enumitem}
\usepackage{url}
\usepackage{array}
\usepackage{booktabs}
\usepackage{longtable}
\usepackage{environ}  % Para definir nuevos entornos
\hypersetup{
    urlcolor=blue,
    linkcolor=blue,
}

% Definición del nuevo ambiente
\NewEnviron{materiales}{%
    \begin{minipage}{12cm}
        \vspace{0.1mm}
        \begin{itemize}[leftmargin=*]
            \BODY % Contenido de la lista
        \end{itemize}
        \vspace{1mm}
    \end{minipage}
}


% -- Datos 
\institucion{Facultad de Ciencias Exactas, Naturales y Ambientales}
\carrera{Ciencia de Datos}
\asignatura{Aprendizaje Automático Inicial}
\tema{Clase 17: Árboles de decisión y métodos combinados}
\autor{Andrés Merino}
\fecha{Periodo 2025-2}

\logouno[0.14\textwidth]{Logos/logoPUCE_04_ac}
\definecolor{colortext}{HTML}{0030A1}
\definecolor{colordef}{HTML}{0030A1}
\fuente{montserrat}


% -- Comandos para tablas
\usepackage{listings}
\input{listings-python.prf}

\usepackage[spanish,onelanguage,vlined,linesnumbered]{algorithm2e}

% -- Comandos adicionales
\newtcolorbox{pscodigo}
    {icono=\faCogs,color=lightgray,postit,top=-1.5mm,bottom=-1.5mm}
    
\definecolor{colcod}{RGB}{174,218,255}
\newtcolorbox{pycodigo}
    {icono=\faKeyboardO, color=colcod, postit, 
    top=-2mm, bottom=-2mm, 
    extras first={bottom=0mm},
    extras last={top=0mm},
    extras middle={top=0mm,bottom=0mm},
    }

\lstloadlanguages{Python}
\lstset{
  language=Python,
  basicstyle=\small\sffamily,
  stringstyle=\color[HTML]{933797},
  commentstyle=\color[HTML]{228B22}\sffamily,
  emph={[2]from,import,pass,return}, emphstyle={[2]\color[HTML]{DD52F0}},
  emph={[3]range}, emphstyle={[3]\color[HTML]{D17032}},
  emph={[4]for,in,def}, emphstyle={[4]\color{blue}},
  showstringspaces=false,
  breaklines=true,
  prebreak=\mbox{{\color{gray}\tiny$\searrow$}},
  xleftmargin=3pt,
  inputencoding=utf8,
  extendedchars=true,
  columns=fullflexible,
  literate={á}{{\'a}}1 {é}{{\'e}}1 {í}{{\'i}}1 {ó}{{\'o}}1 {ú}{{\'u}}1,
}


\SetKwFunction{concat}{Concatenar}
\SetKwProg{Fn}{Función}{\string:}{}
\SetKwFunction{ult}{Ultimo}
\SetKwFunction{pri}{Primero}
\SetKwFunction{sinul}{SinUltimo}
\SetKw{Salir}{Salir}

\newcommand{\fuentecomentario}[1]{\scriptsize\ttfamily #1}
\SetCommentSty{fuentecomentario}
\SetAlFnt{\footnotesize}


\begin{document}

\encabezado


%%%%%%%%%%%%%%%%%%%%%%%%%%%%%%%%%%%%%%%%
\section*{Resultado de Aprendizaje}
%%%%%%%%%%%%%%%%%%%%%%%%%%%%%%%%%%%%%%%%

%%%%%%%%%%%%%%%%%%%%%%%%%%%%%%%%%%%%%%%%
\subsection*{RdA de la asignatura:}
\begin{itemize}[leftmargin=*]
    % \item \textbf{RdA 1:} Plantear los conceptos fundamentales del aprendizaje automático, incluyendo los principios básicos, técnicas de preprocesado de datos, métodos de evaluación y ajuste de modelos, destacando su importancia en el análisis y resolución de problemas de datos.
    \item \textbf{RdA 2:} 
    Aplicar modelos de aprendizaje automático supervisado y no supervisado, así como su validación y optimización, en la resolución de problemas tanto reales como simulados.
    % \item \textbf{RdA 3:} Resolver problemas prácticos mediante el uso de modelos de aprendizaje automático, ajustándolos para la mejora de su rendimiento y precisión.
\end{itemize}

%%%%%%%%%%%%%%%%%%%%%%%%%%%%%%%%%%%%%%%%
\subsection*{RdA de la clase:}
\begin{itemize}[leftmargin=*]
    \item Comprender el funcionamiento de los métodos combinados como bagging, boosting y bosques aleatorios.  
    \item Implementar bosques aleatorios y boosting en un entorno práctico, evaluando resultados.  
\end{itemize}

%%%%%%%%%%%%%%%%%%%%%%%%%%%%%%%%%%%%%%%%
\section*{Introducción}
%%%%%%%%%%%%%%%%%%%%%%%%%%%%%%%%%%%%%%%%

%%%%%%%%%%%%%%%%%%%%%%%%%%%%%%%%%%%%%%%%
\paragraph{Pregunta inicial:} 
¿Por qué combinar varios modelos puede mejorar el rendimiento de un modelo de aprendizaje automático?  

%%%%%%%%%%%%%%%%%%%%%%%%%%%%%%%%%%%%%%%%
\section*{Desarrollo}
%%%%%%%%%%%%%%%%%%%%%%%%%%%%%%%%%%%%%%%%

%%%%%%%%%%%%%%%%%%%%%%%%%%%%%%%%%%%%%%%%  
\subsection*{Actividad 1: Clase magistral sobre métodos combinados}  
%%%%%%%%%%%%%%%%%%%%%%%%%%%%%%%%%%%%%%%%  


En esta actividad, los estudiantes conocerán las bases teóricas y conceptuales de los métodos combinados en aprendizaje automático: bagging, boosting y bosques aleatorios. Se destacará cómo estos métodos combinan múltiples modelos individuales para mejorar la precisión, reducir la varianza y manejar problemas de sobreajuste. Se usarán ejemplos y discusiones guiadas para consolidar los conceptos.

\paragraph{¿Cómo lo haremos?}  
\begin{itemize}[leftmargin=*]  
    \item \textbf{Exposición teórica: }
    Presentación de los conceptos clave:
    \begin{itemize}  
        \item \textbf{Bagging (Bootstrap Aggregating):} Cómo se generan múltiples modelos a partir de muestras bootstrap para reducir la varianza. Se destacará el uso de árboles de decisión como modelo base.
        \item \textbf{Boosting:} Explicación del aprendizaje secuencial, donde cada modelo corrige los errores de los anteriores.
        \item \textbf{Bosques aleatorios:} Introducción a la técnica de combinación de árboles de decisión, resaltando el uso de selección aleatoria de atributos para mejorar la generalización.
    \end{itemize}  

\end{itemize}  

\paragraph{Verificación de aprendizaje:}  
\begin{itemize}[leftmargin=*]  
    \item ¿Qué diferencias fundamentales existen entre bagging y boosting?  
    \item ¿Cómo ayudan los bosques aleatorios a reducir el sobreajuste en comparación con un único árbol de decisión?  
    \item ¿Cuáles son las principales ventajas de combinar múltiples modelos en problemas de aprendizaje automático?  
\end{itemize}  

\subsection*{Actividad 2: Bosques Aleatorios}

En esta actividad, los estudiantes implementarán un modelo de bosques aleatorios para explorar cómo combina múltiples árboles de decisión para reducir la varianza y mejorar la precisión. Se analizará el impacto de los hiperparámetros, como el número de estimadores y la selección aleatoria de atributos, en el rendimiento del modelo.

\paragraph{¿Cómo lo haremos?}  
\begin{itemize}[leftmargin=*]
    \item \textbf{Implementación práctica:}  
    Los estudiantes accederán a un cuaderno de Jupyter preparado para esta actividad. El cuaderno incluye un ejemplo de clasificación utilizando un conjunto de datos real y simulado.  
    \begin{quote}
        Enlace al cuaderno: \href{https://colab.research.google.com/github/andres-merino/AprendizajeAutomaticoInicial-05-N0105/blob/main/2-Notebooks/17_1-Bosques-Aleatorios.ipynb}{17\_1-Bosques-Aleatorios.ipynb}.
    \end{quote}
\end{itemize}

\paragraph{Verificación de aprendizaje:}  
\begin{itemize}[leftmargin=*]
    \item ¿Qué parámetros afectan el rendimiento de un modelo de bosque aleatorio?   
\end{itemize}  

\subsection*{Actividad 3: Boosting y Bagging}

Los estudiantes implementarán modelos de boosting y bagging para entender cómo estas técnicas combinan modelos débiles para mejorar el rendimiento. Se compararán las diferencias en el enfoque de aprendizaje (paralelo vs. secuencial) y su impacto en sesgo y varianza.

\paragraph{¿Cómo lo haremos?}  
\begin{itemize}[leftmargin=*]
    \item \textbf{Implementación práctica:}  
    Los estudiantes trabajarán con un cuaderno de Jupyter para implementar un ejemplo básico de bagging y boosting, ajustando parámetros como el número de estimadores y la tasa de aprendizaje.
    \begin{quote}
        Enlace al cuaderno: \href{https://colab.research.google.com/github/andres-merino/AprendizajeAutomaticoInicial-05-N0105/blob/main/2-Notebooks/17_2-Boosting-Bagging.ipynb}{17\_2-Boosting-Bagging.ipynb}.
    \end{quote}
\end{itemize}


%%%%%%%%%%%%%%%%%%%%%%%%%%%%%%%%%%%%%%%%  
\section*{Cierre}  
%%%%%%%%%%%%%%%%%%%%%%%%%%%%%%%%%%%%%%%%  

\paragraph{Tarea:}  
Implementar un Bosque aleatorio utilizando un conjunto de datos real y comparar los resultados al variar los criterios de partición, poda y profundidad máxima.  

\paragraph{Pregunta de investigación:}  
\begin{enumerate}[leftmargin=*]  
    \item ¿Cómo influyen los métodos combinados en la detección de anomalías en datasets con ruido?
    \item ¿Cómo podrían los métodos combinados adaptarse a datasets dinámicos o en streaming?
\end{enumerate}  



\end{document} 