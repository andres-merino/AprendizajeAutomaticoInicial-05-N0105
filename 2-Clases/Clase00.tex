\documentclass[a4,11pt]{aleph-notas}

% -- Paquetes adicionales 
\usepackage{enumitem}
\usepackage{url}
\usepackage{array}
\usepackage{booktabs}
\hypersetup{
    urlcolor=blue,
    linkcolor=blue,
}

% -- Datos 
\institucion{Facultad de Ciencias Exactas, Naturales y Ambientales}
\carrera{Ciencia de Datos}
\asignatura{Aprendizaje Automático Inicial}
\tema{Clase 00: Nombre de la clase}
\autor{Andrés Merino}
\fecha{Periodo 2025-2}

\logouno[0.14\textwidth]{Logos/logoPUCE_04_ac}
\definecolor{colortext}{HTML}{0030A1}
\definecolor{colordef}{HTML}{0030A1}
\fuente{montserrat}

\begin{document}

\encabezado
% En todo el documento, las indicaciones deben ser simples y directar, con una sola oración.

%%%%%%%%%%%%%%%%%%%%%%%%%%%%%%%%%%%%%%%%
\section*{Resultado de Aprendizaje}
%%%%%%%%%%%%%%%%%%%%%%%%%%%%%%%%%%%%%%%%

%%%%%%%%%%%%%%%%%%%%%%%%%%%%%%%%%%%%%%%%
\subsection*{RdA de la asignatura:}
% Se toma uno de los siguientes
\begin{itemize}[leftmargin=*]
    \item \textbf{RdA 1:} Plantear los conceptos fundamentales del aprendizaje automático, incluyendo los principios básicos, técnicas de preprocesado de datos, métodos de evaluación y ajuste de modelos, destacando su importancia en el análisis y resolución de problemas de datos.
    \item \textbf{RdA 2:} Aplicar modelos de aprendizaje automático supervisado y no supervisado, así como su validación y optimización, en la resolución de problemas tanto reales como simulados.
    \item \textbf{RdA 3:} Resolver problemas prácticos mediante el uso de modelos de aprendizaje automático, ajustándolos para la mejora de su rendimiento y precisión.
\end{itemize}

%%%%%%%%%%%%%%%%%%%%%%%%%%%%%%%%%%%%%%%%
\subsection*{RdA de la clase:}
% Máximo 3 resultados
\begin{itemize}[leftmargin=*]
    \item ...
\end{itemize}

%%%%%%%%%%%%%%%%%%%%%%%%%%%%%%%%%%%%%%%%
\section*{Introducción}
%%%%%%%%%%%%%%%%%%%%%%%%%%%%%%%%%%%%%%%%

%%%%%%%%%%%%%%%%%%%%%%%%%%%%%%%%%%%%%%%%
\paragraph{Pregunta inicial:} 
[Pregunta detonante]

%%%%%%%%%%%%%%%%%%%%%%%%%%%%%%%%%%%%%%%%
\section*{Desarrollo}
%%%%%%%%%%%%%%%%%%%%%%%%%%%%%%%%%%%%%%%%

%%%%%%%%%%%%%%%%%%%%%%%%%%%%%%%%%%%%%%%%
\subsection*{Actividad 1: Nombre de la actividad}

[Párrafo de descripción de la actividad, incluyendo metodología (clase magistral, clase invertida, retos, clase interactiva con ChatGPT, visualización de videos, investigación en clase etc.)]

\paragraph{¿Cómo lo haremos?}  
\begin{itemize}[leftmargin=*]
    \item \textbf{[Nombre del paso]:}  % 
    [Guía simple. Puede incluir temas a tratar, preguntas, enunciado de ejercicios, referencias a videos, talleres, código.]
    \item \textbf{Implementación en Python:} Los estudiantes accederán a un cuaderno de Jupyter previamente preparado.
    \begin{quote}
        Enlace al cuaderno: \href{https://github.com/andres-merino/AprendizajeAutomaticoInicial-05-N0105/blob/main/2-Notebooks/13_2-Perceptron-Multiclase.ipynb}{13\_2-Perceptron-Multiclase.ipynb}.
    \end{quote}
    \item \textbf{Experimentación:} 
    [Descripción de la experimentación]
    \begin{ejer}
    Modifique el código para:
    \begin{itemize}[leftmargin=*]
        \item Actividad 1
    \end{itemize}
    \end{ejer}
\end{itemize}

\subsection*{Actividad 2: Nombre de la actividad}

[Párrafo de descripción de la actividad, incluyendo metodología (clase magistral, clase invertida, retos, clase interactiva con ChatGPT, visualización de videos, investigación en clase etc.)]

\paragraph{¿Cómo lo haremos?}  
\begin{itemize}[leftmargin=*]
    \item \textbf{[Nombre del paso]:}  % 
    [Guía simple. Puede incluir temas a tratar, preguntas, enunciado de ejercicios, referencias a videos, talleres, código.]
    \item \textbf{Implementación en Python:} Los estudiantes accederán a un cuaderno de Jupyter previamente preparado.
    \begin{quote}
        Enlace al cuaderno: \href{https://github.com/andres-merino/AprendizajeAutomaticoInicial-05-N0105/blob/main/2-Notebooks/13_2-Perceptron-Multiclase.ipynb}{13\_2-Perceptron-Multiclase.ipynb}.
    \end{quote}
    \item \textbf{Experimentación:} 
    [Descripción de la experimentación]
    \begin{ejer}
    Modifique el código para:
    \begin{itemize}[leftmargin=*]
        \item Actividad 1
    \end{itemize}
    \end{ejer}
\end{itemize}

%%%%%%%%%%%%%%%%%%%%%%%%%%%%%%%%%%%%%%%%
\section*{Cierre}
%%%%%%%%%%%%%%%%%%%%%%%%%%%%%%%%%%%%%%%%

\paragraph{Verificación de aprendizaje:} [Indicar actividades cortas que evalúen el aprendizaje, pueden ser una lista de 3 preguntas]

\paragraph{Preguntas tipo entrevista:} [Dos preguntas tipo entrevista sobre los temas, una de ellas debe ser una «pregunta trampa»]
\begin{enumerate}[leftmargin=*]
    \item ...
    \item ...
\end{enumerate}

\paragraph{Tarea:} [Plantear resolución de ejercicio o lectura o similar.]

\paragraph{Pregunta de investigación:} [Dos o tres preguntas que guíen al estudiante a profundizar más en los temas o buscar aplciaciones]
\begin{enumerate}[leftmargin=*]
    \item ...
    \item ...
\end{enumerate}
    
\paragraph{Para la próxima clase:}  [Actividad para la próxima clase]


\end{document} 