\documentclass[a4,11pt]{aleph-notas}
% Se puede ver la documentación aquí: 
% https://github.com/alephsub0/LaTeX_aleph-notas

% -- Paquetes adicionales 
\usepackage{enumitem}
\usepackage{url}
\usepackage{array}
\usepackage{booktabs}
\usepackage{longtable}
\usepackage{environ}  % Para definir nuevos entornos
\hypersetup{
    urlcolor=blue,
    linkcolor=blue,
}

% Definición del nuevo ambiente
\NewEnviron{materiales}{%
    \begin{minipage}{12cm}
        \vspace{0.1mm}
        \begin{itemize}[leftmargin=*]
            \BODY % Contenido de la lista
        \end{itemize}
        \vspace{1mm}
    \end{minipage}
}


% -- Datos 
\institucion{Escuela de Ciencias Físicas y Matemática}
\carrera{Ciencia de Datos}
\asignatura{Aprendizaje Automático Inicial}
\tema{Clase 18: Validación cruzada y Optimización de Hiperparámetros}
\autor{Andrés Merino}
\fecha{Semestre 2025-1}

\logouno[0.14\textwidth]{Logos/logoPUCE_04_ac}
\definecolor{colortext}{HTML}{0030A1}
\definecolor{colordef}{HTML}{0030A1}
\fuente{montserrat}


% -- Comandos para tablas
\usepackage{listings}
\input{listings-python.prf}

\usepackage[spanish,onelanguage,vlined,linesnumbered]{algorithm2e}

% -- Comandos adicionales
\newtcolorbox{pscodigo}
    {icono=\faCogs,color=lightgray,postit,top=-1.5mm,bottom=-1.5mm}
    
\definecolor{colcod}{RGB}{174,218,255}
\newtcolorbox{pycodigo}
    {icono=\faKeyboardO, color=colcod, postit, 
    top=-2mm, bottom=-2mm, 
    extras first={bottom=0mm},
    extras last={top=0mm},
    extras middle={top=0mm,bottom=0mm},
    }

\lstloadlanguages{Python}
\lstset{
  language=Python,
  basicstyle=\small\sffamily,
  stringstyle=\color[HTML]{933797},
  commentstyle=\color[HTML]{228B22}\sffamily,
  emph={[2]from,import,pass,return}, emphstyle={[2]\color[HTML]{DD52F0}},
  emph={[3]range}, emphstyle={[3]\color[HTML]{D17032}},
  emph={[4]for,in,def}, emphstyle={[4]\color{blue}},
  showstringspaces=false,
  breaklines=true,
  prebreak=\mbox{{\color{gray}\tiny$\searrow$}},
  xleftmargin=3pt,
  inputencoding=utf8,
  extendedchars=true,
  columns=fullflexible,
  literate={á}{{\'a}}1 {é}{{\'e}}1 {í}{{\'i}}1 {ó}{{\'o}}1 {ú}{{\'u}}1,
}


\SetKwFunction{concat}{Concatenar}
\SetKwProg{Fn}{Función}{\string:}{}
\SetKwFunction{ult}{Ultimo}
\SetKwFunction{pri}{Primero}
\SetKwFunction{sinul}{SinUltimo}
\SetKw{Salir}{Salir}

\newcommand{\fuentecomentario}[1]{\scriptsize\ttfamily #1}
\SetCommentSty{fuentecomentario}
\SetAlFnt{\footnotesize}


\begin{document}

\encabezado


%%%%%%%%%%%%%%%%%%%%%%%%%%%%%%%%%%%%%%%%
\section*{Resultado de Aprendizaje}
%%%%%%%%%%%%%%%%%%%%%%%%%%%%%%%%%%%%%%%%

%%%%%%%%%%%%%%%%%%%%%%%%%%%%%%%%%%%%%%%%
\subsection*{RdA de la asignatura:}
\begin{itemize}[leftmargin=*]
    % \item \textbf{RdA 1:} Plantear los conceptos fundamentales del aprendizaje automático, incluyendo los principios básicos, técnicas de preprocesado de datos, métodos de evaluación y ajuste de modelos, destacando su importancia en el análisis y resolución de problemas de datos.
    % \item \textbf{RdA 2:} Aplicar modelos de aprendizaje automático supervisado y no supervisado, así como su validación y optimización, en la resolución de problemas tanto reales como simulados.
    \item \textbf{RdA 3:} 
    Resolver problemas prácticos mediante el uso de modelos de aprendizaje automático, ajustándolos para la mejora de su rendimiento y precisión.
\end{itemize}

%%%%%%%%%%%%%%%%%%%%%%%%%%%%%%%%%%%%%%%%
\subsection*{RdA de la actividad:}
\begin{itemize}[leftmargin=*]
    \item Comprender los conceptos de validación cruzada y su importancia en la evaluación de modelos.
    \item Aplicar técnicas de optimización de hiperparámetros (Grid Search y Random Search) en modelos de aprendizaje automático.
    \item Implementar el guardado y carga de modelos entrenados para reutilización.
    \item Diseñar un esquema de trabajo estructurado para proyectos de aprendizaje automático.
\end{itemize}

%%%%%%%%%%%%%%%%%%%%%%%%%%%%%%%%%%%%%%%%
\section*{Introducción}
%%%%%%%%%%%%%%%%%%%%%%%%%%%%%%%%%%%%%%%%

%%%%%%%%%%%%%%%%%%%%%%%%%%%%%%%%%%%%%%%%
\paragraph{Pregunta inicial:} 
¿Cómo evaluamos correctamente un modelo para asegurarnos de que no está sobreajustado o subajustado?



%%%%%%%%%%%%%%%%%%%%%%%%%%%%%%%%%%%%%%%%
\section*{Desarrollo}
%%%%%%%%%%%%%%%%%%%%%%%%%%%%%%%%%%%%%%%%

%%%%%%%%%%%%%%%%%%%%%%%%%%%%%%%%%%%%%%%%
%%%%%%%%%%%%%%%%%%%%%%%%%%%%%%%%%%%%%%%%
\subsection*{Actividad 1: Validación cruzada}
%%%%%%%%%%%%%%%%%%%%%%%%%%%%%%%%%%%%%%%%

Esta actividad se centra en comprender la validación cruzada y su importancia en la evaluación de modelos. Se explorarán diferentes tipos de validación cruzada y su impacto en las métricas de desempeño del modelo.

\paragraph{¿Cómo lo haremos?}  
\begin{itemize}[leftmargin=*]
    \item \textbf{Clase magistral:} Se cubrirán los siguientes conceptos clave:
    \begin{itemize}
        \item Propósito de la validación cruzada.
        \item Tipos de validación cruzada: hold-out, K-Fold y LOOCV.
        \item Ventajas y desventajas de cada tipo.
    \end{itemize}
    \item \textbf{Implementación en Python:} Los estudiantes accederán a un cuaderno de Jupyter preparado con ejemplos sobre validación cruzada.
    \begin{quote}
        Enlace al cuaderno: \href{https://colab.research.google.com/github/andres-merino/AprendizajeAutomaticoInicial-05-N0105/blob/main/2-Notebooks/18_1-Validacion-Cruzada.ipynb}{18\_1-Validacion-Cruzada.ipynb}.
    \end{quote}
\end{itemize}

\paragraph{Verificación de aprendizaje:}  
\begin{itemize}[leftmargin=*]
    \item ¿Qué ventajas tiene el uso de K-Fold sobre el hold-out?
    \item ¿Cómo afecta el tamaño de \(k\) en la validación cruzada?
    \item ¿Por qué es importante medir la variabilidad en los resultados de validación cruzada?
\end{itemize}

%%%%%%%%%%%%%%%%%%%%%%%%%%%%%%%%%%%%%%%%
\subsection*{Actividad 2: Optimización de hiperparámetros}
%%%%%%%%%%%%%%%%%%%%%%%%%%%%%%%%%%%%%%%%

Esta actividad introduce las técnicas de optimización de hiperparámetros, como Grid Search y Random Search. Los estudiantes aprenderán a configurar búsquedas de hiperparámetros y a evaluar sus resultados.

\paragraph{¿Cómo lo haremos?}  
\begin{itemize}[leftmargin=*]
    \item \textbf{Clase magistral:} Se cubrirán los siguientes temas:
    \begin{itemize}
        \item Diferencia entre parámetros aprendidos e hiperparámetros.
        \item Importancia de la optimización de hiperparámetros.
        \item Grid Search: concepto y configuración.
        \item Random Search: concepto, ventajas y configuración.
        \item Comparación entre ambos métodos: eficiencia y uso.
    \end{itemize}
    \item \textbf{Implementación en Python:} Los estudiantes explorarán un cuaderno de Jupyter con ejemplos prácticos de Grid Search y Random Search.
    \begin{quote}
        Enlace al cuaderno: \href{https://colab.research.google.com/github/andres-merino/AprendizajeAutomaticoInicial-05-N0105/blob/main/2-Notebooks/18_2-Optimizacion-Hiperparametros.ipynb}{18\_2-Optimizacion-Hiperparametros.ipynb}.
    \end{quote}
\end{itemize}

\paragraph{Verificación de aprendizaje:}  
\begin{itemize}[leftmargin=*]
    \item ¿Cuáles son las principales diferencias entre Grid Search y Random Search?
    \item ¿Cómo afecta la validación cruzada en la búsqueda de hiperparámetros?
    \item ¿En qué casos sería más eficiente usar Random Search en lugar de Grid Search?
\end{itemize}

%%%%%%%%%%%%%%%%%%%%%%%%%%%%%%%%%%%%%%%%
\subsection*{Actividad 3: Guardado y lectura de modelos}
%%%%%%%%%%%%%%%%%%%%%%%%%%%%%%%%%%%%%%%%

Esta actividad aborda la importancia de guardar y cargar modelos entrenados para evitar costos computacionales y asegurar reproducibilidad. 

\paragraph{¿Cómo lo haremos?}  
\begin{itemize}[leftmargin=*]
    \item \textbf{Clase magistral:} Se discutirán los siguientes temas:
    \begin{itemize}
        \item Beneficios de guardar modelos: eficiencia y despliegue.
        \item Métodos de serialización: \texttt{pickle} y \texttt{joblib}.
        \item Buenas prácticas al guardar modelos (compatibilidad y versiones de librerías).
        \item Precauciones al cargar modelos en diferentes entornos.
    \end{itemize}
    \item \textbf{Implementación en Python:} Los estudiantes practicarán con dos cuadernos de Jupyter que incluyen ejemplos sobre cómo guardar y cargar modelos entrenados.
    \begin{quote}
        Enlace al cuaderno 1: \href{https://colab.research.google.com/github/andres-merino/AprendizajeAutomaticoInicial-05-N0105/blob/main/2-Notebooks/18_3-Guardado-Modelos.ipynb}{18\_3-Guardado-Modelos.ipynb}.\\
        Enlace al cuaderno 2: \href{https://colab.research.google.com/github/andres-merino/AprendizajeAutomaticoInicial-05-N0105/blob/main/2-Notebooks/18_4-Lectura-Modelos.ipynb}{18\_4-Lectura-Modelos.ipynb}.
    \end{quote}
\end{itemize}

\paragraph{Verificación de aprendizaje:}  
\begin{itemize}[leftmargin=*]
    \item ¿Cuáles son las diferencias entre \texttt{pickle} y \texttt{joblib}?
    \item ¿Qué problemas pueden surgir al cargar un modelo en otro entorno?
    \item ¿Cómo podrías manejar cambios de versiones en las librerías usadas para entrenar un modelo?
\end{itemize}

%%%%%%%%%%%%%%%%%%%%%%%%%%%%%%%%%%%%%%%%
\subsection*{Actividad 4: Esquema de trabajo para proyectos de Aprendizaje Automático}
%%%%%%%%%%%%%%%%%%%%%%%%%%%%%%%%%%%%%%%%


Esta actividad guía a los estudiantes en el diseño de un flujo de trabajo completo para proyectos de Aprendizaje Automático, destacando las mejores prácticas y cada etapa clave del proceso.

\paragraph{¿Cómo lo haremos?}  
\begin{itemize}[leftmargin=*]
    \item \textbf{Clase magistral:} Se cubrirá el flujo de trabajo típico en Aprendizaje Automático, del Cuadro 1.

    \begin{table}[h]
        \centering\small
        \begin{tabular}{p{4.5cm}p{8.5cm}l}
            \toprule
            \textbf{Paso} & \textbf{Detalle} & \textbf{Datos} \\ \midrule
            Limpieza de datos 
            & Manejar valores faltantes, duplicados y atípicos según el contexto del problema. 
            & Completo \\ \midrule
            Análisis exploratorio 
            & Visualiza distribuciones, correlaciones y patrones en los datos para entender su estructura. 
            & Completo \\ \midrule
            Selección de atributos (opcional) 
            & Elige atributos relevantes usando técnicas estadísticas, de importancia de modelos o reducción de dimensionalidad. Puede ubicarse aquí o en otros puntos. 
            & Completo \\ \midrule
            Separar train-test 
            & Divide los datos en entrenamiento y prueba. 
            & Completo \\ \midrule
            Benchmark inicial 
            & Entrena varios modelos usando train y evalúa su desempeño en test. 
            & Train y Test \\ \midrule
            Seleccionar modelos 
            & Elige los mejores modelos según el benchmark inicial. 
            &  \\ \midrule
            Validación cruzada y ajuste de hiperparámetros
            & Realiza búsqueda de hiperparámetros junto con validación cruzada para los modelos seleccionados. 
            & Train \\ \midrule
            Seleccionar mejor modelo 
            & Escoge el modelo más prometedor tras la validación cruzada. 
            &  \\ \midrule
            Evaluar en test 
            & Evalúa el modelo seleccionado en el conjunto de prueba. 
            & Test \\ \midrule
            Reentrenar con todos 
            & Reentrena el modelo final con todos los datos (train + test combinados). 
            & Completo \\ \midrule
            Guardar modelo 
            & Serializa el modelo para despliegue en producción. 
            &  \\ \bottomrule
        \end{tabular}
        \caption{Esquema de trabajo para proyectos de Aprendizaje Automático.}
    \end{table}
\end{itemize}

\paragraph{Verificación de aprendizaje:}  
\begin{itemize}[leftmargin=*]
    \item ¿Qué pasos son imprescindibles en un flujo de trabajo de Aprendizaje Automático?
    \item ¿Cómo garantizas la reproducibilidad en cada etapa del proceso?
    \item ¿Qué herramientas y librerías usarías para automatizar este esquema?
\end{itemize}



%%%%%%%%%%%%%%%%%%%%%%%%%%%%%%%%%%%%%%%%
\section*{Cierre}
%%%%%%%%%%%%%%%%%%%%%%%%%%%%%%%%%%%%%%%%

\paragraph{Tarea:}
    Desarrollar los ejercicios planteados en el siguiente cuaderno y entregarlo por el aula virtual:
    \begin{quote}
        Enlace al cuaderno: \href{https://colab.research.google.com/github/andres-merino/AprendizajeAutomaticoInicial-05-N0105/blob/main/2-Ejercicios/10-Optimizacion-Hiperparametros.ipynb}{10-Optimizacion-Hiperparametros.ipynb}.
    \end{quote}

\paragraph{Pregunta de investigación:}  
\begin{enumerate}[leftmargin=*]
    \item ¿Qué métodos avanzados de optimización de hiperparámetros existen (como Bayesian Optimization) y en qué casos podrían ser útiles?
\end{enumerate}
    
\paragraph{Para la próxima clase:}  
Ya no hay próxima clase :'C


\end{document} 