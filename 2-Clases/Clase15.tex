\documentclass[a4,11pt]{aleph-notas}
% Se puede ver la documentación aquí: 
% https://github.com/alephsub0/LaTeX_aleph-notas

% -- Paquetes adicionales 
\usepackage{enumitem}
\usepackage{url}
\usepackage{array}
\usepackage{booktabs}
\usepackage{longtable}
\usepackage{environ}  % Para definir nuevos entornos
\hypersetup{
    urlcolor=blue,
    linkcolor=blue,
}

% Definición del nuevo ambiente
\NewEnviron{materiales}{%
    \begin{minipage}{12cm}
        \vspace{0.1mm}
        \begin{itemize}[leftmargin=*]
            \BODY % Contenido de la lista
        \end{itemize}
        \vspace{1mm}
    \end{minipage}
}

% -- Datos 
\institucion{Facultad de Ciencias Exactas, Naturales y Ambientales}
\carrera{Ciencia de Datos}
\asignatura{Aprendizaje Automático Inicial}
\tema{Clase 15: Entrenamiento de Redes Neuronales}
\autor{Andrés Merino}
\fecha{Semestre 2025-2}

\logouno[0.14\textwidth]{Logos/logoPUCE_04_ac}
\definecolor{colortext}{HTML}{0030A1}
\definecolor{colordef}{HTML}{0030A1}
\fuente{montserrat}


% -- Comandos para tablas
\usepackage{listings}
\input{listings-python.prf}

\usepackage[spanish,onelanguage,vlined,linesnumbered]{algorithm2e}

% -- Comandos adicionales
\newtcolorbox{pscodigo}
    {icono=\faCogs,color=lightgray,postit,top=-1.5mm,bottom=-1.5mm}
    
\definecolor{colcod}{RGB}{174,218,255}
\newtcolorbox{pycodigo}
    {icono=\faKeyboardO, color=colcod, postit, 
    top=-2mm, bottom=-2mm, 
    extras first={bottom=0mm},
    extras last={top=0mm},
    extras middle={top=0mm,bottom=0mm},
    }

\lstloadlanguages{Python}
\lstset{
  language=Python,
  basicstyle=\small\sffamily,
  stringstyle=\color[HTML]{933797},
  commentstyle=\color[HTML]{228B22}\sffamily,
  emph={[2]from,import,pass,return}, emphstyle={[2]\color[HTML]{DD52F0}},
  emph={[3]range}, emphstyle={[3]\color[HTML]{D17032}},
  emph={[4]for,in,def}, emphstyle={[4]\color{blue}},
  showstringspaces=false,
  breaklines=true,
  prebreak=\mbox{{\color{gray}\tiny$\searrow$}},
  xleftmargin=3pt,
  inputencoding=utf8,
  extendedchars=true,
  columns=fullflexible,
  literate={á}{{\'a}}1 {é}{{\'e}}1 {í}{{\'i}}1 {ó}{{\'o}}1 {ú}{{\'u}}1,
}


\SetKwFunction{concat}{Concatenar}
\SetKwProg{Fn}{Función}{\string:}{}
\SetKwFunction{ult}{Ultimo}
\SetKwFunction{pri}{Primero}
\SetKwFunction{sinul}{SinUltimo}
\SetKw{Salir}{Salir}

\newcommand{\fuentecomentario}[1]{\scriptsize\ttfamily #1}
\SetCommentSty{fuentecomentario}
\SetAlFnt{\footnotesize}


\begin{document}

\encabezado


%%%%%%%%%%%%%%%%%%%%%%%%%%%%%%%%%%%%%%%%
\section*{Resultado de Aprendizaje}
%%%%%%%%%%%%%%%%%%%%%%%%%%%%%%%%%%%%%%%%

%%%%%%%%%%%%%%%%%%%%%%%%%%%%%%%%%%%%%%%%
\subsection*{RdA de la asignatura:}
\begin{itemize}[leftmargin=*]
    % \item \textbf{RdA 1:} Plantear los conceptos fundamentales del aprendizaje automático, incluyendo los principios básicos, técnicas de preprocesado de datos, métodos de evaluación y ajuste de modelos, destacando su importancia en el análisis y resolución de problemas de datos.
    \item \textbf{RdA 2:} 
    Aplicar modelos de aprendizaje automático supervisado y no supervisado, así como su validación y optimización, en la resolución de problemas tanto reales como simulados.
    % \item \textbf{RdA 3:} Resolver problemas prácticos mediante el uso de modelos de aprendizaje automático, ajustándolos para la mejora de su rendimiento y precisión.
\end{itemize}

\subsection*{Resultados específicos:}  
\begin{itemize}[leftmargin=*]  
    \item Comprender el método de retropropagación y su deducción matemática.  
    \item Revisar las técnicas de optimización en el entrenamiento de redes neuronales.  
\end{itemize}  

%%%%%%%%%%%%%%%%%%%%%%%%%%%%%%%%%%%%%%%%  
\section*{Introducción}  
%%%%%%%%%%%%%%%%%%%%%%%%%%%%%%%%%%%%%%%%  

%%%%%%%%%%%%%%%%%%%%%%%%%%%%%%%%%%%%%%%%  
\paragraph{Pregunta inicial:}  
¿Qué desafíos enfrenta el entrenamiento de redes neuronales y cómo podemos mitigarlos para mejorar su desempeño?  

%%%%%%%%%%%%%%%%%%%%%%%%%%%%%%%%%%%%%%%%  
\section*{Desarrollo}  
%%%%%%%%%%%%%%%%%%%%%%%%%%%%%%%%%%%%%%%%  

%%%%%%%%%%%%%%%%%%%%%%%%%%%%%%%%%%%%%%%%  
\subsection*{Actividad 1: Deducción del método de retropropagación}  
%%%%%%%%%%%%%%%%%%%%%%%%%%%%%%%%%%%%%%%%  

\paragraph{¿Cómo lo haremos?}  
\begin{itemize}[leftmargin=*]  
    \item \textbf{Clase magistral:}  
    Explicación detallada del método de retropropagación:
    \begin{itemize}[leftmargin=*]
        \item Introducción a la retropropagación y su propósito en redes neuronales.
        \item Deducción paso a paso del algoritmo, desde la función de pérdida hasta el cálculo de gradientes usando la regla de la cadena.
    \end{itemize}  
\end{itemize}  

\paragraph{Verificación de aprendizaje:}  
\begin{itemize}[leftmargin=*]  
    \item ¿Cómo se utiliza la regla de la cadena en el cálculo de gradientes?  
    \item ¿Qué rol desempeña la función de pérdida en la retropropagación?  
    \item ¿Por qué es fundamental calcular los gradientes de manera eficiente en redes neuronales?  
\end{itemize}  

%%%%%%%%%%%%%%%%%%%%%%%%%%%%%%%%%%%%%%%%  
\subsection*{Actividad 2: Regularización y optimización en redes neuronales}  
%%%%%%%%%%%%%%%%%%%%%%%%%%%%%%%%%%%%%%%%  

\paragraph{¿Cómo lo haremos?}  
\begin{itemize}[leftmargin=*]  
    \item \textbf{Clase magistral:}  
    Presentación de conceptos clave:
    \begin{itemize}[leftmargin=*]
        \item Regularización: L1, L2 y Dropout.
        \item Early stopping como estrategia para evitar el sobreajuste.
        \item Diferentes optimizadores: SGD, Adam, RMSProp, etc.
        \item Learning rate adaptativo y su impacto en el entrenamiento.
    \end{itemize}  
    \item \textbf{Ejercicio práctico:}  
    Implementación guiada en un notebook de Python:
    \begin{itemize}[leftmargin=*]
        \item Configurar y entrenar una red neuronal básica.
        \item Aplicar técnicas de regularización y evaluar resultados.
        \item Comparar el rendimiento utilizando diferentes  optimizadores.
        \item Implementar early stopping y observar su impacto en el sobreajuste.
    \end{itemize}  
\end{itemize}  

\paragraph{Verificación de aprendizaje:}  
\begin{itemize}[leftmargin=*]  
    \item ¿Cómo afecta la regularización al sobreajuste?  
    \item ¿Cuáles son las diferencias principales entre SGD y Adam como optimizadores?  
    \item ¿Qué ventajas ofrece el learning rate adaptativo frente a un learning rate constante?  
\end{itemize}  

%%%%%%%%%%%%%%%%%%%%%%%%%%%%%%%%%%%%%%%%  
\section*{Cierre}  
%%%%%%%%%%%%%%%%%%%%%%%%%%%%%%%%%%%%%%%%  

\paragraph{Tarea:}  
    Desarrollar los ejercicios planteados en el siguiente cuaderno, usando mejoras en el entrenamiento de la red, y entregarlo por el aula virtual:
    \begin{quote}
        Enlace al cuaderno: \href{https://colab.research.google.com/github/andres-merino/AprendizajeAutomaticoInicial-05-N0105/blob/main/2-Ejercicios/07-Perceptron.ipynb}{07-Perceptron.ipynb}.
    \end{quote}

\paragraph{Preguntas para reflexionar:}  
\begin{enumerate}[leftmargin=*]  
    \item ¿Qué técnicas de regularización funcionan mejor en diferentes escenarios?  
    \item ¿Cómo podrías mejorar el desempeño de una red neuronal sin cambiar su arquitectura?  
    \item ¿Cómo afecta el tamaño del dataset al rendimiento del modelo y la necesidad de regularización?  
\end{enumerate}  

\paragraph{Para la próxima clase:}  
Exploración de arquitecturas avanzadas como redes convolucionales y recurrentes. Previsualizar el notebook: \href{https://example.com}{Arquitecturas Avanzadas}.


\end{document} 