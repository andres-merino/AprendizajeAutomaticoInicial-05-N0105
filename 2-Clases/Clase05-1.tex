\documentclass[a4,11pt]{aleph-notas}

% -- Paquetes adicionales 
\usepackage{enumitem}
\usepackage{url}
\usepackage{array}
\usepackage{booktabs}
\hypersetup{
    urlcolor=blue,
    linkcolor=blue,
}


% -- Datos 
\institucion{Facultad de Ciencias Exactas, Naturales y Ambientales}
\carrera{Ciencia de Datos}
\asignatura{Aprendizaje Automático Inicial}
\tema{Clase 05: Evaluación de modelos}
\autor{Andrés Merino}
\fecha{Periodo 2025-2}

\logouno[0.14\textwidth]{Logos/logoPUCE_04_ac}
\definecolor{colortext}{HTML}{0030A1}
\definecolor{colordef}{HTML}{0030A1}
\fuente{montserrat}

% -- Comandos adicionales (comentados si no se usan)
% \usepackage{listings}
\input{listings-python.prf}

\usepackage[spanish,onelanguage,vlined,linesnumbered]{algorithm2e}

% -- Comandos adicionales
\newtcolorbox{pscodigo}
    {icono=\faCogs,color=lightgray,postit,top=-1.5mm,bottom=-1.5mm}
    
\definecolor{colcod}{RGB}{174,218,255}
\newtcolorbox{pycodigo}
    {icono=\faKeyboardO, color=colcod, postit, 
    top=-2mm, bottom=-2mm, 
    extras first={bottom=0mm},
    extras last={top=0mm},
    extras middle={top=0mm,bottom=0mm},
    }

\lstloadlanguages{Python}
\lstset{
  language=Python,
  basicstyle=\small\sffamily,
  stringstyle=\color[HTML]{933797},
  commentstyle=\color[HTML]{228B22}\sffamily,
  emph={[2]from,import,pass,return}, emphstyle={[2]\color[HTML]{DD52F0}},
  emph={[3]range}, emphstyle={[3]\color[HTML]{D17032}},
  emph={[4]for,in,def}, emphstyle={[4]\color{blue}},
  showstringspaces=false,
  breaklines=true,
  prebreak=\mbox{{\color{gray}\tiny$\searrow$}},
  xleftmargin=3pt,
  inputencoding=utf8,
  extendedchars=true,
  columns=fullflexible,
  literate={á}{{\'a}}1 {é}{{\'e}}1 {í}{{\'i}}1 {ó}{{\'o}}1 {ú}{{\'u}}1,
}


\SetKwFunction{concat}{Concatenar}
\SetKwProg{Fn}{Función}{\string:}{}
\SetKwFunction{ult}{Ultimo}
\SetKwFunction{pri}{Primero}
\SetKwFunction{sinul}{SinUltimo}
\SetKw{Salir}{Salir}

\newcommand{\fuentecomentario}[1]{\scriptsize\ttfamily #1}
\SetCommentSty{fuentecomentario}
\SetAlFnt{\footnotesize}


\begin{document}

\encabezado
% En todo el documento, las indicaciones deben ser simples y directas, con una sola oración.

%%%%%%%%%%%%%%%%%%%%%%%%%%%%%%%%%%%%%%%%
\section*{Resultado de Aprendizaje}
%%%%%%%%%%%%%%%%%%%%%%%%%%%%%%%%%%%%%%%%

%%%%%%%%%%%%%%%%%%%%%%%%%%%%%%%%%%%%%%%%
\subsection*{RdA de la asignatura:}
% Se toma uno de los siguientes
\begin{itemize}[leftmargin=*]
    \item \textbf{RdA 1:} Plantear los conceptos fundamentales del aprendizaje automático, incluyendo los principios básicos, técnicas de preprocesado de datos, métodos de evaluación y ajuste de modelos, destacando su importancia en el análisis y resolución de problemas de datos.
\end{itemize}

%%%%%%%%%%%%%%%%%%%%%%%%%%%%%%%%%%%%%%%%
\subsection*{RdA de la clase:}
% Máximo 3 resultados
\begin{itemize}[leftmargin=*]
    \item Comprender la importancia de la evaluación de modelos y su impacto en la toma de decisiones.
    \item Explicar y calcular las métricas para modelos de clasificación (matriz de confusión, ACC, PRE, REC, F1, Curva ROC).
    \item Interpretar y calcular las métricas para modelos de regresión (MAE, MSE, RMSE).
\end{itemize}

%%%%%%%%%%%%%%%%%%%%%%%%%%%%%%%%%%%%%%%%
\section*{Introducción}
%%%%%%%%%%%%%%%%%%%%%%%%%%%%%%%%%%%%%%%%

%%%%%%%%%%%%%%%%%%%%%%%%%%%%%%%%%%%%%%%%
\paragraph{Pregunta inicial:} 
¿Cómo sabemos si un modelo de aprendizaje automático está funcionando bien y es confiable en un contexto práctico?

%%%%%%%%%%%%%%%%%%%%%%%%%%%%%%%%%%%%%%%%
\section*{Desarrollo}
%%%%%%%%%%%%%%%%%%%%%%%%%%%%%%%%%%%%%%%%

%%%%%%%%%%%%%%%%%%%%%%%%%%%%%%%%%%%%%%%%
\subsection*{Actividad 1: Explorando la Evaluación de Modelos de Clasificación}

Esta actividad combina clase magistral, recursos multimedia y ejercicio práctico en equipos mediante cuaderno de Jupyter para comprender las métricas derivadas de la matriz de confusión y analizar casos donde cada métrica sea más relevante.

\paragraph{¿Cómo lo haremos?}  
\begin{itemize}[leftmargin=*]
    \item \textbf{Clase magistral:}  
    Se explicará la matriz de confusión y cómo calcular las métricas derivadas (Precisión, Sensibilidad, F1-Score, Accuracy), presentando un cuadro resumen para facilitar el aprendizaje.
    
    \item \textbf{Video explicativo:}  
    Los estudiantes verán un video que explica en detalle los conceptos de Precisión (Precision) y Sensibilidad (Recall):
    \begin{quote}
        \url{https://www.youtube.com/watch?v=JXvHsw1WzF4&t=116s}
    \end{quote}
    
    \item \textbf{Discusión guiada con ChatGPT:}  
    Se plantearán interactuar con ChatGPT mediante el siguiente prompt:
    \begin{quote}\small
        Quiero ejemplos prácticos en los que haya que tomar una decisión entre priorizar el Recall (sensibilidad) o la Precisión en un modelo de clasificación. Genera dos casos diferentes, asegurándote de incluir una breve descripción del contexto, las consecuencias de maximizar cada métrica y la razón por la cual una de ellas sería más importante en ese caso. Los casos deben ser variados y abarcar aplicaciones como medicina, finanzas, seguridad, marketing y tecnología.
    \end{quote}
    Luego de leer los casos, utiliza este prompt:
    \begin{quote}\small
        Vas a ser mi tutor de Aprendizaje Supervisado. Quiero evaluar mi capacidad para decidir cuándo es mejor priorizar el Recall o la Precisión en un modelo de clasificación. Plantea un caso práctico en el que deba tomar esta decisión. Debe incluir un contexto breve, una descripción del problema y una instrucción clara que me pida tomar una decisión entre Recall o Precisión con una breve justificación. Luego pídeme que responde y, con base en mi respuesta, dame retroalimentación o un razonamiento sobre cuál métrica sería mejor priorizar y por qué. Posterior a eso, plantéame otro caso.
    \end{quote}
    
    \item \textbf{Implementación en Python:} Los estudiantes accederán a un cuaderno de Jupyter previamente preparado.
    \begin{quote}
        Enlace al cuaderno: \href{https://github.com/andres-merino/AprendizajeAutomaticoInicial-05-N0105/blob/main/2-Notebooks/05-1-Evaluacion-de-Modelos-Clasificacion.ipynb}{05-1-Evaluacion-de-Modelos-Clasificacion.ipynb}.
    \end{quote}
    
    \item \textbf{Experimentación:} 
    
\end{itemize}

%%%%%%%%%%%%%%%%%%%%%%%%%%%%%%%%%%%%%%%%
\subsection*{Actividad 2: Evaluación de Modelos de Regresión}

Esta actividad combina clase magistral y ejercicio práctico mediante cuaderno de Jupyter para aprender a calcular métricas para modelos de regresión, incluyendo MAE, MSE y RMSE.

\paragraph{¿Cómo lo haremos?}  
\begin{itemize}[leftmargin=*]
    \item \textbf{Clase magistral:}  
    Se explicarán los conceptos clave de las métricas de evaluación para modelos de regresión:
    \begin{itemize}
        \item MAE (Mean Absolute Error): Error absoluto medio.
        \item MSE (Mean Squared Error): Error cuadrático medio.
        \item RMSE (Root Mean Squared Error): Raíz cuadrada del MSE.
    \end{itemize}
    
    \item \textbf{Implementación en Python:} Los estudiantes accederán a un cuaderno de Jupyter previamente preparado.
    \begin{quote}
        Enlace al cuaderno: \href{https://github.com/andres-merino/AprendizajeAutomaticoInicial-05-N0105/blob/main/2-Notebooks/05-1-Evaluacion-de-Modelos-Regresion.ipynb}{05-1-Evaluacion-de-Modelos-Regresion.ipynb}.
    \end{quote}

    \item \textbf{Experimentación:} 
    
\end{itemize}

%%%%%%%%%%%%%%%%%%%%%%%%%%%%%%%%%%%%%%%%
\section*{Cierre}
%%%%%%%%%%%%%%%%%%%%%%%%%%%%%%%%%%%%%%%%

\paragraph{Verificación de aprendizaje:} 
\begin{enumerate}[leftmargin=*]
    \item ¿Qué información proporciona la matriz de confusión y cuáles son sus cuatro componentes principales?
    % Respuesta: La matriz de confusión muestra el rendimiento de un modelo de clasificación comparando las predicciones con los valores reales. Sus cuatro componentes son: Verdaderos Positivos (TP), Falsos Positivos (FP), Verdaderos Negativos (TN) y Falsos Negativos (FN).
    
    \item ¿Cuál es la diferencia entre Precisión (Precision) y Sensibilidad (Recall)?
    % Respuesta: Precision mide qué proporción de las predicciones positivas fueron correctas (TP/(TP+FP)), mientras que Recall mide qué proporción de los casos positivos reales fueron identificados correctamente (TP/(TP+FN)).
    
    \item ¿Por qué el MSE penaliza más los errores grandes que el MAE en modelos de regresión?
    % Respuesta: El MSE eleva al cuadrado los errores, lo que amplifica los errores grandes. Por ejemplo, un error de 10 contribuye 100 al MSE pero solo 10 al MAE, haciendo que el MSE sea más sensible a valores atípicos.
\end{enumerate}

\paragraph{Preguntas tipo entrevista:} 
\begin{enumerate}[leftmargin=*]
    \item En un sistema de detección de fraudes bancarios, ¿priorizarías Precision o Recall? 
    
    \item Si un modelo tiene un Accuracy del 95\%, ¿significa que es un buen modelo para cualquier problema de clasificación?
\end{enumerate}

% \paragraph{Tarea:} Elegir un conjunto de datos de clasificación o regresión y aplicar las métricas aprendidas para evaluar el rendimiento de un modelo.

\paragraph{Pregunta de investigación:} 
\begin{enumerate}[leftmargin=*]
    \item ¿Existen más métricas de evaluación de modelos de clasificación? Ver: Gráfico K-S, Índice Jaccard, Gráfico de ganancia y elevación
    \item ¿Qué métricas existen para evaluar modelos de clasificación para múltiples clases?
\end{enumerate}
    
\paragraph{Para la próxima clase:} Realizar la Clase invertida: Métricas de modelos de agrupamiento, disponible en el aula virtual.


\end{document}