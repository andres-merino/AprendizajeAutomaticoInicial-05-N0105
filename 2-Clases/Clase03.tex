\documentclass[a4,11pt]{aleph-notas}

% -- Paquetes adicionales 
\usepackage{enumitem}
\usepackage{url}
\usepackage{array}
\usepackage{booktabs}
\hypersetup{
    urlcolor=blue,
    linkcolor=blue,
}

% -- Datos 
\institucion{Facultad de Ciencias Exactas, Naturales y Ambientales}
\carrera{Ciencia de Datos}
\asignatura{Aprendizaje Automático Inicial}
\tema{Clase 03: Conjuntos de Entrenamiento y Validación}
\autor{Andrés Merino}
\fecha{Periodo 2025-2}

\logouno[0.14\textwidth]{Logos/logoPUCE_04_ac}
\definecolor{colortext}{HTML}{0030A1}
\definecolor{colordef}{HTML}{0030A1}
\fuente{montserrat}

% -- Comandos adicionales (comentados si no se usan)
% \usepackage{listings}
\input{listings-python.prf}

\usepackage[spanish,onelanguage,vlined,linesnumbered]{algorithm2e}

% -- Comandos adicionales
\newtcolorbox{pscodigo}
    {icono=\faCogs,color=lightgray,postit,top=-1.5mm,bottom=-1.5mm}
    
\definecolor{colcod}{RGB}{174,218,255}
\newtcolorbox{pycodigo}
    {icono=\faKeyboardO, color=colcod, postit, 
    top=-2mm, bottom=-2mm, 
    extras first={bottom=0mm},
    extras last={top=0mm},
    extras middle={top=0mm,bottom=0mm},
    }

\lstloadlanguages{Python}
\lstset{
  language=Python,
  basicstyle=\small\sffamily,
  stringstyle=\color[HTML]{933797},
  commentstyle=\color[HTML]{228B22}\sffamily,
  emph={[2]from,import,pass,return}, emphstyle={[2]\color[HTML]{DD52F0}},
  emph={[3]range}, emphstyle={[3]\color[HTML]{D17032}},
  emph={[4]for,in,def}, emphstyle={[4]\color{blue}},
  showstringspaces=false,
  breaklines=true,
  prebreak=\mbox{{\color{gray}\tiny$\searrow$}},
  xleftmargin=3pt,
  inputencoding=utf8,
  extendedchars=true,
  columns=fullflexible,
  literate={á}{{\'a}}1 {é}{{\'e}}1 {í}{{\'i}}1 {ó}{{\'o}}1 {ú}{{\'u}}1,
}


\SetKwFunction{concat}{Concatenar}
\SetKwProg{Fn}{Función}{\string:}{}
\SetKwFunction{ult}{Ultimo}
\SetKwFunction{pri}{Primero}
\SetKwFunction{sinul}{SinUltimo}
\SetKw{Salir}{Salir}

\newcommand{\fuentecomentario}[1]{\scriptsize\ttfamily #1}
\SetCommentSty{fuentecomentario}
\SetAlFnt{\footnotesize}


\begin{document}

\encabezado

%%%%%%%%%%%%%%%%%%%%%%%%%%%%%%%%%%%%%%%%
\section*{Actividad previa}
%%%%%%%%%%%%%%%%%%%%%%%%%%%%%%%%%%%%%%%%

\begin{itemize}
    \item Revisar conjuntos de datos seleccionados por los estudiantes y brindar retroalimentación.
    \item Explicación de la distancia de Mahalanobis.
\end{itemize}

%%%%%%%%%%%%%%%%%%%%%%%%%%%%%%%%%%%%%%%%
\section*{Resultado de Aprendizaje}
%%%%%%%%%%%%%%%%%%%%%%%%%%%%%%%%%%%%%%%%

%%%%%%%%%%%%%%%%%%%%%%%%%%%%%%%%%%%%%%%%
\subsection*{RdA de la asignatura:}
% Se toma uno de los siguientes
\begin{itemize}[leftmargin=*]
    \item \textbf{RdA 1:} Plantear los conceptos fundamentales del aprendizaje automático, incluyendo los principios básicos, técnicas de preprocesado de datos, métodos de evaluación y ajuste de modelos, destacando su importancia en el análisis y resolución de problemas de datos.
\end{itemize}

%%%%%%%%%%%%%%%%%%%%%%%%%%%%%%%%%%%%%%%%
\subsection*{RdA de la clase:}
% Máximo 3 resultados
\begin{itemize}[leftmargin=*]
    \item Comprender las técnicas principales de preprocesado de datos: escalado, discretización y reducción de dimensionalidad.
    \item Aplicar división de conjuntos de datos en entrenamiento y prueba para evaluar modelos de aprendizaje automático.
\end{itemize}

%%%%%%%%%%%%%%%%%%%%%%%%%%%%%%%%%%%%%%%%
\section*{Introducción}
%%%%%%%%%%%%%%%%%%%%%%%%%%%%%%%%%%%%%%%%

%%%%%%%%%%%%%%%%%%%%%%%%%%%%%%%%%%%%%%%%
\paragraph{Pregunta inicial:} 
\begin{itemize}
    \item 
    Si en un modelo una de las varaibles, por ejemplo, la edad, tiene un rango de valores entre 0 y 100, y otra variable, por ejemplo, el ingreso anuel, tiene un rango entre \$0 y \$100000, ¿qué problemas podrían surgir al calcular distancias entre puntos en este espacio de características?
    \item 
    ¿Si en un examen de matemática te evalúan con exactamente los mismos ejercicios que practicaste, qué tan bien crees que mediría tu conocimiento real de la materia?
\end{itemize}
    
%%%%%%%%%%%%%%%%%%%%%%%%%%%%%%%%%%%%%%%%
\section*{Desarrollo}
%%%%%%%%%%%%%%%%%%%%%%%%%%%%%%%%%%%%%%%%

%%%%%%%%%%%%%%%%%%%%%%%%%%%%%%%%%%%%%%%%
\subsection*{Actividad 1: Preprocesado de Datos}

En esta actividad los estudiantes aprenderán las técnicas básicas de preprocesado de datos mediante clase magistral y práctica en cuaderno de Jupyter, incluyendo escalado, discretización y reducción de dimensionalidad para preparar datos de aprendizaje automático.

\paragraph{¿Cómo lo haremos?}  
\begin{itemize}[leftmargin=*]
    \item \textbf{Conceptos fundamentales:}  
    Se presentarán las técnicas de preprocesado: escalado, discretización, one-hot-encoding y reducción de dimensionalidad.
    \item \textbf{Implementación en Python:} Los estudiantes accederán a un cuaderno de Jupyter previamente preparado.
    \begin{quote}
        Enlace al cuaderno: \href{https://github.com/andres-merino/AprendizajeAutomaticoInicial-05-N0105/blob/main/2-Notebooks/03-1-Procesado-Datos.ipynb}{03-1-Preprocesado-Datos.ipynb}.
    \end{quote}
    \item \textbf{Experimentación:} Realizar los ejercicios propuestos en el cuaderno.
    
\end{itemize}

%%%%%%%%%%%%%%%%%%%%%%%%%%%%%%%%%%%%%%%%
\subsection*{Actividad 2: Conjuntos de Entrenamiento y Test}

En esta actividad se abordará la importancia de dividir un conjunto de datos en entrenamiento y prueba mediante exploración de cuaderno de Jupyter.

\paragraph{¿Cómo lo haremos?}  
\begin{itemize}[leftmargin=*]
    \item \textbf{Métodos de partición:}  
    Se presentarán las necesidad de particionar conjuntos de datos.
    \item \textbf{Implementación en Python:} Los estudiantes accederán a un cuaderno de Jupyter previamente preparado.
    \begin{quote}
        Enlace al cuaderno: \href{https://github.com/andres-merino/AprendizajeAutomaticoInicial-05-N0105/blob/main/2-Notebooks/03-2-Conjuntos-Entrenamiento-Prueba.ipynb}{03-2-Conjuntos-Entrenamiento-Prueba.ipynb}.
    \end{quote}
    \item \textbf{Experimentación:} Realizar los ejercicios propuestos en el cuaderno.
    
\end{itemize}

%%%%%%%%%%%%%%%%%%%%%%%%%%%%%%%%%%%%%%%%
\section*{Cierre}
%%%%%%%%%%%%%%%%%%%%%%%%%%%%%%%%%%%%%%%%

\paragraph{Verificación de aprendizaje:} 
\begin{enumerate}[leftmargin=*]
    \item ¿En qué rangos queda una variable luego de aplicar cada tipo de normalización?
    
    \item ¿Cuál es la diferencia entre discretizar una variable usando igual amplitud o igual frecuencia?
    
    \item ¿Dónde ingresan el concepto de valores propios en el ACP?
\end{enumerate}

\paragraph{Preguntas tipo entrevista:} 
\begin{enumerate}[leftmargin=*]
    \item Describe la diferencia entre discretización y one-hot encoding.
    
    \item ¿Qué método de normalización utilizarías para variables categóricas antes de entrenar un modelo?
\end{enumerate}

\paragraph{Tarea:} 
    Realizar el procesamiento de datos necesarios al conjunto de datos seleccionado en la clase anterior y subirlo al repositorio de GitHub.

\paragraph{Pregunta de investigación:} 
\begin{enumerate}[leftmargin=*]
    \item ¿Qué estrategias existen para manejar conjuntos de datos desbalanceados al realizar particiones de entrenamiento y prueba?
    \item ¿Realizar one-hot encoding agrega colinealidad a las caracterísitcas? Ver: \href{https://blog.dailydoseofds.com/p/the-most-overlooked-problem-with}{The Problem With One-Hot Encoding}
\end{enumerate}
    
\paragraph{Para la próxima clase:} 
    Se debe tener el conjunto de datos preprocesados y listos en el repositorio.


\end{document}