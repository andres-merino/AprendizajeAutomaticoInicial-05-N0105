\documentclass[a4,11pt]{aleph-notas}
% Se puede ver la documentación aquí: 
% https://github.com/alephsub0/LaTeX_aleph-notas

% -- Paquetes adicionales 
\usepackage{enumitem}
\usepackage{url}
\usepackage{array}
\usepackage{booktabs}
\usepackage{longtable}
\usepackage{environ}  % Para definir nuevos entornos
\hypersetup{
    urlcolor=blue,
    linkcolor=blue,
}

% Definición del nuevo ambiente
\NewEnviron{materiales}{%
    \begin{minipage}{12cm}
        \vspace{0.1mm}
        \begin{itemize}[leftmargin=*]
            \BODY % Contenido de la lista
        \end{itemize}
        \vspace{1mm}
    \end{minipage}
}


% -- Datos 
\institucion{Facultad de Ciencias Exactas, Naturales y Ambientales}
\carrera{Ciencia de Datos}
\asignatura{Aprendizaje Automático Inicial}
\tema{Clase 05: Evaluación de modelos}
\autor{Andrés Merino}
\fecha{Periodo 2025-2}

\logouno[0.14\textwidth]{Logos/logoPUCE_04_ac}
\definecolor{colortext}{HTML}{0030A1}
\definecolor{colordef}{HTML}{0030A1}
\fuente{montserrat}


% -- Comandos para tablas
\usepackage{listings}
\input{listings-python.prf}

\usepackage[spanish,onelanguage,vlined,linesnumbered]{algorithm2e}

% -- Comandos adicionales
\newtcolorbox{pscodigo}
    {icono=\faCogs,color=lightgray,postit,top=-1.5mm,bottom=-1.5mm}
    
\definecolor{colcod}{RGB}{174,218,255}
\newtcolorbox{pycodigo}
    {icono=\faKeyboardO, color=colcod, postit, 
    top=-2mm, bottom=-2mm, 
    extras first={bottom=0mm},
    extras last={top=0mm},
    extras middle={top=0mm,bottom=0mm},
    }

\lstloadlanguages{Python}
\lstset{
  language=Python,
  basicstyle=\small\sffamily,
  stringstyle=\color[HTML]{933797},
  commentstyle=\color[HTML]{228B22}\sffamily,
  emph={[2]from,import,pass,return}, emphstyle={[2]\color[HTML]{DD52F0}},
  emph={[3]range}, emphstyle={[3]\color[HTML]{D17032}},
  emph={[4]for,in,def}, emphstyle={[4]\color{blue}},
  showstringspaces=false,
  breaklines=true,
  prebreak=\mbox{{\color{gray}\tiny$\searrow$}},
  xleftmargin=3pt,
  inputencoding=utf8,
  extendedchars=true,
  columns=fullflexible,
  literate={á}{{\'a}}1 {é}{{\'e}}1 {í}{{\'i}}1 {ó}{{\'o}}1 {ú}{{\'u}}1,
}


\SetKwFunction{concat}{Concatenar}
\SetKwProg{Fn}{Función}{\string:}{}
\SetKwFunction{ult}{Ultimo}
\SetKwFunction{pri}{Primero}
\SetKwFunction{sinul}{SinUltimo}
\SetKw{Salir}{Salir}

\newcommand{\fuentecomentario}[1]{\scriptsize\ttfamily #1}
\SetCommentSty{fuentecomentario}
\SetAlFnt{\footnotesize}


\begin{document}

\encabezado


\section*{Resultado de Aprendizaje}

\subsection*{RdA de la asignatura:}
\begin{itemize}[leftmargin=*]
    \item \textbf{RdA 1:} 
    Plantear los conceptos fundamentales del aprendizaje automático, incluyendo los principios básicos, técnicas de preprocesado de datos, métodos de evaluación y ajuste de modelos, destacando su importancia en el análisis y resolución de problemas de datos.
\end{itemize}

\subsection*{RdA de la clase:}
    \begin{itemize}[leftmargin=*]
        \item Comprender la importancia de la evaluación de modelos y su impacto en la toma de decisiones.
        \item Explicar y calcular las métricas para modelos de clasificación (matriz de confusión, ACC, PRE, REC, F1, Curva ROC).
        \item Interpretar y calcular las métricas para modelos de regresión (MAE, MSE, RMSE).
    \end{itemize}

\section*{Introducción}

\paragraph{Pregunta inicial:} 
¿Cómo sabemos si un modelo de aprendizaje automático está funcionando bien y es confiable en un contexto práctico?


\section*{Desarrollo}

\subsection*{Actividad 1: Explorando la Evaluación de Modelos de Clasificación (70 minutos)}

Esta actividad combina una clase magistral, recursos multimedia y un ejercicio práctico en equipos para comprender las métricas derivadas de la matriz de confusión. Los estudiantes explorarán las métricas (Precisión, Sensibilidad, F1-Score, Accuracy) y analizarán casos donde cada métrica sea más relevante.

\paragraph{¿Cómo lo haremos?}  
\begin{itemize}[leftmargin=*]
    \item \textbf{Clase magistral:}  
    Se explicará la matriz de confusión y cómo calcular las métricas derivadas. Para facilitar el aprendizaje, se presentará un cuadro resumen.

    \item \textbf{Video explicativo:}  
    Los estudiantes verán un video que explica en detalle los conceptos de Precisión (Precision) y Sensibilidad (Recall): \url{https://www.youtube.com/watch?v=JXvHsw1WzF4&t=116s}

    \item \textbf{Discusión guiada:}  
    Se plantearán interactuar con ChatGPT mediante el siguiente prompt:
    \begin{quote}\small
        Quiero ejemplos prácticos en los que haya que tomar una decisión entre priorizar el Recall (sensibilidad) o la Precisión en un modelo de clasificación. Genera dos casos diferentes, asegurándote de incluir una breve descripción del contexto, las consecuencias de maximizar cada métrica y la razón por la cual una de ellas sería más importante en ese caso. Los casos deben ser variados y abarcar aplicaciones como medicina, finanzas, seguridad, marketing y tecnología.
    \end{quote}
    Luego de leer los casos, utiliza este prompt:
    \begin{quote}\small
        Vas a ser mi tutor de Aprendizaje Supervisado. Quiero evaluar mi capacidad para decidir cuándo es mejor priorizar el Recall o la Precisión en un modelo de clasificación. Plantea un caso práctico en el que deba tomar esta decisión. Debe incluir un contexto breve, una descripción del problema y una instrucción clara que me pida tomar una decisión entre Recall o Precisión con una breve justificación. Luego pídeme que responde y, con base en mi respuesta, dame retroalimentación o un razonamiento sobre cuál métrica sería mejor priorizar y por qué. Posterior a eso, plantéame otro caso.
    \end{quote}

    \item \textbf{Ejercicio práctico guiado :}  
    Se mostrará el siguiente código base para calcular métricas con datos simulados (\texttt{y\_true} y \texttt{y\_pred}):

\begin{pycodigo}
\begin{lstlisting}[language=Python]
from sklearn.metrics import confusion_matrix, accuracy_score, precision_score, recall_score, f1_score
import numpy as np

# Datos simulados
np.random.seed(42)
y_true = np.random.choice([0, 1], size=100, p=[0.5, 0.5])
y_pred = np.random.choice([0, 1], size=100, p=[0.5, 0.5])

# Calcular la matriz de confusión y métricas
cm = confusion_matrix(y_true, y_pred)
accuracy = accuracy_score(y_true, y_pred)
precision = precision_score(y_true, y_pred)
recall = recall_score(y_true, y_pred)
f1 = f1_score(y_true, y_pred)

print("Matriz de Confusión:\n", cm)
print(f"Accuracy: {accuracy:.2f}")
print(f"Precision: {precision:.2f}")
print(f"Recall: {recall:.2f}")
print(f"F1-Score: {f1:.2f}")
\end{lstlisting}
\end{pycodigo}

    \item \textbf{Exploración por equipos:}  
    Los estudiantes, organizados en parejas o tríos, realizarán las siguientes actividades:
    \begin{enumerate}[leftmargin=*]
        \item Modificar el código para generar nuevos datos (\texttt{y\_true} y \texttt{y\_pred}) y calcular las métricas.
        \item Responder las siguientes preguntas:
            \begin{itemize}[leftmargin=*]
                \item ¿Cómo cambia la Precisión y el Recall al modificar la distribución de \texttt{y\_true} y \texttt{y\_pred}?
                \item ¿Qué sucede con el F1-Score si una de las métricas (Precisión o Recall) es muy baja?
            \end{itemize}
        \end{enumerate}
    \end{itemize}

\paragraph{Verificación de aprendizaje:}  
    Mediante los casos presentados por el prompt proporcionado.

\subsection*{Actividad 2: Evaluación de Modelos de Regresión (40 minutos)}

Esta actividad combina una clase magistral y un ejercicio práctico para aprender a calcular para modelos de regresión, incluyendo MAE, MSE y RMSE (Raíz del Error Cuadrático Medio).

\paragraph{¿Cómo lo haremos?}  

\begin{itemize}[leftmargin=*]
    \item \textbf{Clase magistral:}  
    Se explicarán los conceptos clave de las métricas de evaluación para modelos de regresión: MAE (Mean Absolute Error), MSE (Mean Squared Error), RMSE (Raíz cuadrada del MSE).

    \item \textbf{Ejercicio práctico (25 minutos):}  
    Los estudiantes implementarán un modelo de regresión lineal y calcularán las métricas utilizando datos simulados. Se les proporcionará el siguiente código base:

\begin{pycodigo}
\begin{lstlisting}[language=Python]
import numpy as np
from sklearn.metrics import mean_absolute_error, mean_squared_error
from sklearn.linear_model import LinearRegression
from sklearn.model_selection import train_test_split

# Datos simulados
np.random.seed(42)
X = np.random.rand(100, 1) * 10  # Característica
y = 3 * X.squeeze() + np.random.randn(100) * 5  # Variable objetivo

# División de datos en entrenamiento y prueba
X_train, X_test, y_train, y_test = train_test_split(X, y, test_size=0.2, random_state=42)

# Entrenar modelo de regresión lineal
model = LinearRegression()
model.fit(X_train, y_train)

# Predicciones
y_pred = model.predict(X_test)

# Cálculo de métricas
mae = mean_absolute_error(y_test, y_pred)
mse = mean_squared_error(y_test, y_pred)
rmse = np.sqrt(mse)

print(f"MAE: {mae:.2f}")
print(f"MSE: {mse:.2f}")
print(f"RMSE: {rmse:.2f}")
\end{lstlisting}
\end{pycodigo}

    Los estudiantes deberán:
    \begin{enumerate}
        \item Interpretar los valores de MAE, MSE y RMSE obtenidos.
        \item Comparar el impacto de aumentar o disminuir los errores del modelo en las métricas calculadas.
        \item Modificar el modelo o los datos para observar cómo cambian las métricas al ajustar el nivel de ruido en los datos.
    \end{enumerate}
\end{itemize}

\paragraph{Verificación de aprendizaje:}  
Al finalizar, los estudiantes responderán las siguientes preguntas:
\begin{itemize}
    \item ¿Qué métrica prefieren para evaluar el modelo y por qué?
    \item Si el modelo se ajusta a un problema donde los grandes errores son inaceptables, ¿qué métrica priorizarían? Justifique su elección.
    \item ¿Cómo interpretan los resultados obtenidos y qué ajustes podrían realizar para mejorar el rendimiento del modelo?
\end{itemize}


\section*{Cierre}

\paragraph{Tarea:} Elegir un conjunto de datos de clasificación o regresión y aplicar las métricas aprendidas para evaluar el rendimiento de un modelo.
    
    
\paragraph{Pregunta de investigación:} 
\begin{enumerate}
    \item ¿Es posible diseñar una métrica de evaluación personalizada que combine las ventajas de Precision, Recall y F1-score para un problema específico?
    \item ¿Las normalizaciones de los datos podrían sesgar las métricas hacia resultados que no reflejan el verdadero rendimiento del modelo?
    \item ¿Qué métricas existen para evaluar modelos de clasificación para múltiples clases?

\end{enumerate}
    


\end{document} 