\documentclass[a4,11pt]{aleph-notas}
% Se puede ver la documentación aquí: 
% https://github.com/alephsub0/LaTeX_aleph-notas

% -- Paquetes adicionales 
\usepackage{enumitem}
\usepackage{url}
\usepackage{array}
\usepackage{booktabs}
\usepackage{longtable}
\usepackage{environ}  % Para definir nuevos entornos
\hypersetup{
    urlcolor=blue,
    linkcolor=blue,
}

% Definición del nuevo ambiente
\NewEnviron{materiales}{%
    \begin{minipage}{12cm}
        \vspace{0.1mm}
        \begin{itemize}[leftmargin=*]
            \BODY % Contenido de la lista
        \end{itemize}
        \vspace{1mm}
    \end{minipage}
}

% -- Datos
\institucion{Escuela de Ciencias Físicas y Matemática}
\carrera{Ciencia de Datos}
\asignatura{Aprendizaje Automático Inicial}
\tema{Clase 02: Conceptos básicos del Aprendizaje Automático}
\autor{Andrés Merino}
\fecha{Semestre 2025-1}

\logouno[0.14\textwidth]{Logos/logoPUCE_04_ac}
\definecolor{colortext}{HTML}{0030A1}
\definecolor{colordef}{HTML}{0030A1}
\fuente{montserrat}


% -- Comandos para tablas
\usepackage{listings}
\input{listings-python.prf}

\usepackage[spanish,onelanguage,vlined,linesnumbered]{algorithm2e}

% -- Comandos adicionales
\newtcolorbox{pscodigo}
    {icono=\faCogs,color=lightgray,postit,top=-1.5mm,bottom=-1.5mm}
    
\definecolor{colcod}{RGB}{174,218,255}
\newtcolorbox{pycodigo}
    {icono=\faKeyboardO, color=colcod, postit, 
    top=-2mm, bottom=-2mm, 
    extras first={bottom=0mm},
    extras last={top=0mm},
    extras middle={top=0mm,bottom=0mm},
    }

\lstloadlanguages{Python}
\lstset{
  language=Python,
  basicstyle=\small\sffamily,
  stringstyle=\color[HTML]{933797},
  commentstyle=\color[HTML]{228B22}\sffamily,
  emph={[2]from,import,pass,return}, emphstyle={[2]\color[HTML]{DD52F0}},
  emph={[3]range}, emphstyle={[3]\color[HTML]{D17032}},
  emph={[4]for,in,def}, emphstyle={[4]\color{blue}},
  showstringspaces=false,
  breaklines=true,
  prebreak=\mbox{{\color{gray}\tiny$\searrow$}},
  xleftmargin=3pt,
  inputencoding=utf8,
  extendedchars=true,
  columns=fullflexible,
  literate={á}{{\'a}}1 {é}{{\'e}}1 {í}{{\'i}}1 {ó}{{\'o}}1 {ú}{{\'u}}1,
}


\SetKwFunction{concat}{Concatenar}
\SetKwProg{Fn}{Función}{\string:}{}
\SetKwFunction{ult}{Ultimo}
\SetKwFunction{pri}{Primero}
\SetKwFunction{sinul}{SinUltimo}
\SetKw{Salir}{Salir}

\newcommand{\fuentecomentario}[1]{\scriptsize\ttfamily #1}
\SetCommentSty{fuentecomentario}
\SetAlFnt{\footnotesize}


\begin{document}

\encabezado

\section*{Resultado de Aprendizaje}

\subsection*{RdA de la asignatura:}
\begin{itemize}[leftmargin=*]
    \item \textbf{RdA 1:} 
    Plantear los conceptos fundamentales del aprendizaje automático, incluyendo los principios básicos, técnicas de preprocesado de datos, métodos de evaluación y ajuste de modelos, destacando su importancia en el análisis y resolución de problemas de datos.
\end{itemize}

\subsection*{RdA de la actividad:}
\begin{enumerate}[leftmargin=*]
    \item Definir qué es aprendizaje supervisado y no supervisado.
    \item Identificar los tipos de tareas asociadas (clasificación, regresión y agrupamiento).
    \item Enumerar los principales modelos utilizados en cada tipo de tarea.
    \item Comprender las métricas comunes utilizadas para medir distancias.
\end{enumerate}


\section*{Introducción}

\paragraph{Pregunta inicial:}  
Si tuvieras un conjunto de datos de imágenes de flores, ¿qué datos se necesitaría para construir un modelo que identifique su tipo automáticamente?


\section*{Desarrollo}

\subsection*{Actividad 1: Definición de aprendizaje supervisado y no supervisado (10 minutos)}  
Explicaremos las definiciones básicas:
\begin{itemize}
    \item \textbf{Supervisado:} El modelo aprende a partir de datos etiquetados, buscando una relación entre entradas y salidas. 
    \item \textbf{No supervisado:} El modelo identifica patrones en datos no etiquetados. 
\end{itemize}

\paragraph{Recursos:} Presentación visual con ejemplos.  

\paragraph{Verificación de aprendizaje:} Presentación de ejemplos para determinar tipo de aprendizaje.

\subsection*{Actividad 2: Tipos de tareas y modelos principales  (20 minutos)}  
Explicaremos los tipos de tareas y los modelos más utilizados:
\begin{itemize}
    \item \textbf{Clasificación:} Regresión logística, SVM, Árboles de decisión, Redes neuronales.
    \item \textbf{Regresión:} Regresión lineal, Regresión polinómica, KNN, Redes neuronales.
    \item \textbf{Agrupamiento:} Modelos: K-Means, DBSCAN, Hierarchical Clustering.
\end{itemize}

\paragraph{Recursos:} Cuadro resumen.  

\paragraph{Verificación de aprendizaje:} Presentación de ejemplos para determinar tipo de tarea.

\subsection*{Actividad 3: Explicación de métricas de distancia (15 minutos)}  
Detallaremos las métricas más comunes:
\begin{itemize}
    \item Euclidiana: $\sqrt{\sum (x_i - y_i)^2}$
    \item Mahalanobis: Basada en la covarianza.
    \item Hamming: Comparación binaria.
    \item Minkowski: Generalización de las métricas Euclidiana y Manhattan.
\end{itemize}

\paragraph{¿Cómo lo haremos?}  
\begin{itemize}[leftmargin=*]
    \item \textbf{Exploración del cuaderno de Jupyter:}  
    Se proporcionará un cuaderno de Jupyter con ejemplos prácticos de cálculo de distintas métricas de distancia utilizadas en aprendizaje automático.
    \begin{quote}
        Enlace al cuaderno: \href{https://colab.research.google.com/github/andres-merino/AprendizajeAutomaticoInicial-05-N0105/blob/main/2-Notebooks/02-Metricas-Distancia.ipynb}{02-Metricas-Distancia}.
    \end{quote}
\end{itemize}

\section*{Cierre}

\paragraph{Tarea:}  
    No se deja tarea.

\paragraph{Pregunta de investigación:}  
    \begin{itemize}
        \item ¿Qué es la distancia de Mahalanobis y en qué casos se usa?
        \item ¿Cómo afecta la escala de las variables a las métricas de distancia? ¿Cómo podemos solucionarlo?
        \item ¿Qué es el método del codo (Elbow Method) en KMeans?
    \end{itemize}

\end{document} 