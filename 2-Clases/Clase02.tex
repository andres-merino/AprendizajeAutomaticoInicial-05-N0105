\documentclass[a4,11pt]{aleph-notas}

% -- Paquetes adicionales 
\usepackage{enumitem}
\usepackage{url}
\usepackage{array}
\usepackage{booktabs}
\hypersetup{
    urlcolor=blue,
    linkcolor=blue,
}


% -- Datos 
\institucion{Facultad de Ciencias Exactas, Naturales y Ambientales}
\carrera{Ciencia de Datos}
\asignatura{Aprendizaje Automático Inicial}
\tema{Clase 02: Conceptos básicos del Aprendizaje Automático}
\autor{Andrés Merino}
\fecha{Periodo 2025-2}

\logouno[0.14\textwidth]{Logos/logoPUCE_04_ac}
\definecolor{colortext}{HTML}{0030A1}
\definecolor{colordef}{HTML}{0030A1}
\fuente{montserrat}

% -- Comandos adicionales (comentados si no se usan)
% \usepackage{listings}
\input{listings-python.prf}

\usepackage[spanish,onelanguage,vlined,linesnumbered]{algorithm2e}

% -- Comandos adicionales
\newtcolorbox{pscodigo}
    {icono=\faCogs,color=lightgray,postit,top=-1.5mm,bottom=-1.5mm}
    
\definecolor{colcod}{RGB}{174,218,255}
\newtcolorbox{pycodigo}
    {icono=\faKeyboardO, color=colcod, postit, 
    top=-2mm, bottom=-2mm, 
    extras first={bottom=0mm},
    extras last={top=0mm},
    extras middle={top=0mm,bottom=0mm},
    }

\lstloadlanguages{Python}
\lstset{
  language=Python,
  basicstyle=\small\sffamily,
  stringstyle=\color[HTML]{933797},
  commentstyle=\color[HTML]{228B22}\sffamily,
  emph={[2]from,import,pass,return}, emphstyle={[2]\color[HTML]{DD52F0}},
  emph={[3]range}, emphstyle={[3]\color[HTML]{D17032}},
  emph={[4]for,in,def}, emphstyle={[4]\color{blue}},
  showstringspaces=false,
  breaklines=true,
  prebreak=\mbox{{\color{gray}\tiny$\searrow$}},
  xleftmargin=3pt,
  inputencoding=utf8,
  extendedchars=true,
  columns=fullflexible,
  literate={á}{{\'a}}1 {é}{{\'e}}1 {í}{{\'i}}1 {ó}{{\'o}}1 {ú}{{\'u}}1,
}


\SetKwFunction{concat}{Concatenar}
\SetKwProg{Fn}{Función}{\string:}{}
\SetKwFunction{ult}{Ultimo}
\SetKwFunction{pri}{Primero}
\SetKwFunction{sinul}{SinUltimo}
\SetKw{Salir}{Salir}

\newcommand{\fuentecomentario}[1]{\scriptsize\ttfamily #1}
\SetCommentSty{fuentecomentario}
\SetAlFnt{\footnotesize}


\begin{document}

\encabezado
% En todo el documento, las indicaciones deben ser simples y directas, con una sola oración.

%%%%%%%%%%%%%%%%%%%%%%%%%%%%%%%%%%%%%%%%
\section*{Resultado de Aprendizaje}
%%%%%%%%%%%%%%%%%%%%%%%%%%%%%%%%%%%%%%%%

%%%%%%%%%%%%%%%%%%%%%%%%%%%%%%%%%%%%%%%%
\subsection*{RdA de la asignatura:}
% Se toma uno de los siguientes
\begin{itemize}[leftmargin=*]
    \item \textbf{RdA 1:} Plantear los conceptos fundamentales del aprendizaje automático, incluyendo los principios básicos, técnicas de preprocesado de datos, métodos de evaluación y ajuste de modelos, destacando su importancia en el análisis y resolución de problemas de datos.
\end{itemize}

%%%%%%%%%%%%%%%%%%%%%%%%%%%%%%%%%%%%%%%%
\subsection*{RdA de la clase:}
% Máximo 3 resultados
\begin{itemize}[leftmargin=*]
    \item Definir qué es aprendizaje supervisado y no supervisado, identificando los tipos de tareas asociadas (clasificación, regresión y agrupamiento).
    \item Enumerar los principales modelos utilizados en cada tipo de tarea.
    \item Comprender las métricas comunes utilizadas para medir distancias.
\end{itemize}

%%%%%%%%%%%%%%%%%%%%%%%%%%%%%%%%%%%%%%%%
\section*{Introducción}
%%%%%%%%%%%%%%%%%%%%%%%%%%%%%%%%%%%%%%%%

%%%%%%%%%%%%%%%%%%%%%%%%%%%%%%%%%%%%%%%%
\paragraph{Pregunta inicial:} 
Si tuvieras un conjunto de datos de imágenes de flores, ¿qué datos se necesitaría para construir un modelo que identifique su tipo automáticamente?

%%%%%%%%%%%%%%%%%%%%%%%%%%%%%%%%%%%%%%%%
\section*{Desarrollo}
%%%%%%%%%%%%%%%%%%%%%%%%%%%%%%%%%%%%%%%%

%%%%%%%%%%%%%%%%%%%%%%%%%%%%%%%%%%%%%%%%
\subsection*{Actividad 1: Definición de aprendizaje supervisado y no supervisado}

En esta actividad se explicarán las definiciones básicas del aprendizaje supervisado y no supervisado mediante clase magistral con presentación visual y ejemplos prácticos para identificar cada tipo de aprendizaje.

\paragraph{¿Cómo lo haremos?}  
\begin{itemize}[leftmargin=*]
    \item \textbf{Conceptos fundamentales:}  
    Se presentarán las definiciones: Aprendizaje supervisado (el modelo aprende a partir de datos etiquetados, buscando una relación entre entradas y salidas) y aprendizaje no supervisado (el modelo identifica patrones en datos no etiquetados).
    \item \textbf{Presentación de ejemplos:}
    Se mostrarán casos prácticos para que los estudiantes identifiquen el tipo de aprendizaje correspondiente.

    
\end{itemize}

%%%%%%%%%%%%%%%%%%%%%%%%%%%%%%%%%%%%%%%%
\subsection*{Actividad 2: Tipos de tareas y modelos principales}

En esta actividad se explicarán los tipos de tareas del aprendizaje automático y los modelos más utilizados en cada una, mediante clase magistral con cuadro resumen y ejemplos de aplicación.

\paragraph{¿Cómo lo haremos?}  
\begin{itemize}[leftmargin=*]
    \item \textbf{Clasificación de tareas:}  
    Se presentarán las tres tareas principales:
    \begin{itemize}
        \item \textbf{Clasificación:} Regresión logística, SVM, Árboles de decisión, Redes neuronales.
        \item \textbf{Regresión:} Regresión lineal, Regresión polinómica, KNN, Redes neuronales.
        \item \textbf{Agrupamiento:} K-Means, DBSCAN, Hierarchical Clustering.
    \end{itemize}
    \item \textbf{Análisis de ejemplos:}
    Se presentarán casos prácticos para que los estudiantes identifiquen el tipo de tarea correspondiente.
    
\end{itemize}

%%%%%%%%%%%%%%%%%%%%%%%%%%%%%%%%%%%%%%%%
\subsection*{Actividad 3: Explicación de métricas de distancia}

En esta actividad se detallarán las métricas de distancia más comunes utilizadas en aprendizaje automático, mediante exploración de un cuaderno de Jupyter con ejemplos prácticos de cálculo.

\paragraph{¿Cómo lo haremos?}  
\begin{itemize}[leftmargin=*]
    \item \textbf{Métricas de distancia:}  
    Se presentarán las métricas más comunes:
    \begin{itemize}
        \item Euclidiana: $\sqrt{\sum (x_i - y_i)^2}$
        \item Mahalanobis: Basada en la covarianza.
        \item Hamming: Comparación binaria.
        \item Minkowski: Generalización de las métricas Euclidiana y Manhattan.
    \end{itemize}
    % \item \textbf{Implementación en Python:} Los estudiantes accederán a un cuaderno de Jupyter previamente preparado.
    % \begin{quote}
    %     Enlace al cuaderno: \href{https://colab.research.google.com/github/andres-merino/AprendizajeAutomaticoInicial-05-N0105/blob/main/2-Notebooks/02-Metricas-Distancia.ipynb}{02-Metricas-Distancia.ipynb}.
    % \end{quote}
    % \item \textbf{Experimentación:} 
    
\end{itemize}

%%%%%%%%%%%%%%%%%%%%%%%%%%%%%%%%%%%%%%%%
\section*{Cierre}
%%%%%%%%%%%%%%%%%%%%%%%%%%%%%%%%%%%%%%%%

\paragraph{Verificación de aprendizaje:} 
\begin{enumerate}[leftmargin=*]
    \item ¿Cuál es la diferencia fundamental entre el aprendizaje supervisado y el no supervisado?
    % Respuesta: Supervisado usa datos etiquetados (objetivo conocido) para aprender una función de mapeo entre entradas y salidas, haciendo predicciones. No supervisado trabaja con datos sin etiquetas, buscando patrones, estructuras o agrupaciones naturales sin un objetivo predefinido.
    
    \item ¿Qué tipo de problemas se resuelven con aprendizaje supervisado y cuáles con no supervisado?
    % Respuesta: Supervisado: clasificación (spam/no spam), regresión (predicción de precios), reconocimiento de patrones con ejemplos conocidos. No supervisado: clustering (segmentación de clientes), reducción de dimensionalidad (visualización), detección de anomalías, descubrimiento de patrones ocultos.
\end{enumerate}

\paragraph{Preguntas tipo entrevista:} 
\begin{enumerate}[leftmargin=*]
    \item Explica con un ejemplo práctico cómo decidirías entre usar aprendizaje supervisado o no supervisado para un problema de negocio real.
    % Respuesta: Ejemplo e-commerce: supervisado si tengo histórico de "clientes que compraron/no compraron" (clasificación de conversión). No supervisado si quiero descubrir segmentos naturales de clientes sin criterio predefinido (clustering por comportamiento). Decisión: ¿tengo el outcome que quiero predecir? → supervisado. ¿Quiero explorar estructura sin objetivo específico? → no supervisado.
    
    \item Si tienes datos etiquetados, ¿siempre debes usar aprendizaje supervisado?
    % Respuesta: No siempre. Razones para usar no supervisado con datos etiquetados: (1) Exploración inicial: descubrir clusters naturales antes de modelar, (2) Detección de outliers: identificar datos anómalos que pueden afectar el entrenamiento, (3) Feature engineering: PCA para crear features, (4) Validación de etiquetas: verificar si los clusters coinciden con etiquetas (posibles errores de etiquetado). Usar ambos es común: no supervisado para exploración, supervisado para predicción.
\end{enumerate}

\paragraph{Tarea:} 
    Identificar un conjunto de datos con la menos 5 características sobre los cuales se pueda aplicar modelos de aprendizaje supervisado y no supervisado, el cual será usado en los talleres durante todo el curso.

\paragraph{Pregunta de investigación:} 
\begin{enumerate}[leftmargin=*]
    \item ¿Qué es la distancia de Mahalanobis y en qué casos se usa?
    \item ¿Cómo afecta la escala de las variables a las métricas de distancia? ¿Cómo podemos solucionarlo?
    \item ¿Qué es el método del codo (Elbow Method) en KMeans?
\end{enumerate}
    
\paragraph{Para la próxima clase:}


\end{document}