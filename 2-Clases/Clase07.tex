\documentclass[a4,11pt]{aleph-notas}
% Se puede ver la documentación aquí: 
% https://github.com/alephsub0/LaTeX_aleph-notas


% -- Paquetes adicionales 
\usepackage{enumitem}
\usepackage{url}
\usepackage{array}
\usepackage{booktabs}
\usepackage{longtable}
\usepackage{environ}  % Para definir nuevos entornos
\hypersetup{
    urlcolor=blue,
    linkcolor=blue,
}

% Definición del nuevo ambiente
\NewEnviron{materiales}{%
    \begin{minipage}{12cm}
        \vspace{0.1mm}
        \begin{itemize}[leftmargin=*]
            \BODY % Contenido de la lista
        \end{itemize}
        \vspace{1mm}
    \end{minipage}
}


% -- Datos 
\institucion{Escuela de Ciencias Físicas y Matemática}
\carrera{Ciencia de Datos}
\asignatura{Aprendizaje Automático Inicial}
\tema{Clase 07: Introducción al Aprendizaje No Supervisado}
\autor{Andrés Merino}
\fecha{Semestre 2025-1}

\logouno[0.14\textwidth]{Logos/logoPUCE_04_ac}
\definecolor{colortext}{HTML}{0030A1}
\definecolor{colordef}{HTML}{0030A1}
\fuente{montserrat}


% -- Comandos para tablas
\usepackage{listings}
\input{listings-python.prf}

\usepackage[spanish,onelanguage,vlined,linesnumbered]{algorithm2e}

% -- Comandos adicionales
\newtcolorbox{pscodigo}
    {icono=\faCogs,color=lightgray,postit,top=-1.5mm,bottom=-1.5mm}
    
\definecolor{colcod}{RGB}{174,218,255}
\newtcolorbox{pycodigo}
    {icono=\faKeyboardO, color=colcod, postit, 
    top=-2mm, bottom=-2mm, 
    extras first={bottom=0mm},
    extras last={top=0mm},
    extras middle={top=0mm,bottom=0mm},
    }

\lstloadlanguages{Python}
\lstset{
  language=Python,
  basicstyle=\small\sffamily,
  stringstyle=\color[HTML]{933797},
  commentstyle=\color[HTML]{228B22}\sffamily,
  emph={[2]from,import,pass,return}, emphstyle={[2]\color[HTML]{DD52F0}},
  emph={[3]range}, emphstyle={[3]\color[HTML]{D17032}},
  emph={[4]for,in,def}, emphstyle={[4]\color{blue}},
  showstringspaces=false,
  breaklines=true,
  prebreak=\mbox{{\color{gray}\tiny$\searrow$}},
  xleftmargin=3pt,
  inputencoding=utf8,
  extendedchars=true,
  columns=fullflexible,
  literate={á}{{\'a}}1 {é}{{\'e}}1 {í}{{\'i}}1 {ó}{{\'o}}1 {ú}{{\'u}}1,
}


\SetKwFunction{concat}{Concatenar}
\SetKwProg{Fn}{Función}{\string:}{}
\SetKwFunction{ult}{Ultimo}
\SetKwFunction{pri}{Primero}
\SetKwFunction{sinul}{SinUltimo}
\SetKw{Salir}{Salir}

\newcommand{\fuentecomentario}[1]{\scriptsize\ttfamily #1}
\SetCommentSty{fuentecomentario}
\SetAlFnt{\footnotesize}


\begin{document}

\encabezado


\section*{Resultado de Aprendizaje}

\subsection*{RdA de la asignatura:}
\begin{itemize}[leftmargin=*]
    \item \textbf{RdA 1:} 
    Plantear los conceptos fundamentales del aprendizaje automático, incluyendo los principios básicos, técnicas de preprocesado de datos, métodos de evaluación y ajuste de modelos, destacando su importancia en el análisis y resolución de problemas de datos.
\end{itemize}

\subsection*{RdA de la actividad:}
    \begin{itemize}[leftmargin=*]
        \item Comprender qué es el aprendizaje no supervisado y sus aplicaciones generales.
        \item Diferenciar entre los tipos de agrupamiento: exclusivo, superposición, jerárquico y probabilístico.
        \item Identificar preguntas clave y reflexiones iniciales sobre el tema mediante la interacción con un modelo de IA.
    \end{itemize}

\section*{Introducción}

\paragraph{Pregunta inicial:} 
¿Cómo podríamos encontrar patrones en un conjunto de datos sin etiquetas?

\section*{Desarrollo}

\subsection*{Actividad 1: Introducción al Aprendizaje No Supervisado}

Se utilizará un video introductorio para familiarizar a los estudiantes con el concepto de aprendizaje no supervisado y sus aplicaciones en problemas de datos sin etiquetas.

\paragraph{¿Cómo lo haremos?}  
\begin{itemize}[leftmargin=*]
    \item \textbf{Proyección de video:}  
    Se presentará un video educativo que explique qué es el aprendizaje no supervisado.
    \begin{quote}
        Enlace al video: \href{https://www.youtube.com/watch?v=WM7XaTyX7O8}{Qué es el Aprendizaje No Supervisado}.
    \end{quote}
    \item \textbf{Interacción con ChatGPT:}  
    Después del video, cada estudiante formulará preguntas a ChatGPT relacionadas con los términos o conceptos que no entendieron durante la visualización.
\end{itemize}

\paragraph{Verificación de aprendizaje:}  
Se llevará a cabo una discusión grupal donde los estudiantes compartirán:
\begin{itemize}[leftmargin=*]
    \item Los términos que investigaron.
    \item Las respuestas proporcionadas por ChatGPT.
\end{itemize}
Términos guía para las preguntas: K-Means, Clusterización jerárquica, DBSCAN, Modelos de mezclas gaussianas, Ruido en los datos, Detección de anomalías.

\subsection*{Actividad 2: Definiciones clave}

En esta actividad, los estudiantes explorarán las definiciones y características de los principales tipos de agrupamiento en el aprendizaje no supervisado.

\paragraph{¿Cómo lo haremos?}  
\begin{itemize}[leftmargin=*]
    \item \textbf{Presentación de conceptos:}  
    El docente explicará brevemente las definiciones y ejemplos de los principales tipos de agrupamiento en aprendizaje no supervisado:
    \begin{itemize}
        \item \textbf{Agrupamiento exclusivo:} Cada dato pertenece exclusivamente a un único grupo. Ejemplo: k-means.
        \item \textbf{Agrupamiento con superposición:} Los datos pueden pertenecer a múltiples grupos con grados de pertenencia. Ejemplo: clustering difuso (\textit{fuzzy clustering}).
        \item \textbf{Agrupamiento jerárquico:} Los datos son organizados en una estructura de árbol jerárquico. Ejemplo: dendrogramas.
        \item \textbf{Agrupamiento probabilístico:} Los datos se asignan a clústeres basándose en modelos probabilísticos. Ejemplo: Gaussian Mixture Models (GMMs).
    \end{itemize}
    \item \textbf{Exploración guiada:}  
    Los estudiantes realizarán dos tareas principales en grupos pequeños:
    \begin{enumerate}
        \item \textbf{Preguntar a ChatGPT:} Cada grupo formulará las siguientes preguntas al modelo:
        \begin{itemize}
            \item ¿Cuál es la principal diferencia entre el agrupamiento jerárquico y el agrupamiento exclusivo?
            \item ¿Cuándo es preferible usar un modelo probabilístico como GMM frente a k-means?
            \item ¿Qué limitaciones tienen los métodos de agrupamiento con superposición en aplicaciones reales?
        \end{itemize}
        \item \textbf{Búsqueda de referencias:} Usando la plataforma \href{https://consensus.app/}{Consensus}, los estudiantes buscarán un artículo científico que respalde o amplíe las respuestas obtenidas de ChatGPT. Deberán identificar la idea principal del artículo y cómo se relaciona con el tema discutido.
    \end{enumerate}
\end{itemize}

\paragraph{Verificación de aprendizaje:}  
Cada grupo presentará sus hallazgos de la siguiente manera:
\begin{itemize}[leftmargin=*]
    \item Resumen de las respuestas obtenidas de ChatGPT.
    \item Referencia científica encontrada en Consensus y cómo esta valida o amplía las respuestas del modelo.
\end{itemize}
El docente facilitará la discusión y aclarará dudas adicionales sobre los conceptos.

\subsection*{Actividad 3: Efecto de la Normalización en la Clusterización}

En esta actividad, los estudiantes analizarán cómo las diferentes técnicas de normalización pueden afectar el resultado de los algoritmos de clusterización.

\paragraph{¿Cómo lo haremos?}  
\begin{itemize}[leftmargin=*]
    \item \textbf{Presentación inicial:}  
    Se presenta un gráfico que muestra cómo diferentes normalizaciones (por ejemplo, \textit{StandardScaler}, \textit{MinMaxScaler}, \textit{RobustScaler}, \textit{Normalizer}) afectan la distribución de los datos en un espacio bidimensional.
    
    \item \textbf{Exploración individual:}  
    Cada estudiante deberá analizar el gráfico y responder preguntas como:
    \begin{itemize}
        \item ¿Cuál técnica de normalización parece más adecuada para separar los grupos visualmente?
        \item ¿Cómo podrían estas diferencias impactar en algoritmos como k-means o clustering jerárquico?
    \end{itemize}

     \item \textbf{Exploración del cuaderno de Jupyter:}  
        Se proporcionará un cuaderno de Jupyter con ejemplos prácticos sobre el efecto de la normalización y el escalado en el agrupamiento K-Means.
    \begin{quote}
        Enlace al cuaderno: \href{https://colab.research.google.com/github/andres-merino/AprendizajeAutomaticoInicial-05-N0105/blob/main/2-Notebooks/07-Normalizacion.ipynb}{07-Normalizacion}.
    \end{quote}

\end{itemize}


\paragraph{Verificación de aprendizaje:}  
\begin{itemize}[leftmargin=*]
    \item Cada estudiante dará un análisis breve que incluya:
        \begin{itemize}
            \item Observaciones sobre cómo la normalización afecta los resultados de k-means.
            \item Una conclusión sobre qué técnica consideran más adecuada para este conjunto de datos y por qué.
        \end{itemize}
\end{itemize}



\section*{Cierre}

\paragraph{Tarea:}  
Los estudiantes deberán realizar las siguientes actividades de forma individual:
\begin{itemize}[leftmargin=*]
    \item Buscar en \href{https://www.kaggle.com/}{Kaggle} un conjunto de datos que sea adecuado para implementar un algoritmo de clusterización. Ejemplos de conjuntos de datos incluyen: segmentación de clientes, agrupación de películas o canciones, o análisis de patrones en datos demográficos.
    \item Descargar el conjunto de datos seleccionado e identificar qué características podrían necesitar preprocesamiento (como normalización o manejo de valores faltantes).
\end{itemize}

\paragraph{Pregunta de investigación:}  
\begin{enumerate}[leftmargin=*]
    \item ¿Cómo podrían detectarse anomalías en un conjunto de datos usando aprendizaje no supervisado?
    \item ¿Qué factores pueden dificultar la identificación de clústeres claros en datos reales, y cómo podrías abordarlos?
\end{enumerate}
    


\end{document} 