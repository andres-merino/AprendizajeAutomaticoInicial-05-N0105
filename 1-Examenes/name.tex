% \documentclass[11pt,respuestas,a4]{aleph-examen}
\documentclass[11pt,a4]{aleph-examen}
% Se puede ver la documentación aquí: 
% https://github.com/alephsub0/LaTeX_aleph-examen

% -- Paquetes adicionales 
\usepackage{aleph-comandos}
\usepackage{booktabs}
\usepackage{multicol}

% -- Datos 
\institucion{Escuela de Ciencias Físicas y Matemática}
\carrera{Biología}
\asignatura{Matemática I}
\tema{Examen escrito}
\autor{}
\fecha{}


\logouno[0.14\textwidth]{Logos/logoPUCE_04_ac}
\definecolor{colortext}{HTML}{0030A1}
\definecolor{colordef}{HTML}{0030A1}
\fuente{montserrat}


\begin{document}

\encabezado

\section*{Indicaciones}
\begin{itemize}[leftmargin=*]
\item
    Se encuentra prohibido el uso de cualquier fuente de información durante todo el examen.
\item
    En caso de considerar que existe un error en la pregunta o que esta se encuentra mal planteada, se debe indicar cuál es el error y justificarlo.
\item
    Todas las soluciones deben estar correctamente redactadas y explicadas.
\end{itemize}

\section*{Ejercicios}

\begin{preguntas}

%%%%%%%%%%%%%%%%%%%%%%%%%%%%%%%%%%%%%%%%
%%%%%%%%%%%%%%%%%%%%%%%%%%%%%%%%%%%%%%%%
%%%%%%%%%%%%%%%%%%%%%%%%%%%%%%%%%%%%%%%%
\item
    Use la definición de derivada y determine la derivada \( f'(x) \) de la función \( f(x) = x^2 + 7 \).
%%%%%%%%%%%%%%%%%%%%%%%%%%%%%%%%%%%%%%%%
%%%%%%%%%%%%%%%%%%%%%%%%%%%%%%%%%%%%%%%%
%%%%%%%%%%%%%%%%%%%%%%%%%%%%%%%%%%%%%%%%
\item
    Sea la función real definida por \( f(x) = 3x^2e^x \). Mediante regla de derivación, determine \( f'(x) \).
%%%%%%%%%%%%%%%%%%%%%%%%%%%%%%%%%%%%%%%%
%%%%%%%%%%%%%%%%%%%%%%%%%%%%%%%%%%%%%%%%
%%%%%%%%%%%%%%%%%%%%%%%%%%%%%%%%%%%%%%%%
\item
    Sea la función real definida por \( f(x) = \dfrac{1 + x}{1 - x} \). Mediante regla de derivación, determine \( f'(x) \).

%%%%%%%%%%%%%%%%%%%%%%%%%%%%%%%%%%%%%%%%
%%%%%%%%%%%%%%%%%%%%%%%%%%%%%%%%%%%%%%%%
%%%%%%%%%%%%%%%%%%%%%%%%%%%%%%%%%%%%%%%%
\item
    Obtenga los extremos absolutos (o relativos) si existen de la función:
    \( f(x) = x^2 (x - 2)^2 \)

%%%%%%%%%%%%%%%%%%%%%%%%%%%%%%%%%%%%%%%%
%%%%%%%%%%%%%%%%%%%%%%%%%%%%%%%%%%%%%%%%
%%%%%%%%%%%%%%%%%%%%%%%%%%%%%%%%%%%%%%%%
\item
    Durante una epidemia de influenza, el porcentaje de la población de Montreal que ha sido infectada en el tiempo \( t \) (medido en días desde el inicio de la epidemia) está dado por \[ p(t) = \dfrac{100t}{t^2 + 36}. \] Determine el tiempo en el cual \( p(t) \) es máximo y dibuje la gráfica de \( p(t) \).


\begin{respuesta}
    Solución
\end{respuesta}


\end{preguntas}


\end{document}