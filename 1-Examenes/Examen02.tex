% \documentclass[11pt,respuestas,a4]{aleph-examen}
\documentclass[11pt,a4]{aleph-examen}
% Se puede ver la documentación aquí: 
% https://github.com/alephsub0/LaTeX_aleph-examen

% -- Paquetes adicionales 
\usepackage{aleph-comandos}
\usepackage{booktabs}
\usepackage{multicol}

% -- Datos 
\institucion{Escuela de Ciencias Físicas y Matemática}
\carrera{Ciencia de Datos}
\asignatura{Aprendizaje Automático Inicial}
\tema{Examen escrito no. 2}
\autor{Andrés Merino}
\fecha{Semestre 2024-2}


\logouno[0.14\textwidth]{Logos/logoPUCE_04_ac}
\definecolor{colortext}{HTML}{0030A1}
\definecolor{colordef}{HTML}{0030A1}
\fuente{montserrat}


\begin{document}

\encabezado

\section*{Indicaciones}
\begin{itemize}[leftmargin=*]
\item 
    En esta actividad se evalúa si el estudiante \textit{(Criterio 2.2) Resuelve sistemas de ecuaciones e inecuaciones de varias variables, aplicando métodos algebraicos y geométricos adecuados.} 
\item
    Se encuentra prohibido el uso de cualquier fuente de información durante todo el examen.
\item
    En caso de considerar que existe un error en la pregunta o que esta se encuentra mal planteada, se debe indicar cuál es el error y justificarlo.
\item
    Todas las soluciones deben estar correctamente redactadas y explicadas.
\end{itemize}

\section*{Ejercicios}

\begin{preguntas}

%%%%%%%%%%%%%%%%%%%%%%%%%%%%%%%%%%%%%%%%
%%%%%%%%%%%%%%%%%%%%%%%%%%%%%%%%%%%%%%%%
%%%%%%%%%%%%%%%%%%%%%%%%%%%%%%%%%%%%%%%%
\item
    Pregunta

\begin{respuesta}
    Solución
\end{respuesta}


\end{preguntas}


\end{document}