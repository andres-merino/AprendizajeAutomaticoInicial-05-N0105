\documentclass[a4,11pt]{aleph-notas}
% Se puede ver la documentación aquí: 
% https://github.com/alephsub0/LaTeX_aleph-notas

% -- Paquetes adicionales 
\usepackage{enumitem}
\usepackage{aleph-comandos}
\usepackage{booktabs}
\usepackage{pdflscape}

\usepackage{array}
\newcolumntype{L}[1]{>{\raggedright\arraybackslash}p{#1}}

% -- Datos 
\institucion{Facultad de Ciencias Exactas, Naturales y Ambientales}
\carrera{Ciencia de Datos}
\asignatura{Aprendizaje Automático Inicial}
\tema{Presentación Interactiva no. 1}
\autor{Andrés Merino}
\fecha{Semestre 2024-2}

\logouno[0.14\textwidth]{Logos/logoPUCE_04_ac}
\definecolor{colortext}{HTML}{0030A1}
\definecolor{colordef}{HTML}{0030A1}
\fuente{montserrat}


% -- Comandos adicionales
% \setlist[enumerate]{label=\roman*.}


\begin{document}

\encabezado

%%%%%%%%%%%%%%%%%%%%%%%%%%%%%%%%%%%%%%%%
\section{Indicaciones}
%%%%%%%%%%%%%%%%%%%%%%%%%%%%%%%%%%%%%%%%

\begin{itemize}[leftmargin=*]
\item En esta actividad se evalúa si el estudiante \textit{(Criterio 1.2) Describe los métodos de aprendizaje no supervisado, sus características, aplicaciones, alcance y limitaciones.}
\item Se espera la utilización de herramientas tecnológicas para crear una presentación interactiva que cumpla con los estándares establecidos en la rúbrica.
\item La presentación deberá tener una duración máxima de 7 minutos.
\end{itemize}

%%%%%%%%%%%%%%%%%%%%%%%%%%%%%%%%%%%%%%%%
\section{Descripción}
%%%%%%%%%%%%%%%%%%%%%%%%%%%%%%%%%%%%%%%%

\begin{enumerate}[leftmargin=*]
    \item \textbf{Preparación del Material:} Se debe elaborar una presentación interactiva sobre los métodos de aprendizaje no supervisado.
    \begin{itemize}
        \item Incluirán características principales, aplicaciones, alcance y limitaciones de los métodos de aprendizaje no supervisado.
        \item Utilizarán herramientas tecnológicas para diseñar y presentar los contenidos.
    \end{itemize}
    \item \textbf{Estructura de la Presentación:}
    \begin{itemize}[leftmargin=*]
        \item \textbf{Introducción:} Explicación general del aprendizaje no supervisado y su importancia en el campo de la ciencia de datos.
        \item \textbf{Desarrollo:}
        \begin{itemize}
            \item Descripción de métodos clave (al menos dos).
            \item Características técnicas de los métodos presentados.
            \item Ejemplos prácticos y aplicaciones reales.
        \end{itemize}
        \item \textbf{Conclusión:} Resumen del alcance y las limitaciones de los métodos discutidos.
    \end{itemize}
    \item \textbf{Claridad y Profundidad:} La presentación deberá ser clara, estructurada y dirigida a un público con conocimientos básicos en aprendizaje automático.
    \item \textbf{Duración y Entrega:} 
    \begin{itemize}
        \item La duración máxima de la presentación es de 7 minutos.
        \item Se deberá entregar el enlace (de OneDrive) al video de la exposición a través del aula virtual antes de la fecha límite establecida.
    \end{itemize}
\end{enumerate}





%%%%%%%%%%%%%%%%%%%%%%%%%%%%%%%%%%%%%%%%
\section{Rúbrica de evaluación}
%%%%%%%%%%%%%%%%%%%%%%%%%%%%%%%%%%%%%%%%



\begin{center}\small
% \begin{tabular}{p{3cm}*{4}{p{4.8cm}}}
\begin{tabular}{L{3cm}*{4}{L{2.8cm}}}
\toprule
\textbf{Indicador} & \textbf{Excelente (4)} & \textbf{Bueno (3)} & \textbf{Mejorable (2)} & \textbf{Deficiente (1)} \\ \midrule
\textbf{Métodos de Aprendizaje No Supervisado} & Explica claramente los métodos y su contexto. & Describe los métodos de forma adecuada. & Menciona los métodos superficialmente. & No describe correctamente los métodos. \\ \midrule
\textbf{Características} & Detalla las principales características. & Menciona características con claridad moderada. & Describe características de forma incompleta. & No aborda las características principales. \\ \midrule
\textbf{Aplicaciones} & Identifica y explica aplicaciones prácticas. & Describe aplicaciones relevantes sin ejemplos. & Menciona aplicaciones sin profundidad. & No menciona aplicaciones prácticas. \\ \midrule
\textbf{Alcance y Limitaciones} & Detalla el alcance y las limitaciones con ejemplos. & Menciona alcance y limitaciones brevemente. & Describe alcance o limitaciones superficialmente. & No aborda alcance ni limitaciones. \\ \midrule
\textbf{Interacción Tecnológica} & Integra herramientas de forma innovadora. & Usa herramientas adecuadamente. & Uso básico de herramientas. & No usa herramientas relevantes. \\ \midrule
\textbf{Uso del Tiempo} & Cumple estrictamente con los 7 minutos. & Gestiona el tiempo con ligeros desbalances. & Excede o no utiliza todo el tiempo asignado. & No respeta el tiempo asignado. \\ \midrule
\textbf{Diseño} & Diseño profesional y coherente. & Diseño visualmente agradable. & Diseño funcional pero poco atractivo. & Diseño desorganizado. \\ \midrule
\textbf{Claridad y Fluidez} & Expone con claridad y fluidez. & Expone claramente, con algunas interrupciones. & Exposición comprensible pero poco fluida. & Exposición confusa o interrumpida. \\ \bottomrule
\end{tabular}
\end{center}



\end{document}