\documentclass[a4,11pt]{aleph-notas}
% Se puede ver la documentación aquí: 
% https://github.com/alephsub0/LaTeX_aleph-notas

% -- Paquetes adicionales 
\usepackage{enumitem}
\usepackage{aleph-comandos}
\usepackage{booktabs}
\usepackage{array}
\newcolumntype{L}[1]{>{\raggedright\arraybackslash}p{#1}}

% -- Datos 
\institucion{Facultad de Ciencias Exactas, Naturales y Ambientales}
\carrera{Ciencia de Datos}
\asignatura{Aprendizaje Automático Inicial}
\tema{Video exposición no. 2}
\autor{Andrés Merino}
\fecha{Periodo 2025-2}

\logouno[0.14\textwidth]{Logos/logoPUCE_04_ac}
\definecolor{colortext}{HTML}{0030A1}
\definecolor{colordef}{HTML}{0030A1}
\fuente{montserrat}


% -- Comandos adicionales
\setlist[enumerate]{label=\roman*.}


\begin{document}

\encabezado

%%%%%%%%%%%%%%%%%%%%%%%%%%%%%%%%%%%%%%%%
\section{Indicaciones}
%%%%%%%%%%%%%%%%%%%%%%%%%%%%%%%%%%%%%%%%

\begin{itemize}[leftmargin=*]
\item 
    En esta actividad se evalúa si el estudiante \textit{(Criterio 1.3) explica los métodos de aprendizaje supervisado, sus características, aplicaciones, alcance y limitaciones.}
\item 
    El estudiante deberá elaborar diapositivas profesionales y grabar un video presentándolas, con el rostro visible durante toda la exposición.
\item 
    En la presentación se deberá incluir el uso de código en \texttt{Python} para la aplicación del método seleccionado, junto con la explicación de los fragmentos más relevantes.
\item 
    La duración máxima del video será de 6 minutos.
\end{itemize}

%%%%%%%%%%%%%%%%%%%%%%%%%%%%%%%%%%%%%%%%
\section{Descripción}
%%%%%%%%%%%%%%%%%%%%%%%%%%%%%%%%%%%%%%%%

\begin{enumerate}[leftmargin=*]
    \item \textbf{Elaboración de diapositivas:}
    \begin{itemize}
        \item Crear una presentación profesional sobre el aprendizaje supervisado (¿qué es?).
        \item Incluir un método de aprendizaje supervisado.
        \item Presentar características, aplicaciones, alcance y limitaciones.
        \item Incluir los hiperparámetros del método, explicando su función.
        \item Se deberá explicar al menos un hiperparámetro adicional a los revisados en clase.
    \end{itemize}

    \item \textbf{Uso de código en Python:}
    \begin{itemize}
        \item Mostrar código en Python para la aplicación del algoritmo.
        \item Explicar los componentes principales del código (modelo, hiperparámetros y entrenamiento).
    \end{itemize}

    \item \textbf{Grabación del video:}
    \begin{itemize}
        \item El estudiante deberá grabarse presentando las diapositivas.
        \item El rostro debe ser visible durante toda la exposición.
        \item No se permite únicamente voz en off.
    \end{itemize}

    \item \textbf{Estructura mínima:}
    \begin{itemize}
        \item \textbf{Introducción:} concepto general de aprendizaje supervisado.
        \item \textbf{Desarrollo:} explicación del método, hiperparámetros y código.
        \item \textbf{Cierre:} alcance y limitaciones.
    \end{itemize}

    \item \textbf{Entrega:}
    \begin{itemize}
        \item Subir al aula virtual el enlace (OneDrive) del video.
        \item Respetar el tiempo máximo establecido.
    \end{itemize}
\end{enumerate}


%%%%%%%%%%%%%%%%%%%%%%%%%%%%%%%%%%%%%%%%
\section{Rúbrica de evaluación}
%%%%%%%%%%%%%%%%%%%%%%%%%%%%%%%%%%%%%%%%

En caso de presentar un video con Voz en off o rostro no visible o fuera del rango de tiempo de 6:00 $\pm$ 90 s, se reducirá un nivel a cada punto de la rúbrica.

\begin{center}\footnotesize
\begin{tabular}{L{3cm}*{4}{L{2.9cm}}}
\toprule
\textbf{Indicador} & \textbf{Excelente} & \textbf{Bueno} & \textbf{Aceptable} & \textbf{Insuficiente} \\ \midrule

\textbf{¿Qué es aprendizaje supervisado? (6 pts)}
& Define con precisión el objetivo del aprendizaje (6)
& Define correctamente con un detalle omitido (5)
& Definición general y poco precisa (3)
& Definición incorrecta o ausente (1) \\ \midrule

\textbf{Explicación del modelo (12 pts)}
& Explica el modelo, componentes y funcionamiento (12)
& Explica correctamente con leves omisiones (10)
& Explicación superficial o parcialmente correcta (7)
& Explicación incorrecta o ausente (1) \\ \midrule

\textbf{Hiperparámetros (8 pts)}
& Explica hiperparámetros y 1 adicional a los vistos en clase (8)
& Explica hiperparámetros con poca profundidad (6)
& Menciona hiperparámetros sin explicar impacto (4)
& No explica hiperparámetros o no incluye el adicional (1) \\ \midrule

\textbf{Código en Python y explicación (10 pts)}
& Muestra código del algoritmo y explica secciones clave (10)
& Código adecuado con explicación parcial (8)
& Código mostrado con explicación mínima o confusa (5)
& No incluye código o no lo explica (1) \\ \midrule

\textbf{Alcance y limitaciones (6 pts)}
& Explica alcance y limitaciones con ejemplos claros (6)
& Explica alcance y limitaciones de forma general (5)
& Menciona sin justificar (3)
& No aborda alcance/limitaciones (1) \\ \midrule

\textbf{Diapositivas (4 pts)}
& Diseño profesional, coherente, legible y bien estructurado (4)
& Diseño correcto con detalles mejorables (3)
& Diseño simple con problemas de legibilidad/orden (2)
& Diseño desorganizado o poco legible (1) \\ \midrule

\textbf{Presentación en video (2 pts)}
& Exposición clara y fluida; rostro visible durante toda la presentación (2)
& Exposición clara con ligeras interrupciones; rostro visible (1.5)
& Exposición poco fluida o lectura excesiva; rostro visible parcial (0.5)
& Voz en off o rostro no visible / exposición ininteligible (0) \\ \midrule

\textbf{Gestión del tiempo (2 pts)}
& 6:00 $\pm$ 15 s (2)
& 6:00 $\pm$ 30 s (1.5)
& 6:00 $\pm$ 60 s (0.5)
& Fuera de $\pm$ 90 s (0) \\ \bottomrule

\end{tabular}
\end{center}

\end{document}