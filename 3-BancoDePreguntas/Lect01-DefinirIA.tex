\documentclass[a4,11pt]{aleph-notas}
% Actualizado en febrero de 2024
% Funciona con TeXLive 2022
% Para obtener solo el pdf, compilar con pdfLaTeX. 
% Para obtener el xml compilar con XeLaTeX.
   
\usepackage{aleph-moodle}
\moodleregisternewcommands
% Todos los comandos nuevos deben ir luego del comando anterior
\usepackage{aleph-comandos}


% -- Datos 
\institucion{Escuela de Ciencias Físicas y Matemática}
\carrera{Ciencia de Datos}
\asignatura{Aprendizaje Automático Inicial}
\tema{Cuestionario de diagnóstico}
\autor{Andrés Merino}
\fecha{Semestre 2024-2}

\logouno[0.14\textwidth]{Logos/logoPUCE_04_ac}
\definecolor{colortext}{HTML}{0030A1}
\definecolor{colordef}{HTML}{0030A1}
\fuente{montserrat}

% -- Otros comandos



\begin{document}

\encabezado

\vspace*{-8mm}
%%%%%%%%%%%%%%%%%%%%%%%%%%%%%%%%%%%%%%%%
\section{Indicaciones}
%%%%%%%%%%%%%%%%%%%%%%%%%%%%%%%%%%%%%%%%

\begin{itemize}[leftmargin=*]
\item 
    En esta actividad se evalúa si el estudiante entendió la lectura «¿Cómo deberíamos definir la IA?», disponible en: \url{https://course.elementsofai.com/es/1/1}.

\end{itemize}

%%%%%%%%%%%%%%%%%%%%%%%%%%%%%%%%%%%%%%%%
\section{Banco de preguntas}
%%%%%%%%%%%%%%%%%%%%%%%%%%%%%%%%%%%%%%%%

%%%%%%%%%%%%%%%%%%%%%%%%%%%%%%%%%%%%%%%%
\begin{quiz}{Lect01-DefinirIA}
%%%%%%%%%%%%%%%%%%%%%%%%%%%%%%%%%%%%%%%%

\begin{multi}[]
{Lect01-DefinirIA01}
¿Qué implica la autonomía como propiedad característica de la IA?
\item* La capacidad para ejecutar tareas en situaciones complejas sin la dirección constante del usuario.
\item La capacidad para analizar datos sin necesidad de programación previa.
\item La habilidad para aprender sin intervención humana.
\item La posibilidad de sustituir completamente al humano en cualquier tarea.
\end{multi}

\begin{multi}[]
{Lect01-DefinirIA02}
Según el texto, ¿cuál es una de las principales dificultades al intentar definir la IA?
\item* No existe una definición oficial consensuada entre los investigadores.
\item La IA solo puede definirse en términos de ciencia ficción.
\item La definición de IA depende exclusivamente de su capacidad de imitar al cerebro humano.
\item Las aplicaciones prácticas de la IA son demasiado limitadas para justificar una definición.
\end{multi}

\begin{multi}[]
{Lect01-DefinirIA03}
¿Qué es una burbuja de filtro en el contexto de los sistemas de recomendación?
\item* Una limitación que restringe a los usuarios a contenido similar al que han consumido previamente.
\item Un error que impide que los algoritmos aprendan de los datos de los usuarios.
\item Un método para mejorar la precisión de las recomendaciones personalizadas.
\item Una técnica utilizada para proteger la privacidad de los usuarios al filtrar sus datos.
\end{multi}

\begin{multi}[]
{Lect01-DefinirIA04}
¿Por qué el término «inteligencia» puede ser engañoso cuando se habla de sistemas de IA?
\item* Puede llevar a pensar que los sistemas son capaces de realizar cualquier tarea humana.
\item Implica que los sistemas tienen emociones humanas.
\item Asume que la IA no comete errores.
\item Supone que la IA puede operar de forma independiente de los datos de entrenamiento.
\end{multi}

\begin{multi}[]
{Lect01-DefinirIA05}
¿Cuál es un ejemplo de cómo las palabras maleta pueden causar confusión en la IA?
\item* Decir que un sistema «entiende» una imagen cuando en realidad segmenta objetos.
\item Describir a la IA como una disciplina científica.
\item Utilizar «IA» como un término no contable.
\item Hablar de autonomía y adaptabilidad en los mismos términos.
\end{multi}

\begin{multi}[]
{Lect01-DefinirIA06}
¿Cuál es una de las diferencias clave entre tareas como mover un objeto y jugar al ajedrez, según el texto?
\item* Mover un objeto es intuitivamente fácil para los humanos pero extremadamente difícil para los robots.
\item Jugar al ajedrez no requiere cálculos complejos, pero mover un objeto sí.
\item Los robots han dominado mover objetos, pero jugar al ajedrez sigue siendo un desafío.
\item Mover un objeto requiere más datos que jugar al ajedrez.
\end{multi}

\begin{multi}[]
{Lect01-DefinirIA07}
¿Por qué no es correcto referirse a la IA como un nombre contable, según el texto?
\item* Porque la IA es una disciplina científica, no una entidad individual.
\item Porque implica que puede haber diferentes tipos de IA independientes entre sí.
\item Porque la IA no es un concepto que se pueda cuantificar.
\item Porque hablar de «inteligencias artificiales" es un término confuso.
\end{multi}

\end{quiz}

\end{document}