\documentclass[a4,11pt]{aleph-notas}
% Actualizado en febrero de 2024
% Funciona con TeXLive 2022
% Para obtener solo el pdf, compilar con pdfLaTeX. 
% Para obtener el xml compilar con XeLaTeX.

% -- Paquetes adicionales
\usepackage{aleph-moodle}
\moodleregisternewcommands
% Todos los comandos nuevos deben ir luego del comando anterior
\usepackage{aleph-comandos}


% -- Datos 
\institucion{Escuela de Ciencias Físicas y Matemática}
\carrera{Ciencia de Datos}
\asignatura{Aprendizaje Automático Inicial}
\tema{Cuestionario de diagnóstico}
\autor{Andrés Merino}
\fecha{Semestre 2024-2}

\logouno[0.14\textwidth]{Logos/logoPUCE_04_ac}
\definecolor{colortext}{HTML}{0030A1}
\definecolor{colordef}{HTML}{0030A1}
\fuente{montserrat}

% -- Otros comandos



\begin{document}

\encabezado

\vspace*{-8mm}
%%%%%%%%%%%%%%%%%%%%%%%%%%%%%%%%%%%%%%%%
\section{Indicaciones}
%%%%%%%%%%%%%%%%%%%%%%%%%%%%%%%%%%%%%%%%

\begin{itemize}[leftmargin=*]
\item 
    En esta actividad se evalúa si el estudiante tiene los conocimientos previos necesarios para la asignatura.

\end{itemize}

%%%%%%%%%%%%%%%%%%%%%%%%%%%%%%%%%%%%%%%%
\section{Banco de preguntas}
%%%%%%%%%%%%%%%%%%%%%%%%%%%%%%%%%%%%%%%%

%%%%%%%%%%%%%%%%%%%%%%%%%%%%%%%%%%%%%%%%
\begin{quiz}{Diagnostico-Conceptos}
%%%%%%%%%%%%%%%%%%%%%%%%%%%%%%%%%%%%%%%%

%%%%%%%%%%%%%%%%%%%%%%%%%%%%%%%%%%%%%%%%
\begin{multi}[]
    {ConceptoVector}
    ¿Cuál es la característica fundamental que define un vector en álgebra lineal?
    \item* Se puede sumar con otros vectores y multiplicar por un escalar
    \item Es siempre un segmento geométrico con magnitud y dirección
    \item Se representa únicamente en un espacio tridimensional
    \item Es una colección de números ordenados sin propiedades adicionales
\end{multi}

\begin{multi}[]
    {TipoVectores}
    ¿Qué tipo de objetos pueden considerarse vectores según el texto?
    \item Geometricos, polinomios, señales de audio y elementos de \( \mathbb{R}^n \)
    \item* Polinomios, matrices y sistemas de ecuaciones
    \item Exclusivamente representaciones geométricas con magnitud
    \item Elementos de \( \mathbb{Z}^n \), pero no de \( \mathbb{R}^n \)
\end{multi}

\begin{multi}[]
    {SistemaEcuaciones}
    ¿Qué puede suceder con el conjunto de soluciones de un sistema de ecuaciones lineales?
    \item* Puede no tener solución, tener una solución única, o tener infinitas soluciones
    \item Siempre tiene una solución única
    \item Puede tener soluciones no lineales además de soluciones lineales
    \item Las soluciones siempre dependen de la matriz identidad
\end{multi}

\end{quiz}


%%%%%%%%%%%%%%%%%%%%%%%%%%%%%%%%%%%%%%%%
\begin{quiz}{Diagnostico-Práctico}
%%%%%%%%%%%%%%%%%%%%%%%%%%%%%%%%%%%%%%%%

\begin{multi}[]
    {EjemploVectorRn}
    Dados los vectores \( a = \begin{bmatrix} 1 \\ 2 \\ 3 \end{bmatrix} \) y \( b = \begin{bmatrix} 4 \\ 5 \\ 6 \end{bmatrix} \), ¿cuál es \( a + b \)?
    \item \( \begin{bmatrix} 3 \\ 7 \\ 9 \end{bmatrix} \)
    \item \( \begin{bmatrix} 5 \\ 7 \\ 9 \end{bmatrix} \)
    \item \( \begin{bmatrix} 6 \\ 8 \\ 10 \end{bmatrix} \)
    \item* \( \begin{bmatrix} 5 \\ 7 \\ 9 \end{bmatrix} \)
\end{multi}

\begin{multi}[]
    {EjemploSistema}
    Considere el sistema de ecuaciones:
    \[
    x_1 + x_2 = 4;\qquad
    2x_1 - x_2 = 2
    \]
    ¿Cuál es la solución?
    \item \( x_1 = 1, x_2 = 2 \)
    \item* \( x_1 = 2, x_2 = 2 \)
    \item \( x_1 = 3, x_2 = 1 \)
    \item Ninguna de las anteriores
\end{multi}

\end{quiz}

%%%%%%%%%%%%%%%%%%%%%%%%%%%%%%%%%%%%%%%%
\begin{quiz}{Diagnostico-Cálculo}
%%%%%%%%%%%%%%%%%%%%%%%%%%%%%%%%%%%%%%%%

\begin{multi}[]
    {ProductoMatrices}
    Dados los matrices:
    \[
    A = \begin{bmatrix} 1 & 2 \\ 3 & 4 \end{bmatrix}, \quad B = \begin{bmatrix} 2 & 0 \\ 1 & 3 \end{bmatrix}
    \]
    ¿Cuál es el producto \( AB \)?
    \item \( \begin{bmatrix} 2 & 6 \\ 8 & 12 \end{bmatrix} \)
    \item \( \begin{bmatrix} 1 & 6 \\ 7 & 12 \end{bmatrix} \)
    \item* \( \begin{bmatrix} 4 & 6 \\ 10 & 12 \end{bmatrix} \)
    \item \( \begin{bmatrix} 3 & 5 \\ 7 & 11 \end{bmatrix} \)
\end{multi}

\begin{multi}[]
    {Determinante}
    Calcule el determinante de la matriz \( A = \begin{bmatrix} 2 & 3 \\ 1 & 4 \end{bmatrix} \).
    \item \( -5 \)
    \item* \( 5 \)
    \item \( 6 \)
    \item \( 1 \)
\end{multi}


\end{quiz}

\end{document}