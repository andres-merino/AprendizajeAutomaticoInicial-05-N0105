\documentclass[a4,11pt]{aleph-notas}
% Se puede ver la documentación aquí: 
% https://github.com/alephsub0/LaTeX_aleph-notas

% -- Paquetes adicionales 
\usepackage{enumitem}
\usepackage{aleph-comandos}
\usepackage{booktabs}
\usepackage[spanish, ruled, vlined, onelanguage]{algorithm2e}

% -- Datos 
\institucion{Facultad de Ciencias Exactas, Naturales y Ambientales}
\carrera{Ciencia de Datos}
\asignatura{Aprendizaje Automático Inicial}
\tema{Resumen no. 9: Agrupamiento $k$-Means}
\autor{Andrés Merino}
\fecha{Periodo 2025-2}

\logouno[0.14\textwidth]{Logos/logoPUCE_04_ac}
\definecolor{colortext}{HTML}{0030A1}
\definecolor{colordef}{HTML}{0030A1}
\fuente{montserrat}


% -- Comandos adicionales
\setlist[enumerate]{label=\roman*.}
\SetAlCapFnt{\normalfont\bfseries}
\SetAlgoCaptionSeparator{\par\nobreak}
\SetAlCapNameFnt{\unskip\itshape}
\SetKwInOut{Input}{Entrada}
\SetKwInOut{Output}{Salida}

\begin{document}

\encabezado

\section{Agrupamiento $k$-Means}

\begin{algorithm}[H]
    \SetKwInput{KwInput}{Entrada}
    \caption{Pseudocódigo del algoritmo $k$-means}

    \Input{$D$ (conjunto de datos) y $k$ (número de clústeres)}
    \Output{El conjunto de clústeres $P$}

    Inicializar los clústeres vacíos $P \leftarrow \{p_1, p_2, \dots, p_k\}$\;
    Seleccionar los \emph{centroides} iniciales $\zeta_1, \zeta_2, \dots, \zeta_k$\;
    
    \While{Condición de parada no satisfecha}{
        \tcc{Fase de asignación}
        \For{todo $d_i \in D$}{
            Encontrar el índice $j$ tal que la distancia $d(d_i, \zeta_j)$ sea mínima\;
            Asignar la instancia $d_i$ al clúster $p_j$\;
        }
        
        \tcc{Fase de actualización}
        \For{todo $p_i \in P$}{
            Recalcular $\zeta_i$ como el promedio de los puntos $d \in p_i$\;
        }
    }
    \Return{$P$}
\end{algorithm}

\end{document}