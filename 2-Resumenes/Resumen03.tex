\documentclass[a4,11pt]{aleph-notas}
% Se puede ver la documentación aquí: 
% https://github.com/alephsub0/LaTeX_aleph-notas

% -- Paquetes adicionales 
\usepackage{enumitem}
\usepackage{aleph-comandos}
\usepackage{booktabs}


% -- Datos 
\institucion{Facultad de Ciencias Exactas, Naturales y Ambientales}
\carrera{Ciencia de Datos}
\asignatura{Aprendizaje Automático Inicial}
\tema{Resumen no. 3: Preprocesamiento de Datos}
\autor{Andrés Merino}
\fecha{Periodo 2025-2}

\logouno[0.14\textwidth]{Logos/logoPUCE_04_ac}
\definecolor{colortext}{HTML}{0030A1}
\definecolor{colordef}{HTML}{0030A1}
\fuente{montserrat}


% -- Comandos adicionales
\setlist[enumerate]{label=\roman*.}


\begin{document}

\encabezado


\section{Normalización de datos}

\begin{defi}[Normalización por el máximo]
    Dado un conjunto de datos \(X = \{x_1, x_2, \ldots, x_n\}\), la normalización por el máximo transforma cada dato \(x_i\) en \(x_i'\) según la fórmula:
    \[
    x_i' = \frac{x_i}{\max(X)}
    \]
\end{defi}

\begin{defi}[Normalización Min-Max]
    Dado un conjunto de datos \(X = \{x_1, x_2, \ldots, x_n\}\), la normalización Min-Max transforma cada dato \(x_i\) en \(x_i'\) según la fórmula:
    \[
    x_i' = \frac{x_i - \min(X)}{\max(X) - \min(X)}
    \]
\end{defi}

\begin{defi}[Estandarización (Z-score)]
    Dado un conjunto de datos \(X = \{x_1, x_2, \ldots, x_n\}\), la estandarización transforma cada dato \(x_i\) en \(x_i'\) según la fórmula:
    \[
    x_i' = \frac{x_i - \mu}{\sigma}
    \]
    donde \(\mu\) es la media del conjunto de datos y \(\sigma\) es la desviación estándar. 
\end{defi}

\section{Discretización de datos}

\begin{defi}[Discretización de igual amplitud]
    Dado un conjunto de datos continuo \(X = \{x_1, x_2, \ldots, x_n\}\) y un número \(k\) de intervalos, la discretización de igual amplitud divide el rango de los datos en \(k\) intervalos de igual tamaño. Cada dato \(x_i\) se asigna al intervalo correspondiente. El tamaño de cada intervalo es:
    \[
    \text{Tamaño del intervalo} = \frac{\max(X) - \min(X)}{k}
    \]
\end{defi}

\begin{defi}[Discretización de igual frecuencia]
    Dado un conjunto de datos continuo \(X = \{x_1, x_2, \ldots, x_n\}\) y un número \(k\) de intervalos, la discretización de igual frecuencia divide los datos en \(k\) intervalos de tal manera que cada intervalo contenga aproximadamente el mismo número de datos. Cada dato \(x_i\) se asigna al intervalo correspondiente.
\end{defi}

\section{Datos categóricos}

\begin{defi}[Codificación One-Hot]
    Dado un conjunto de datos categóricos \(C = \{c_1, c_2, \ldots, c_m\}\) con \(m\) categorías distintas, la codificación One-Hot transforma cada categoría \(c_j\) en un vector binario de longitud \(m\), donde el elemento correspondiente a la categoría es 1 y los demás son 0.
\end{defi}

Ejemplo: En la siguiente tabla se aplica la codificación One-Hot a la variable categórica "Color" con las categorías "Rojo", "Verde" y "Azul".
\begin{center}
    \begin{tabular}{l}
        \toprule
        Color \\
        \midrule
        Rojo \\
        Verde \\
        Azul \\
        Verde \\
        Rojo \\
        \bottomrule
    \end{tabular}
    \quad
    $\xrightarrow{\text{One-Hot}}$
    \quad
    \begin{tabular}{ccc}
        \toprule
        Rojo & Verde & Azul \\
        \midrule
        1 & 0 & 0 \\
        0 & 1 & 0 \\
        0 & 0 & 1 \\
        0 & 1 & 0 \\
        1 & 0 & 0 \\
        \bottomrule 
    \end{tabular}
\end{center}

\end{document}