\documentclass[a4,11pt]{aleph-notas}
% Se puede ver la documentación aquí: 
% https://github.com/alephsub0/LaTeX_aleph-notas

% -- Paquetes adicionales 
\usepackage{enumitem}
\usepackage{aleph-comandos}
\usepackage{booktabs}
\usepackage[spanish, ruled, vlined, onelanguage]{algorithm2e}

% -- Datos 
\institucion{Facultad de Ciencias Exactas, Naturales y Ambientales}
\carrera{Ciencia de Datos}
\asignatura{Aprendizaje Automático Inicial}
\tema{Resumen no. 8: Algoritmos de Agrupamiento Jerárquico}
\autor{Andrés Merino}
\fecha{Periodo 2025-2}

\logouno[0.14\textwidth]{Logos/logoPUCE_04_ac}
\definecolor{colortext}{HTML}{0030A1}
\definecolor{colordef}{HTML}{0030A1}
\fuente{montserrat}


% -- Comandos adicionales
\setlist[enumerate]{label=\roman*.}
\SetAlCapFnt{\normalfont\bfseries}
\SetAlgoCaptionSeparator{\par\nobreak}
\SetAlCapNameFnt{\unskip\itshape}
\SetKwInOut{Input}{Entrada}
\SetKwInOut{Output}{Salida}

\begin{document}

\encabezado

\section{Algoritmo Aglomerativo}

\begin{algorithm}[H]
   
    \caption{Pseudocódigo del agrupamiento jerárquico aglomerativo (AHC)}
    
    \Input{$D$ (conjunto de datos individuales)}
    \Output{Estructura jerárquica de clústeres}
    
    Inicializar $\ell \leftarrow 0$\;
    Inicializar el conjunto de clústeres $C_\ell \leftarrow D$\;
    
    \While{$|C_\ell| > 1$}{
        Encontrar par $c_i, c_j \in C_\ell$ tal que $L(c_i, c_j)$ sea mínima\;
        Crear nuevo clúster $c_* \leftarrow c_i \cup c_j$\;
        Actualizar $C_{\ell+1} \leftarrow (C_\ell \setminus \{c_i, c_j\}) \cup \{c_*\}$\;
        $\ell \leftarrow \ell + 1$\;
    }
    
    \Return{$C_\ell$}
\end{algorithm}

\subsection{Tipos de enlace}

\begin{defi}[Función de enlace]
    Una función de enlace es una función
    \[
        \func{L}{\mathcal{P}(X) \times \mathcal{P}(X)}{\mathbb{R}}
    \]
    que asigna una distancia entre dos clústeres $A$ y $B$, dada una distancia $d$ entre puntos de $X$.
\end{defi}

\begin{defi}[Enlace simple]
    La distancia entre dos clústeres es la distancia mínima entre los puntos de ambos clústeres. Es decir, dados $A$ y $B$ dos conjuntos, la distancia entre ellos es:
    \[
        L(A,B) = \min_{x \in A, y \in B} d(x,y)
    \]
\end{defi}

\begin{defi}[Enlace completo]
    La distancia entre dos clústeres es la distancia máxima entre los puntos de ambos clústeres. Es decir, dados $A$ y $B$ dos conjuntos, la distancia entre ellos es:
    \[
        L(A,B) = \max_{x \in A, y \in B} d(x,y)
    \]
\end{defi}

\begin{defi}[Enlace promedio]
    La distancia entre dos clústeres es el promedio de las distancias entre los puntos de ambos clústeres. Es decir, dados $A$ y $B$ dos conjuntos, la distancia entre ellos es:
    \[
        L(A,B) = \frac{1}{|A||B|} \sum_{x \in A} \sum_{y \in B} d(x,y)
    \]  
\end{defi}

\begin{defi}[Enlace de centroide]
    La distancia entre dos clústeres es la distancia entre los centroides de ambos clústeres. Es decir, dados $A$ y $B$ dos conjuntos, la distancia entre ellos es:
    \[
        L(A,B) = d\left( \frac{1}{|A|} \sum_{x \in A} x, \frac{1}{|B|} \sum_{y \in B} y \right)
    \]
\end{defi}

\section{Algoritmo Divisivo}

\begin{algorithm}[H]
   
    \caption{Pseudocódigo del agrupamiento jerárquico divisivo (DHC)}
    
    \Input{$D$ (conjunto de datos individuales)}
    \Output{Estructura jerárquica de clústeres}
    
    Inicializar $\ell \leftarrow 0$\;
    Inicializar el conjunto de clústeres $C_\ell \leftarrow \{D\}$\;
    
    \While{exista $c \in C_\ell$ que se pueda dividir}{
        Seleccionar clúster $c \in C_\ell$ para dividir\;
        Dividir $c$ en subclústeres $c_1, c_2, \ldots, c_k$\;
        Actualizar $C_{\ell+1} \leftarrow (C_\ell \setminus \{c\}) \cup \{c_1, c_2, \ldots, c_k\}$\;
        $\ell \leftarrow \ell + 1$\;
    }
    
    \Return{$C_\ell$}
\end{algorithm}

\end{document}