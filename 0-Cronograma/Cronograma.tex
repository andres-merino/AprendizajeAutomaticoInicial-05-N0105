\documentclass[a4,11pt]{aleph-notas}
% Se puede ver la documentación aquí: 
% https://github.com/alephsub0/LaTeX_aleph-notas

% -- Paquetes adicionales 
\usepackage{enumitem}
\usepackage{array,booktabs,multirow,makecell}
\usepackage{colortbl}
\usepackage{longtable}
\usepackage{url}
\usepackage{pdflscape}

% -- Datos 
\institucion{Escuela de Ciencias Físicas y Matemática}
\carrera{Ciencia de Datos}
\asignatura{Aprendizaje Automático Inicial}
\tema{Cronograma y actividades}
\autor{Andrés Merino}
\fecha{Semestre 2024-2}

\logouno[0.14\textwidth]{Logos/logoPUCE_04_ac}
\definecolor{colortext}{HTML}{0030A1}
\definecolor{colordef}{HTML}{0030A1}
\fuente{montserrat}


% -- Comandos para tablas
\newcolumntype{C}[1]{>{\hspace{0pt}\centering\arraybackslash}p{#1}}
\newcolumntype{L}[1]{>{\raggedright\arraybackslash}p{#1}}

\definecolor{verde}{RGB}{0, 255, 127}
\definecolor{celeste}{RGB}{68,195,218}

\begin{document}
\addtolength{\headheight}{1.8\baselineskip}
\addtolength{\voffset}{-1.5\baselineskip}

\encabezado

%%%%%%%%%%%%%%%%%%%%%%%%%%%%%%%%%%%%%%%%
\section{Resultados de aprendizaje} 
%%%%%%%%%%%%%%%%%%%%%%%%%%%%%%%%%%%%%%%%

\begin{itemize}[leftmargin=*]
\item 
    \textbf{RdA 1:} Plantear los conceptos fundamentales del aprendizaje automático, incluyendo los principios básicos, técnicas de preprocesado de datos, métodos de evaluación y ajuste de modelos, destacando su importancia en el análisis y resolución de problemas de datos.
    \begin{itemize}[leftmargin=*]
        \item \textbf{Criterio 1.1:} Identifica los conceptos básicos del aprendizaje automático, incluyendo las técnicas de preprocesado de datos, validación y evaluación de modelos.
        \item \textbf{Criterio 1.2:} Describe los métodos de aprendizaje no supervisado, sus características, aplicaciones, alcance y limitaciones.
        \item \textbf{Criterio 1.3:} Explica los métodos de aprendizaje supervisado, sus características, aplicaciones, alcance y limitaciones.
    \end{itemize}
\item 
    \textbf{RdA 2:} Aplicar modelos de aprendizaje automático supervisado y no supervisado, así como su validación y optimización, en la resolución de problemas tanto reales como simulados.
    \begin{itemize}[leftmargin=*]
        \item \textbf{Criterio 2.1:} Emplea modelos de aprendizaje no supervisado, realizando un análisis crítico de su rendimiento y aplicabilidad en diferentes contextos.
        \item \textbf{Criterio 2.2:} Desarrolla modelos de aprendizaje supervisado, optimizando sus hiperparámetros utilizando técnicas de validación y evaluación.
    \end{itemize}
\item
    \textbf{RdA 3:} Resolver problemas prácticos mediante el uso de modelos de aprendizaje automático, ajustándolos para la mejora de su rendimiento y precisión.
    \begin{itemize}[leftmargin=*]
        \item \textbf{Criterio 3.1:} Aplica modelos de aprendizaje no supervisado en casos prácticos complejos, analizando los resultados y proponiendo mejoras basadas en métricas de rendimiento.
        \item \textbf{Criterio 3.2:} Aplica modelos de aprendizaje supervisado en escenarios del mundo real, ajustando los modelos para maximizar su precisión y eficiencia mediante técnicas de ajuste de hiperparámetros y regularización.
    \end{itemize}
\end{itemize}

%%%%%%%%%%%%%%%%%%%%%%%%%%%%%%%%%%%%%%%%
\section{Contenidos generales} 
%%%%%%%%%%%%%%%%%%%%%%%%%%%%%%%%%%%%%%%%

\begin{itemize}
\item 
    Introducción al Aprendizaje Automático
\item 
    Preprocesamiento de Datos
\item 
    Métodos de Evaluación y Validación de Modelos
\item 
    Aprendizaje Supervisado
\item 
    Aprendizaje No Supervisado
\item 
    Ajuste y Optimización de Modelos:
\end{itemize}

%%%%%%%%%%%%%%%%%%%%%%%%%%%%%%%%%%%%%%%%
\section{Actividades de evaluación} 
%%%%%%%%%%%%%%%%%%%%%%%%%%%%%%%%%%%%%%%%

\begin{itemize}[leftmargin=*]
    \item \textbf{Criterio 1.1}
        \begin{itemize}[leftmargin=*]
            \item \textbf{Cuestionario en Línea 1 (100\%):} Evaluará la comprensión de los conceptos básicos del aprendizaje automático, incluyendo las técnicas de preprocesado de datos, validación y evaluación de modelos, a través de preguntas de opción múltiple y preguntas de desarrollo cortas.
        \end{itemize}
    \item \textbf{Criterio 1.2}
        \begin{itemize}[leftmargin=*]
            \item \textbf{Cuestionario en Línea 2 (50\%):} Evaluará la capacidad de describir los métodos de aprendizaje no supervisado, sus características, aplicaciones, alcance y limitaciones, mediante preguntas de opción múltiple y análisis de casos.
            \item \textbf{Video exposición 1 (50\%):} Consistirá en una presentación grabada donde los estudiantes explicarán un método de aprendizaje no supervisado y analizarán su aplicabilidad en distintos escenarios.
        \end{itemize}
    \item \textbf{Criterio 1.3}
        \begin{itemize}[leftmargin=*]
            \item \textbf{Cuestionario en Línea 3 (50\%):} Evaluará la capacidad de explicar los métodos de aprendizaje supervisado, sus características, aplicaciones, alcance y limitaciones, mediante preguntas de opción múltiple y de desarrollo.
            \item \textbf{Video exposición 2 (50\%):} Los estudiantes presentarán un método de aprendizaje supervisado en un video de análisis crítico, abordando sus ventajas y limitaciones en situaciones reales.
        \end{itemize}
    \item \textbf{Criterio 2.1}
        \begin{itemize}[leftmargin=*]
            \item \textbf{Examen 1 (100\%):} Evaluará la capacidad de emplear modelos de aprendizaje no supervisado y realizar un análisis crítico de su rendimiento y aplicabilidad en diferentes contextos, mediante ejercicios prácticos y teóricos.
        \end{itemize}
    \item \textbf{Criterio 2.2}
        \begin{itemize}[leftmargin=*]
            \item \textbf{Examen 2 (100\%):} Evaluará el desarrollo y optimización de modelos de aprendizaje supervisado, con un enfoque en la optimización de hiperparámetros utilizando técnicas de validación y evaluación.
        \end{itemize}
    \item \textbf{Criterio 3.1}
        \begin{itemize}[leftmargin=*]
            \item \textbf{Reto 1 (100\%):} Consistirá en la aplicación de modelos de aprendizaje no supervisado en un caso práctico complejo. Los estudiantes deberán analizar los resultados y proponer mejoras basadas en métricas de rendimiento.
        \end{itemize}
    \item \textbf{Criterio 3.2}
        \begin{itemize}[leftmargin=*]
            \item \textbf{Reto 2 (100\%):} Los estudiantes aplicarán modelos de aprendizaje supervisado en un escenario del mundo real, ajustando los modelos para maximizar su precisión y eficiencia mediante el ajuste de hiperparámetros y regularización.
        \end{itemize}
\end{itemize}


\begin{landscape}
%%%%%%%%%%%%%%%%%%%%%%%%%%%%%%%%%%%%%%%%
\section{Cronograma de Desarrollo del Curso} %%%%%%%%%%%%%%%%%%%%%%%%%%%%%%%%%%%%%%%%

\begin{center}\small
\setlength{\extrarowheight}{0ex}
\setlength{\belowrulesep}{.6ex}
\begin{longtable}{cccL{12cm}L{7.5cm}}
    \toprule
    &&\thead{Fecha}&\thead{Detalle de contenido} & \thead{Observación} \\
    \midrule
  \endfirsthead
    \multicolumn{5}{l}{\footnotesize \ldots viene de la página anterior}\\
    \toprule
    &&\thead{Fecha}&\thead{Detalle de contenido} & \thead{Observación} \\
    \midrule
  \endhead
        \bottomrule  \multicolumn{5}{r}{\footnotesize Continúa en la siguiente página\ldots}
  \endfoot
        \bottomrule
  \endlastfoot
1	&	1	&	25-nov	&	Introducción al Aprendizaje Automático	&		\\	
	&	2	&	26-nov	&	Conceptos básicos del Aprendizaje Automático	&		\\	
	&	3	&	27-nov	&	Conjuntos de Entrenamiento y Validación	&		\\	
	&	4	&	28-nov	&	Preparación de los Datos	&		\\ \midrule	
2	&	5	&	02-dic	&	Evaluación de Modelos I	&		\\	
	&	6	&	03-dic	&	Evaluación de Modelos II	&		\\	
	&	7	&	04-dic	&	Reducción de Dimensionalidad y Extracción de Características	&		\\	
	&	8	&	05-dic	&	Taller de implementación	&		\\ \midrule	\rowcolor{celeste!50}
3	&	9	&	09-dic	&	Cuestionario en línea 1 (Criterio 1.1)	&	Evaluación	\\	
	&	10	&	10-dic	&	Introducción al Aprendizaje No Supervisado	&	Envío de la Video exp. 1 (Criterio 1.2)	\\	
	&	11	&	11-dic	&	Algoritmos de Agrupamiento Jerárquico	&		\\	
	&	12	&	12-dic	&	Agrupamiento k-Means	&		\\ \midrule	
4	&	13	&	16-dic	&	Taller de implementación	&	Envío del Reto 1 (Criterio 3.1)	\\	
	&	14	&	17-dic	&	Otros Métodos de Clustering	&		\\	
	&	15	&	18-dic	&	Taller de implementación	&	Entrega de la Video exp. 1 (Criterio 1.2)	\\	
	&	16	&	19-dic	&	Desarrollo del Reto 1	&	Entrega del Reto 1 (Criterio 3.1)	\\ \midrule	\rowcolor{celeste!50}
	&	17	&	23-dic	&	Cuestionario en línea 2 (Criterio 1.2); Examen 1 (Criterio 2.1)	&	Evaluación	\\	
	&	18	&	24-dic	&	Introducción al Aprendizaje Supervisado	&	Envío de la Video exp. 2 (Criterio 1.3)	\\	\rowcolor{verde!50}
	&		&	25-dic	&		&	Feriado	\\	\rowcolor{verde!50}
	&		&	26-dic	&		&	Feriado	\\ \midrule	\rowcolor{verde!50}
	&		&	30-dic	&		&	Feriado	\\	\rowcolor{verde!50}
	&		&	31-dic	&		&	Feriado	\\	\rowcolor{verde!50}
	&		&	01-ene	&		&	Feriado	\\	
	&	19	&	02-ene	&	Algoritmo k-Nearest Neighbors	&		\\ \midrule	
6	&	20	&	06-ene	&	Máquinas de Vectores Soporte (SVM)	&		\\	
	&	21	&	07-ene	&	Ajuste y Optimización de SVM	&		\\	
	&	22	&	08-ene	&	Taller de implementación	&	Entrega de la Video exp. 2 (Criterio 1.3)	\\	
	&	23	&	09-ene	&	Redes Neuronales: Introducción	&		\\ \midrule	
7	&	24	&	13-ene	&	Aprendizaje en Redes Neuronales	&		\\	
	&	25	&	14-ene	&	Taller de implementación	&	Envío del Reto 2 (Criterio 3.2)	\\	
	&	26	&	15-ene	&	Redes Neuronales Profundas	&		\\	
	&	27	&	16-ene	&	Árboles de Decisión	&		\\ \midrule	
8	&	28	&	20-ene	&	Bosques Aleatorios	&		\\	
	&	29	&	21-ene	&	Taller de implementación	&		\\	\rowcolor{celeste!50}
	&	30	&	22-ene	&	Cuestionario en línea 3 (Criterio 1.3); Examen 2 (Criterio 2.2)	&	Evaluación	\\	
	&	31	&	23-ene	&	Evaluación de Modelos: Validación cruzada	&		\\ \midrule	
9	&	32	&	27-ene	&	Búsqueda de Hiperparámetros	&		\\	
	&	33	&	28-ene	&	Taller de implementación	&		\\	
	&	34	&	29-ene	&	Desarrollo del Reto 2	&	Entrega del Reto 2 (Criterio 3.2)	\\	
	&	35	&	30-ene	&	Conclusiones y retroalimentación	&		\\ 
\end{longtable}
\end{center}
\end{landscape}

\end{document} 