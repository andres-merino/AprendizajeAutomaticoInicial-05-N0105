\documentclass[a4,11pt]{aleph-notas}

% -- Paquetes adicionales
\usepackage{enumitem}
\usepackage{aleph-comandos}
\hypersetup{urlcolor=blue}

% -- Datos  
\institucion{Facultad de Ciencias Exactas, Naturales y Ambientales}
\carrera{Ciencia de Datos}
\asignatura{Aprendizaje Automático Inicial}
\tema{Clase invertida no. 4: Árboles de decisión}
\autor[A. Merino]{Andrés Merino}
\fecha{Semestre 2024-2}

\logouno[0.14\textwidth]{Logos/logoPUCE_04_ac}
\definecolor{colortext}{HTML}{0030A1}
\definecolor{colordef}{HTML}{0030A1}
\fuente{montserrat}

% -- Comandos adicionales
\begin{document}

\encabezado


\section*{Introducción}

\begin{itemize}
    \item \textbf{Tema:} Árboles de Decisión
    \item \textbf{Resultado de Aprendizaje:} Comprender los conceptos fundamentales de los árboles de decisión, su funcionamiento básico y sus aplicaciones prácticas en problemas de clasificación y regresión.
\end{itemize}

\section*{Lección en casa}

\subsection*{Actividades}

\begin{enumerate}[leftmargin=*]
    \item Interactuar con ChatGPT mediante los siguientes \textit{prompts}, leyendo detenidamente el \textit{prompt} y su respuesta:
    \begin{enumerate}[label=\textit{Prompt \arabic*.},leftmargin=2.1cm]
        \item Vas a ser mi profesor de Árboles de Decisión, te daré instrucciones y me explicarás de manera clara y formal lo que te pida. Quiero que uses ejemplos simples y visualizaciones conceptuales si es necesario. Sé ameno y paciente. Usa ejemplos prácticos cotidianos ecuatorianos. ¿Entendido?
        \item ¿Qué es un árbol de decisión y cómo funciona? Explica de manera sencilla, usando ejemplos como clasificar licores.
        \item Describe el proceso de construcción de un árbol de decisión. Incluye cómo se eligen las características y cómo se divide el conjunto de datos.
    \end{enumerate}
    \item Visualiza el siguiente video: \href{https://youtu.be/7VeUPuFGJHk?si=LQzPfVSBrvLKbHAM}{StatQuest: Decision Trees}.
    \item Continúa la interacción con ChatGPT:
    \begin{enumerate}[label=\textit{Prompt \arabic*.},leftmargin=2.1cm, start=4]
        \item ¿Qué es el criterio de Gini y cómo se utiliza para dividir datos en un árbol de decisión?
        \item ¿Qué significa «profundidad del árbol» y cómo afecta al rendimiento del modelo? Da ejemplos prácticos.
    \end{enumerate}
    \item Lee la sección 9.1 del artículo: \href{https://link.springer.com/article/10.1140/epjs/s11734-021-00209-7}{Beginning with machine learning: a comprehensive primer}
    \item Continúa la interacción con ChatGPT:
    \begin{enumerate}[label=\textit{Prompt \arabic*.},leftmargin=2.1cm, start=6]
        \item Explica de manera simple cómo evitar el sobreajuste en árboles de decisión. Usa ejemplos como clasificar animales.
    \end{enumerate}
    \item Continúa la interacción con ChatGPT con preguntas relacionadas con los videos que viste y al artículo que leíste. Plantea al menos una pregunta.
    \item Realiza el cuestionario del aula virtual.
\end{enumerate}



\end{document}