\documentclass[a4,11pt]{aleph-notas}

% -- Paquetes adicionales
\usepackage{enumitem}
\usepackage{aleph-comandos}
\hypersetup{urlcolor=blue}

% -- Datos  
\institucion{Escuela de Ciencias Físicas y Matemática}
\carrera{Ciencia de Datos}
\asignatura{Aprendizaje Automático Inicial}
\tema{Clase invertida no. 2: Introducción al Aprendizaje Supervisado}
\autor[A. Merino]{Andrés Merino}
\fecha{Semestre 2024-2}

\logouno[0.14\textwidth]{Logos/logoPUCE_04_ac}
\definecolor{colortext}{HTML}{0030A1}
\definecolor{colordef}{HTML}{0030A1}
\fuente{montserrat}

% -- Comandos adicionales
\begin{document}

\encabezado

\section*{Introducción}

\begin{itemize}
    \item \textbf{Tema:} Introducción al aprendizaje supervisado.
    \item \textbf{Resultado de Aprendizaje:} Identificar y clasificar diferentes modelos de aprendizaje supervisado para tareas de regresión y clasificación, explicando sus principales aplicaciones y características.
\end{itemize}

\section*{Lección en casa}

\subsection*{Actividades}

\begin{enumerate}[leftmargin=*]
    \item Interactuar con ChatGPT mediante los siguientes \textit{prompts}, leyendo detenidamente el \textit{prompt} y su respuesta:
    \begin{enumerate}[label=\textit{Prompt \arabic*.},leftmargin=2.1cm]
        \item Vas a ser mi profesor de la asignatura de Aprendizaje Automático, te daré instrucciones y me explicarás de manera clara y formal lo que te pida. Quiero que seas muy preciso con los conceptos matemáticos, pero también que uses ejemplos simples para ilustrar los conceptos si es necesario. Sé ameno y paciente. ¿Entendido?
        \item ¿Qué es el aprendizaje supervisado y cómo se diferencia del aprendizaje no supervisado? Usa ejemplos prácticos.
        \item Dame ejemplos de aplicaciones reales donde se utiliza aprendizaje supervisado, tanto para regresión como para clasificación.
    \end{enumerate}
    \item Visualiza el siguiente video: \href{https://youtu.be/ijJYeLFtiVM?si=_ye7QDqMGJc9lvB8}{Aplicaciones prácticas del aprendizaje supervisado}.
    \item Continúa la interacción con ChatGPT:
    \begin{enumerate}[label=\textit{Prompt \arabic*.},leftmargin=2.1cm, start=4]
        \item Enumera y describe brevemente los modelos comunes para regresión en aprendizaje supervisado (e.g., regresión lineal, regresión polinómica, árboles de decisión, etc.).
        \item Enumera y describe brevemente los modelos comunes para clasificación en aprendizaje supervisado (e.g., regresión logística, SVM, KNN, árboles de decisión, etc.).
        \item Dame un ejemplo de cómo seleccionar el modelo más adecuado según el tipo de problema (regresión o clasificación).
    \end{enumerate}
    \item Continúa la interacción con ChatGPT con las preguntas sobre el video que viste, plantea al menos una pregunta.
    \item Realiza el cuestionario del aula virtual.
\end{enumerate}



\end{document}