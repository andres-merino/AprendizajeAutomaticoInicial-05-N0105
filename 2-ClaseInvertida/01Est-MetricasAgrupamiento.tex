\documentclass[a4,11pt]{aleph-notas}

% -- Paquetes adicionales
\usepackage{enumitem}
\usepackage{aleph-comandos}
\hypersetup{urlcolor=blue}

% -- Datos  
\institucion{Escuela de Ciencias Físicas y Matemática}
\carrera{Ciencia de Datos}
\asignatura{Aprendizaje Automático Inicial}
\tema{Clase invertida no. 1: Métricas de modelos de agrupamiento}
\autor[A. Merino]{Andrés Merino}
\fecha{Semestre 2024-2}

\logouno[0.14\textwidth]{Logos/logoPUCE_04_ac}
\definecolor{colortext}{HTML}{0030A1}
\definecolor{colordef}{HTML}{0030A1}
\fuente{montserrat}

% -- Comandos adicionales
\begin{document}

\encabezado

\section*{Introducción}

\begin{itemize}
    \item \textbf{Tema:} Métricas de modelos de agrupamiento
    \item \textbf{Resultado de Aprendizaje:} Explicar, calcular y analizar métricas de calidad de partición como diámetro y separación en modelos de agrupamiento.
\end{itemize}

\section*{Lección en casa}

\subsection*{Actividades}

\begin{enumerate}[leftmargin=*]
    \item Interactuar con ChatGPT mediante los siguientes \textit{prompts}, leyendo detenidamente el \textit{prompt} y su respuesta:
    \begin{enumerate}[label=\textit{Prompt \arabic*.},leftmargin=2.1cm]
        \item Vas a ser mi profesor de la asignatura de Aprendizaje Automático, te daré instrucciones y me explicarás de manera clara y formal lo que te pida. Quiero que seas muy preciso con los conceptos matemáticos, pero también que uses ejemplos simples para ilustrar los conceptos si es necesario. Sé ameno y paciente. ¿Entendido?
        \item ¿Qué son las métricas de calidad de partición en un modelo de agrupamiento? Explícalo en términos generales sin entrar en detalles matemáticos aún.
        \item Explícame qué es el diámetro de un grupo en un modelo de agrupamiento. Usa una definición matemática simple.
        \item Ahora explícame qué significa la separación entre dos grupos. Define este concepto de manera matemática.
        \item ¿Por qué son importantes el diámetro y la separación para evaluar un modelo de agrupamiento? Dime cómo se relacionan con la calidad del agrupamiento.
        \item Dame un ejemplo general de cómo calcular el diámetro de un grupo y la separación entre dos grupos usando conceptos teóricos. No utilices datos numéricos todavía.
        \item Ahora sí, dame un ejemplo numérico donde calcules el diámetro y la separación usando datos de puntos en un espacio bidimensional.
    \end{enumerate}
    \item Visualiza el siguiente video: \href{https://youtu.be/b920s9nXGao?si=es5xjKYBKtfcdeTM}{¿Qué tan buenos son tus Clusters?}.
    \item Continúa la interacción con ChatGPT con las preguntas sobre el video que acabas de ver.
    \item Realiza el cuestionario del aula virtual.
\end{enumerate}

\end{document}