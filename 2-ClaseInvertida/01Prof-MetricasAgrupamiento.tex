\documentclass[a4,11pt]{aleph-notas}

% -- Paquetes adicionales
\usepackage{enumitem}
\usepackage{aleph-comandos}
\usepackage{aleph-moodle}

% -- Datos   
\institucion{Escuela de Ciencias Físicas y Matemática}
\carrera{Ciencia de Datos}
\asignatura{Aprendizaje Automático Inicial}
\tema{Clase invertida no. 1: Métricas de modelos de agrupamiento}
\autor[A. Merino]{Andrés Merino}
\fecha{Semestre 2024-2}

\logouno[0.14\textwidth]{Logos/logoPUCE_04_ac}
\definecolor{colortext}{HTML}{0030A1}
\definecolor{colordef}{HTML}{0030A1}
\fuente{montserrat}

% -- Comandos adicionales
\usepackage{listings}
\input{listings-python.prf}

\usepackage[spanish,onelanguage,vlined,linesnumbered]{algorithm2e}

% -- Comandos adicionales
\newtcolorbox{pscodigo}
    {icono=\faCogs,color=lightgray,postit,top=-1.5mm,bottom=-1.5mm}
    
\definecolor{colcod}{RGB}{174,218,255}
\newtcolorbox{pycodigo}
    {icono=\faKeyboardO, color=colcod, postit, 
    top=-2mm, bottom=-2mm, 
    extras first={bottom=0mm},
    extras last={top=0mm},
    extras middle={top=0mm,bottom=0mm},
    }

\lstloadlanguages{Python}
\lstset{
  language=Python,
  basicstyle=\small\sffamily,
  stringstyle=\color[HTML]{933797},
  commentstyle=\color[HTML]{228B22}\sffamily,
  emph={[2]from,import,pass,return}, emphstyle={[2]\color[HTML]{DD52F0}},
  emph={[3]range}, emphstyle={[3]\color[HTML]{D17032}},
  emph={[4]for,in,def}, emphstyle={[4]\color{blue}},
  showstringspaces=false,
  breaklines=true,
  prebreak=\mbox{{\color{gray}\tiny$\searrow$}},
  xleftmargin=3pt,
  inputencoding=utf8,
  extendedchars=true,
  columns=fullflexible,
  literate={á}{{\'a}}1 {é}{{\'e}}1 {í}{{\'i}}1 {ó}{{\'o}}1 {ú}{{\'u}}1,
}


\SetKwFunction{concat}{Concatenar}
\SetKwProg{Fn}{Función}{\string:}{}
\SetKwFunction{ult}{Ultimo}
\SetKwFunction{pri}{Primero}
\SetKwFunction{sinul}{SinUltimo}
\SetKw{Salir}{Salir}

\newcommand{\fuentecomentario}[1]{\scriptsize\ttfamily #1}
\SetCommentSty{fuentecomentario}
\SetAlFnt{\footnotesize}



\begin{document}

\encabezado

\vspace*{-10mm}
\section*{Introducción}

\begin{itemize}
    \item \textbf{Tema:} Métricas de modelos de agrupamiento
    \item \textbf{Resultado de Aprendizaje:} Explicar, calcular y analizar métricas de calidad de partición como diámetro y separación en modelos de agrupamiento.
\end{itemize}

\section{Lección en casa}

\subsection{Adquisición de concepto}

Para la adquisición del concepto, se solicitará al estudiante interactuar con ChatGPT mediante la siguiente serie de indicaciones:

\begin{enumerate}[leftmargin=*]
    \item Interactuar con ChatGPT mediante los siguientes \textit{prompts}, leyendo detenidamente el \textit{prompt} y su respuesta:
    \begin{enumerate}[label=\textit{Prompt \arabic*.},leftmargin=2.1cm]
        \item Vas a ser mi profesor de la asignatura de Aprendizaje Automático, te daré instrucciones y me explicarás de manera clara y formal lo que te pida. Quiero que seas muy preciso con los conceptos matemáticos, pero también que uses ejemplos simples para ilustrar los conceptos si es necesario. Sé ameno y paciente. ¿Entendido?
        \item ¿Qué son las métricas de calidad de partición en un modelo de agrupamiento? Explícalo en términos generales sin entrar en detalles matemáticos aún.
        \item Explícame qué es el diámetro de un grupo en un modelo de agrupamiento. Usa una definición matemática simple.
        \item Ahora explícame qué significa la separación entre dos grupos. Define este concepto de manera matemática.
        \item ¿Por qué son importantes el diámetro y la separación para evaluar un modelo de agrupamiento? Dime cómo se relacionan con la calidad del agrupamiento.
        \item Dame un ejemplo general de cómo calcular el diámetro de un grupo y la separación entre dos grupos usando conceptos teóricos. No utilices datos numéricos todavía.
        \item Ahora sí, dame un ejemplo numérico donde calcules el diámetro y la separación usando datos de puntos en un espacio bidimensional.
    \end{enumerate}
    \item Visualiza el siguiente video: \href{https://youtu.be/b920s9nXGao?si=es5xjKYBKtfcdeTM}{¿Qué tan buenos son tus Clusters?}.
    \item Continúa la interacción con ChatGPT con las preguntas sobre el video que acabas de ver.
    \item Realiza el cuestionario del aula virtual.
\end{enumerate}

\subsection{Personalización de la actividad}

Se la consigue en la interacción personal de cada estudiante con ChatGPT, solicitando también continuar la interacción hasta que se sienta preparado en el tema.

\subsection{Solventación de dudas}

En caso de tener dudas sobre el tema, se solicitará al estudiante interactuar con ChatGPT para solventarlas.

\subsection{Micro-tarea}

Para realizar un seguimiento de la actividad, se solicitará al estudiante copiar el enlace del chat como evidencia del proceso. Adicionalmente, se le pedirá realizar el cuestionario del aula virtual. El cuestionario se encuentra detallado en el Anexo.

\section{Tareas en clase}

\subsection{Visión conjunta}

Se muestra la relación entre las actividades realizadas en casa y las tareas a realizar en clase. De manera específica, cómo el cálculo de determinantes se relaciona con sus diferentes propiedades y con la inversión de matrices.

\subsection{Retroalimentación}

Se brinda retroalimentación a los estudiantes sobre las respuestas dadas en la micro-tarea.

\subsection{Actividad de aplicación}

Se utilizará el siguiente código base para calcular métricas en modelos de agrupamiento: \href{https://colab.research.google.com/github/andres-merino/AprendizajeAutomaticoInicial-05-N0105/blob/main/2-Notebooks/06-Evaluacion-de-Modelos-II.ipynb}{06-Evaluacion-de-Modelos-II.ipynb}


\subsection{Micro-evaluación}

No se realizará micro-evaluación en esta clase.


\section*{Anexo}

\begin{quiz}{Metrica Cluster}

\begin{multi}
    {MetricaCluster01}
    ¿Qué es un cluster en el contexto de análisis de datos?
    \item Un punto de datos que se utiliza para realizar cálculos estadísticos.
    \item* Un agrupamiento natural de datos que comparten similitudes.
    \item Un algoritmo utilizado para clasificar datos en categorías.
    \item Una métrica para evaluar la calidad de una visualización.
\end{multi}

\begin{multi}
    {MetricaCluster02}
    ¿Qué mide el coeficiente de silueta en el análisis de clusters?
    \item El número total de clusters en los datos.
    \item* Qué tan compactos están los puntos dentro de un cluster y qué tan separados están los clusters entre sí.
    \item La distancia promedio entre todos los puntos del dataset.
    \item El rendimiento computacional del algoritmo utilizado.
\end{multi}

\begin{multi}
    {MetricaCluster03}
    ¿Cuál de las siguientes afirmaciones describe correctamente el índice Davies-Bouldin?
    \item Es una métrica que evalúa la separación de los clusters únicamente.
    \item* Es una métrica que mide la cohesión y separación de los clusters, donde valores cercanos a 0 indican una mejor clusterización.
    \item Evalúa exclusivamente la cantidad de datos en cada cluster.
    \item Se utiliza para calcular el número óptimo de clusters en el método K-Medias.
\end{multi}

\begin{multi}
    {MetricaCluster04}
    ¿Cuál es la principal diferencia entre los algoritmos K-Medias y Propagación de Afinidad?
    \item* K-Medias requiere especificar el número de clusters como entrada, mientras que Propagación de Afinidad los determina automáticamente.
    \item K-Medias clasifica datos basándose en probabilidades, mientras que Propagación de Afinidad utiliza métricas geométricas.
    \item Propagación de Afinidad requiere más memoria que K-Medias para procesar grandes datasets.
    \item K-Medias funciona solo con datos numéricos, mientras que Propagación de Afinidad trabaja con datos categóricos.
\end{multi}

\end{quiz}


\begin{quiz}{Clase Invertida}
    
\begin{essay}[response format=text, response field lines=5]%
    % - Identificador
    {ClaseInvertida-Chat}
    % - Enunciado
    Copia el enlace del chat con ChatGPT como evidencia de la actividad realizada en casa.
    \item Acceder al enlace.
\end{essay}

\begin{essay}[response format=text, response field lines=5]%
    % - Identificador
    {ClaseInvertida-Dudas}
    % - Enunciado
    ¿Qué dudas tuviste sobre el tema cubierto en la actividad?
    \item Solo para registro.
\end{essay}



\end{quiz}


\end{document}