\documentclass[a4,11pt]{aleph-notas}

% -- Paquetes adicionales
\usepackage{enumitem}
\usepackage{aleph-comandos}
\hypersetup{urlcolor=blue}

% -- Datos  
\institucion{Escuela de Ciencias Físicas y Matemática}
\carrera{Ciencia de Datos}
\asignatura{Aprendizaje Automático Inicial}
\tema{Clase invertida no. 3: Introducción a Redes Neuronales}
\autor[A. Merino]{Andrés Merino}
\fecha{Semestre 2024-2}

\logouno[0.14\textwidth]{Logos/logoPUCE_04_ac}
\definecolor{colortext}{HTML}{0030A1}
\definecolor{colordef}{HTML}{0030A1}
\fuente{montserrat}

% -- Comandos adicionales
\begin{document}

\encabezado


\section*{Introducción}

\begin{itemize}
    \item \textbf{Tema:} Introducción a Redes Neuronales.
    \item \textbf{Resultado de Aprendizaje:} Comprender los conceptos fundamentales de las redes neuronales, su funcionamiento básico y sus aplicaciones prácticas en problemas de aprendizaje supervisado.
\end{itemize}

\section*{Lección en casa}

\subsection*{Actividades}

\begin{enumerate}[leftmargin=*]
    \item Interactuar con ChatGPT mediante los siguientes \textit{prompts}, leyendo detenidamente el \textit{prompt} y su respuesta:
    \begin{enumerate}[label=\textit{Prompt \arabic*.},leftmargin=2.1cm]
        \item Vas a ser mi profesor de Redes Neuronales, te daré instrucciones y me explicarás de manera clara y formal lo que te pida. Quiero que uses ejemplos simples y visualizaciones conceptuales si es necesario. Sé ameno y paciente. Usa tono navideño ¿Entendido?
        \item ¿Qué es una red neuronal artificial y cuál es su inspiración en la biología? Explica de manera sencilla.
        \item Describe el funcionamiento básico de una red neuronal de una sola capa (perceptrón simple). Incluye un ejemplo práctico navideño.
    \end{enumerate}
    \item Visualiza el siguiente video: \href{https://youtu.be/MRIv2IwFTPg?si=6ocz4bs1Wi5SkeSz}{¿Qué es una Red Neuronal? (Dot CSV)}
    \item Continúa la interacción con ChatGPT:
    \begin{enumerate}[label=\textit{Prompt \arabic*.},leftmargin=2.1cm, start=4]
        \item ¿Qué son las funciones de activación y por qué son importantes en una red neuronal? Da ejemplos de funciones de activación comunes.
        \item ¿Cuántos parámetros suele tener una red neuronal? Dame dos ejemplos navideños.
    \end{enumerate}
    \item Visualiza el siguiente video: \href{https://youtu.be/jKCQsndqEGQ?si=2hLeG20lU4I5ff44}{¿Qué es una Red Neuronal? (3Blue1Brown)}
    \item Lee las secciones 11.1, 11.2 y 11.3 del artículo: \href{https://link.springer.com/article/10.1140/epjs/s11734-021-00209-7}{Beginning with machine learning: a comprehensive primer}
    \item Continúa la interacción con ChatGPT:
    \begin{enumerate}[label=\textit{Prompt \arabic*.},leftmargin=2.1cm, start=6]
        \item Explica de manera supersimple cómo se entrena una red neuronal usando el descenso del gradiente. Usa términos básicos y un ejemplo navideño.
    \end{enumerate}
    \item Visualiza el siguiente video: \href{https://youtu.be/mwHiaTrQOiI?si=f-0Y43MwfJ-Q3Af7}{Descenso de Gradiente (3Blue1Brown)}
    \item Continúa la interacción con ChatGPT con preguntas relacionadas a los videos que viste y al artículo que leíste. Plantea al menos una pregunta.
    \item Realiza el cuestionario del aula virtual.
\end{enumerate}


\end{document}